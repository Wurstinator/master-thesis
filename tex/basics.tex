\section{Basic Definitions}

\subsection{Sets and Functions}
\begin{defn}
	The \emph{natural numbers} $\mathbb{N} = \{0, 1, 2, \dots\}$ are the set of all non-negative integers. We define $0 := \emptyset$, $1 := \{0\}$, $2 := \{0, 1\}$, and so forth.
	
	The value $\omega$ denotes the \enquote{smallest} infinity, $\omega := \mathbb{N}$. For all natural numbers, we write $n < \omega$ and $\omega \not < \omega$. Also, we sometimes use the convention $n + \omega = \omega$.
	
	We denote the set $\mathbb{N} \cup \{\omega\}$ by $\mathbb{N}_\omega$.
\end{defn}

\begin{defn}
	Let $X$ and $Y$ be two sets. We use the usual definition of union ($\cup$), intersection ($\cap$), and set difference ($\setminus$). If some domain ($X \subseteq D$) is clear in the context, we write $X^\complement = D \setminus X$.
	
	We use the Cartesian product $X \times Y = \{ (x, y) \mid x \in X, y \in Y \}$.
	
	We write $X^Y$ for the set of all functions with domain $Y$ and range $X$. If we have a function $f : D \rightarrow \{0, 1\}$, then we sometimes implicitly use it as a set $X \subseteq D$ with $x \in X$ iff $f(x) = 1$. In particular, $2^Y$ is the power set of $Y$.
\end{defn}

\begin{defn}
	Let $f : D \rightarrow R$ be a function and let $X \subseteq D$ and $Y \subseteq R$. We describe by $f(X) = \{ f(x) \in R \mid x \in X\}$ and $f^{-1}(Y) = \{ x \in D \mid \exists y \in Y: f(x) = y \}$.
\end{defn}

\begin{defn}
	Let $X \subseteq D$ be a set. For $D' \subseteq D$, we define $X \upharpoonright_{D'} = X \cap D$. In particular, we use this notation for relations, e.g. $R \subseteq \mathbb{N} \times \mathbb{N}$ and $R \upharpoonright_{\{0\} \times \mathbb{N}}$.
	
	For a function $f : D \rightarrow R$, we write $f \upharpoonright_{D'}$ for the function $f' : D' \rightarrow R, x \mapsto f(x)$.
\end{defn}


\subsection{Relations and Orders}
\begin{defn}
	Let $X$ be a set. We call a set $R \subseteq X \times X$ a \emph{relation} over $X$. $R$ is
	\begin{itemize}
		\item \emph{reflexive}, if for all $x \in X$, $(x, x) \in R$.
		\item \emph{irreflexive}, if for all $x \in X$, $(x, x) \notin R$.
		\item \emph{symmetric}, if for all $(x, y) \in R$, also $(y, x) \in R$.
		\item \emph{asymmetric}, if for all $(x, y) \in R$, $(y, x) \notin R$.
		\item \emph{transitive}, if for all $(x, y), (y, z) \in R$, also $(x, z) \in R$.
		\item \emph{total}, if for all $x, y \in X$, $(x, y) \in R$ or $(y, x) \in R$ is true.
	\end{itemize}
	
	We call $R$ 
	\begin{itemize}
		\item a \emph{partial order}, if it is irreflexive, asymmetric, and transitive.
		\item a \emph{total order}, if it is a partial order and total.
		\item a \emph{preorder}, if it is reflexive and transitive.
		\item a \emph{total preorder}, if it is a preorder and total.
		\item an \emph{equivalence relation}, if it is a preorder and symmetric.
	\end{itemize}
	
	If $R$ is a partial order or a preorder, we call an element $x \in X$ \emph{minimal} (w.r.t. $R$), if for all $y \in X$, $(y, x) \in R$ implies $(x, y) \in R$. Similarly, we call it \emph{maximal}, if for all $y \in X$, $(x, y) \in R$ implies $(y, x) \in R$. 
	
	We call $x$ the \emph{minimum} of $R$ if for all $y \neq x$, $(y, x) \in R$. We write $x = \min_R X$.
\end{defn}

\begin{defn}
	Let $R$ be a partial order over $X$. We call a set $S \subseteq Y$ an \emph{extension of $R$ to $Y$} if $X \subseteq Y$, $R \subseteq S$, and $S$ is a partial order over $Y$. We use the same notation for total orders, preorders, and total preorders.
\end{defn}

\begin{defn}
	Let $R$ be an equivalence relation over $X$. $R$ implicitly forms a partition of $X$ into \emph{equivalence classes}. For an element $x \in X$, we call $[x]_R := \{ y \in X \mid (x, y) \in R \}$ the equivalence class of $x$. We denote the set of equivalence classes by $\mathfrak{C}(R) = \{ [x]_R \mid x \in R \}$.
\end{defn}



\subsection{Words and Languages}
\begin{defn}
	A non-empty set of symbols can be called an \emph{alphabet}, which we will denote by a variable $\Sigma$ most of the time. As symbols, we usually use lower case letters, i.e. $a$ or $b$.
	
	A \emph{finite word}, usually denoted by $u$, $v$, or $w$, over an alphabet $\Sigma$ is a function $w : n \rightarrow \Sigma$ for some $n$. We call $n$ the \emph{length} of $w$ and write $|w| = n$. The unique word of length $0$ is called \emph{empty word} and is written as $\varepsilon$.
	
	Given $\Sigma^n = \{ w \mid w \text{ is a word of length } n \text{ over } \Sigma \}$, we define $\Sigma^* = \bigcup\limits_{n \in \mathbb{N}} \Sigma^n$ as the set of all finite words over $\Sigma$. 
\end{defn}

\begin{defn}
	An \emph{$\omega$-word}, usually denoted by $\alpha$ or $\beta$, over an alphabet $\Sigma$ is a function \linebreak $\alpha : \omega \rightarrow \Sigma$. $\omega$ is the length of $\alpha$ and we write $|\alpha| = \omega$. The set $\Sigma^\omega$ then describes the set of all $\omega$-words over $\Sigma$. 
\end{defn}

\begin{defn}
	A \emph{language} over an alphabet $\Sigma$ is a set of words $L \subseteq \Sigma^* \cup \Sigma^\omega$. In the context we use it should always be clear whether we are using finite words or $\omega$-words.
\end{defn}

\begin{defn}
	Let $v, w \in \Sigma^*$ and $w_i \in \Sigma^*$ for all $i \in \mathbb{N}$ be words over $\Sigma$ and $\alpha \in \Sigma^\omega$ be an $\omega$-word over $\Sigma$.
	
	The \emph{concatenation} of $v$ and $w$ (denoted by $v \cdot w$) is a word $u$ such that:
	\begin{align*}
	u : |v|+|w| \allowbreak \rightarrow \Sigma, i \mapsto 
	\begin{cases}
		v(i) & \text{if } i < |v| \\
		w(i-|v|) & \text{else}
	\end{cases}
	\end{align*}

	The \emph{concatenation} of $w$ and $\alpha$ (denoted by $w \cdot \alpha$) is an $\omega$-word $\beta$ such that:
	\begin{align*}
	\beta : \mathbb{N} \rightarrow \Sigma, i \mapsto 
	\begin{cases}
		w(i) & \text{if } i < |w| \\
		\alpha(i-|w|) & \text{else}
	\end{cases}
	\end{align*}
	
	For some $n \in \mathbb{N}$, the \emph{$n$-iteration} of $w$ (denoted by $w^n$) is a word $u$ such that:
	\begin{align*}
		u : |w|^n \rightarrow \Sigma, i \mapsto w(i \mod |w|)
	\end{align*}
	
	The \emph{$\omega$-iteration} of $w$ (denoted by $w^\omega$) is an $\omega$-word $\alpha$ such that:
	\begin{align*}
		\beta : \mathbb{N} \rightarrow \Sigma, i \mapsto w(i \mod |w|)
	\end{align*}
\end{defn}

\vspace{20pt}
For the purpose of easier notation and readability, we write singular symbols as words, i.e. for an $a \in \Sigma$ we write $a$ for the word $w_a : \{0\} \rightarrow \Sigma, i \mapsto a$.\\
We also abbreviate $v \cdot w$ to $vw$ and $w \cdot \alpha$ to $w \alpha$. Further, we use $\alpha \cdot \varepsilon = \alpha$ for $\alpha \in \Sigma^\omega$.
\vspace{10pt}

\begin{defn}
	Let $L, K \subseteq \Sigma^*$ be a language and $U \subseteq \Sigma^\omega$ be an $\omega$-language.\\
	The \emph{concatenation} of $L$ and $K$ is $L \cdot K = \{ u \in \Sigma^* \mid \text{There are } v \in L \text{ and } \allowbreak w \in K \text{ such that } u = v \cdot w\}$.\\
	The \emph{concatenation} of $L$ and $U$ is $L \cdot U = \{ \alpha \in \Sigma^\omega \mid \text{There are } w \in L \text{ and } \allowbreak \beta \in U \text{ such that } \alpha = w \cdot \beta\}$.\\
	For some $n \in \mathbb{N}$, the \emph{$n$-iteration} of $L$ is $L^n = \{ w \in \Sigma^* \mid \text{There is } v \in L \text{ such that } w = v^n\}$.\\
	The \emph{Kleene closure} of $L$ is $L^* = \bigcup\limits_{n \in \mathbb{N}} L^n$.\\
\end{defn}

\begin{defn}
	Let $w \in \Sigma^* \cup \Sigma^\omega$ be a word. We define a sub string or sub word of $w$ for some $n \leq m \leq |w|$ as $w[n, m] = w(n) \cdot w(n+1) \cdots w(m-1)$. In the case that $m = |w| = \omega$, it is simply $w[n, m] = w(n) \cdot w(n+1) \cdots$. Note that for $n = m$, we have $w[n, m] = \varepsilon$.
\end{defn}

\begin{defn}
	Let $v, w \in \Sigma^* \cup \Sigma^\omega$ be words. We call $v$ 
	\begin{itemize}
		\item a \emph{prefix} of $w$, if there is an $n \in \mathbb{N}_\omega$ with $v = w[0, n]$.
		\item a \emph{suffix} of $w$, if there is an $n \in \mathbb{N}_\omega$ with $v = w[n, |w|]$.
		\item an \emph{infix} of $w$, if there are $n, m \in \mathbb{N}_\omega$ with $v = w[n, m]$.
	\end{itemize}
\end{defn}

\begin{defn}
	The \emph{occurrence set} of a word $w \in \Sigma^* \cup \Sigma^\omega$ is the set of symbols which occur at least once in $w$. 
	$$\text{Occ}(w) = \{ a \in \Sigma \mid \text{There is an } n \in |w| \text{ such that } w(n) = a \text{.}\}$$
	
	The \emph{infinity set} of a word $w \in \Sigma^\omega$ is the set of symbols which occur infinitely often in $w$.
	$$\text{Inf}(w) = \{ a \in \Sigma \mid \text{For every } n \in \mathbb{N} \text{ there is a } m > n \text{ such that } w(m) = a \text{.}\}$$
\end{defn}

\begin{defn}
	Let $w \in \Sigma^* \cup \Sigma^\omega$ be a word and $f : \Sigma \rightarrow \Gamma$ be a funcction. We define $f(w) \in \Gamma^{|w|}$ as $(f(w))(n) = f(w(n))$.
\end{defn}


\subsection{Automata}

\begin{defn}
	Let $Q$ be a set, $\Sigma$ an alphabet, and $\delta : Q \times \Sigma \rightarrow Q$ a function. We call $\mathcal{S} = (Q, \Sigma, \delta)$ a \emph{deterministic transition structure}. We call $Q$ the states or state space.
	
	For $q \in Q$ and a word $w \in \Sigma^* \cup \Sigma^\omega$, we call $\rho \in Q^{1+|w|}$ the \emph{run} of $\mathcal{S}$ on $w$ starting in $q$ if $\rho(0) = q$ and for all $i$, $\rho(i+1) = \delta(\rho(i), w(i))$.
\end{defn}

\begin{defn}
	Let $\mathcal{S} = (Q, \Sigma, \delta)$ be a deterministic transition structure. We define $\delta^* : Q \times \Sigma^* \rightarrow Q$ as $\delta^*(q, \varepsilon) = q$ and $\delta^*(q, w \cdot a) = \delta(\delta^*(q, w), a)$.
\end{defn}

\begin{defn}
	Let $\mathcal{S} = (Q, \Sigma, \delta)$ be a deterministic transition structure. For a set $\Omega \subseteq Q^* \cup Q^\omega$, we say that $\mathcal{S}$ has acceptance condition $\Omega$.
	
	We say that a run $\rho$ of $\mathcal{A}$ on some $w \in \Sigma^*$ is \emph{accepting}, if $\rho \in \Omega$; otherwise, the run is \emph{rejecting}. In either case, we say that $\mathcal{A}$ accepts or rejects $w$. 
	
	The \emph{language} of $\mathcal{A}$ with $\Omega$ from $q \in Q$ is the set of all words and $\omega$-words that are accepted by $\mathcal{A}$ from $q$.
\end{defn}

\begin{defn}
	Let $\mathcal{S} = (Q, \Sigma, \delta)$ be a deterministic transition structure. A \emph{strongly connected component}, or SCC, is a set $S \subseteq Q$ such that for all $p, q \in S$, there is a $w \in \Sigma^*$ with $\delta^*(p, w) = q$.
	
	An SCC $S$ is \emph{trivial} if it contains only one state $q$ and $\delta(q, a) \neq q$ for all $a \in \Sigma$.
\end{defn}

\begin{defn}
	A \emph{deterministic finite automaton} (or DFA) is a tuple $\mathcal{A} = (Q, \Sigma, \delta, F)$, where $F \subseteq Q$, such that $(Q, \Sigma, \delta)$ is a deterministic transition structure and has acceptance condition $\Omega = \{ \rho \in Q^* \mid \rho(|\rho|+1) \in F \}$. For the language of $(Q, \Sigma, \delta)$ with $\Omega$ from $q$, we write $L(\mathcal{A}, q)$.
\end{defn}

\begin{defn}
	A \emph{deterministic parity automaton} (or DPA) is a tuple $\mathcal{A} = (Q, \Sigma, \delta, c)$, where $c : Q \rightarrow \mathbb{N}$, such that $(Q, \Sigma, \delta)$ is a deterministic transition structure and has acceptance condition $\Omega = \{ \rho \in Q^* \mid \min \text{Inf}(c(\rho)) \text{ is even} \}$. For the language of $(Q, \Sigma, \delta)$ with $\Omega$ from $q$, we write $L(\mathcal{A}, q)$.
	
	We call the DPA a \emph{B\"uchi automaton} (or DBA) if $c(Q) \subseteq \{0, 1\}$. In that case, we use $F$ instead of $c$.
\end{defn}

\begin{defn}
	Let $\mathcal{A} = (Q, \Sigma, \delta, c)$ be a DPA. We define $c^* : Q \times (\Sigma^* \cup \Sigma^\omega) \rightarrow (\mathbb{N}^* \cup \mathbb{N}^\omega)$ as $c^*(q, w) : 1+|w| \rightarrow \mathbb{N}, i \mapsto c(\delta^*(q, w[0, i]))$.
\end{defn}

\begin{defn}
	Let $\mathcal{S} = (Q, \Sigma, \delta)$ be a deterministic transition structure and $P \subseteq Q$. We define $\mathcal{S} \upharpoonright_P = (P, \Sigma, \delta')$ as a transition structure in which $\delta'$ might only be a partial function. We set $\delta'(p, a) = \delta(p, a)$ if $\delta(p, a) \in P$, or undefined otherwise.
\end{defn}

