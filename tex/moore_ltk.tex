\section{Threshold Moore}
\begin{defn}
	Let $\mathcal{A} = (Q, \Sigma, \delta, c)$ be a DPA. For a set $K \subseteq \mathbb{N}$, we define the \emph{priority threshold} of $\mathcal{A}$ as $\mathcal{T}(\mathcal{A}, K) := (Q, \Sigma, \delta, c')$ with $c'(q) = \begin{cases} c(q) & \text{if } c(q) \notin K \\ \max c(Q) + 1 & \text{else} \end{cases}$.
	
	For a relation $\sim$, we define $\mathcal{T}(\sim, K) :=\, \approx$ as $(\mathcal{A}, p) \approx (\mathcal{B}, q)$ iff $(\mathcal{T}(\mathcal{A}, K), p) \sim (\mathcal{T}(\mathcal{B}, K), q)$.
\end{defn}

\begin{lem}
	If $\sim$ is an equivalence relation, then so is $\mathcal{T}(\sim, K)$. If $\sim$ is a congruence relation, then so is $\mathcal{T}(\sim, K)$.
\end{lem}

\begin{proof}
	%TODO
\end{proof}

\begin{defn}
	We set $\equiv_M^{\leq k} \,:= \mathcal{T}(\equiv_M, \{ n \in \mathbb{N} \mid n > k \})$.
\end{defn}

\vspace{5pt}

\begin{defn}
	Let $\mathcal{A}$ and $\mathcal{B}$ be DPAs and let $\sim$ be an equivalence relation that implies language equivalence. We define a relation $\equiv_\text{TM}^\sim$ such that $(\mathcal{A}, p) \equiv^\sim_\text{TM} (\mathcal{B}, q)$ if and only if all of the following are satisfied:
	\begin{enumerate}
		\item $c_1(p) = c_2(q)$
		\item $(\mathcal{A}, p) \equiv_M^{\leq c(p)} (\mathcal{B}, q)$
		\item $(\mathcal{A}, p) \sim (\mathcal{B}, q)$
	\end{enumerate}
\end{defn}


\begin{defn}
	Let $\sim$ be an equivalence relation. We write $\sim^{\leq k} := \mathcal{T}(\sim, \{n \in \mathbb{N} \mid n > k\})$. We call $\sim$ \emph{TM-suitable} if it satisfies the following: 
	
	Let $\mathcal{A}$ be a DPA and let $\lambda \subseteq \kappa \in \mathfrak{C}(\sim)$ such that all states in $\lambda$ share the same priority $k$. For all representative merges $\mathcal{A}'$ of $\mathcal{A}$ w.r.t. $\lambda$, for all $q \in Q'$ with $c(q) \leq k$, $(\mathcal{A}, q) \sim^{\leq c(q)} (\mathcal{A}', q)$.
	
	Furthermore, for all representative merges $\mathcal{A}''$ of $\mathcal{T}(\mathcal{A}, \{n \in \mathbb{N} \mid n > k\})$ w.r.t. $\lambda$, for all words $\alpha \in \Sigma^\omega$ and all runs $\rho$ and $\rho''$ of $\mathcal{A}$ and $\mathcal{A}''$ on that word starting in the same state such that $\min \text{Inf}(c(\rho)) \leq k$, $\rho$ and $\rho''$ have the same acceptance. %TODO
\end{defn}





\vspace{20pt}

\begin{lem}
	$\equiv_M$ is TM-suitable.
\end{lem}

\begin{proof}
	Let $\mathcal{A}$ be a DPA and let $\lambda \subseteq \kappa \in \mathfrak{C}(\sim)$ such that all states in $\lambda$ share the same priority $k$.
	
	For the first part, let $\mathcal{A}'$ of $\mathcal{A}$ w.r.t. $\lambda$ and $q \in Q'$ with $c(q) \leq k$. Let $\rho$ and $\rho'$ be the runs of $\mathcal{A}$ and $\mathcal{A}'$ on $\alpha \in \Sigma^\omega$ starting from $q$. By Lemma \ref{lem:general:cong_stays_in_merge}, $(\mathcal{A}, \rho(i)) \equiv_M^{\leq k} (\mathcal{A}, \rho'(i))$ for all $i$ which especially means that for all $i$, $c(\rho(i)) =^{\leq k} c(\rho'(i))$. Since $c(\rho'(i)) = c'(\rho'(i))$, that also implies $c(\rho(i)) =^{\leq k} c'(\rho'(i))$ which means that $(\mathcal{A}, q) \equiv_M^{\leq k} (\mathcal{A}', q)$.
	
	For the second part, let $\mathcal{A}''$ be a representativ merge of $\mathcal{T}(\mathcal{A}, \{n \in \mathbb{N} \mid n > k\})$ w.r.t. $\lambda$. Let $\alpha \in \Sigma^\omega$ with runs $\rho$ and $\rho''$ of the two automata on $\alpha$ starting in some $q \in Q''$ such that $\min \text{Inf}(c(\rho)) \leq k$. Again by Lemma \ref{lem:general:cong_stays_in_merge}, $(\mathcal{A}, \rho(i)) \equiv_M^{\leq k} (\mathcal{A}, \rho'(i))$ for all $i$. In particular, whenever $\min \text{Inf}(c(\rho))$ or $\min \text{Inf}(c''(\rho''))$ is seen, the same is true for the other run. That is why $\min \text{Inf}(c(\rho)) = \min \text{Inf}(c''(\rho''))$ and the two runs have the same acceptance.
\end{proof}


\vspace{25pt}

\begin{defn}
	Let $\mathcal{A}$ and $\mathcal{B}$ be DPAs. Let $\sim$ be an equivalence relation that implies language equivalence and let $\approx$ be an equivalence relation that is TM suitable. We define a relation $\equiv_\text{TM}^{\sim, \approx}$ such that $(\mathcal{A}, p) \equiv^{\sim, \approx}_\text{TM} (\mathcal{B}, q)$ if and only if all of the following are satisfied:
	\begin{enumerate}
		\item $c_1(p) = c_2(q)$
		\item $(\mathcal{A}, p) \approx^{\leq c(p)} (\mathcal{B}, q)$
		\item $(\mathcal{A}, p) \sim (\mathcal{B}, q)$
	\end{enumerate}
\end{defn}


We can notice that merging classes of $\equiv_\text{TM}^{\sim, \approx}$ in a specific order is a valid language-preserving operation. This is then used to define a fitting merger function. For this next part, let $\mathcal{A}$ be a DPA and let $\kappa \in \mathfrak{C}(\equiv^{\sim, \approx}_\text{TM})$ be a set of states in $\mathcal{A}$. By definition of $\equiv^{\sim, \approx}_\text{TM}$, all states in this class have a unique priority $k$. Let $\mathcal{A}'$ be a representative merge of $\mathcal{A}$ w.r.t. $\kappa$.

\begin{lem}
	For all $q \in Q'$, $(\mathcal{A}, q) \sim (\mathcal{A}', q)$.
	\label{lem:tremoore:merge_keep_sim}
\end{lem}

\begin{proof} 
	Let $\rho$ and $\rho'$ be the runs of $\mathcal{A}$ and $\mathcal{A}'$ on $\alpha \in \Sigma^\omega$ starting from $q$. We claim that $\rho$ is accepting iff $\rho'$ is accepting.
	
	By Lemma \ref{lem:general:cong_stays_in_merge}, $(\mathcal{A}, \rho(i)) \equiv_L (\mathcal{A}, \rho'(i))$ and $(\mathcal{A}, \rho(i)) \approx^{\leq k} (\mathcal{A}, \rho'(i))$ for all $i$. If $c(\rho)$ sees infinitely many priorities of at most $k$, then %TODO
	
	 Now there are two cases: if $c(\rho)$ sees infinitely many priorities of at most $k$, then $c(\rho')$ sees the same priorities at the same positions and thus $\min \text{Inf}(c(\rho)) = \min \text{Inf}(c(\rho'))$. By definition of a representative merge, $c(\rho') = c'(\rho')$.
	
	 Otherwise there is a position $n$ from which $c(\rho)$ only is greater than $k$ and therefore the same is true for $c(\rho')$. That means reading $\alpha[n,\omega]$ from $\rho'(n)$ in either $\mathcal{A}$ or $\mathcal{A}'$ yields the same run which has the same acceptance as $\rho$.
\end{proof}

\begin{lem}
	For all $q \in Q'$ with $c(q) \leq k$, $(\mathcal{A}, q) \approx^{\leq k} (\mathcal{A}', q)$.
	\label{lem:tremoore:merge_keep_tmoore}
\end{lem}

\begin{proof} 
	Let $\rho$ and $\rho'$ be the runs of $\mathcal{A}$ and $\mathcal{A}'$ on $\alpha \in \Sigma^\omega$ starting from $q$. We claim that $\rho$ is accepting iff $\rho'$ is accepting.
	
	By Lemma \ref{lem:general:cong_stays_in_merge}, $(\mathcal{A}, \rho(i)) \equiv_M^{\leq k} (\mathcal{A}, \rho'(i))$ for all $i$ which especially means that for all $i$, $c(\rho(i)) =^{\leq k} c(\rho'(i))$. Since $c(\rho'(i)) = c'(\rho'(i))$, that also implies $c(\rho(i)) =^{\leq k} c'(\rho'(i))$ which means that $(\mathcal{A}, q) \equiv_M^{\leq k} (\mathcal{A}', q)$.
\end{proof}

\begin{lem}
	Let $L \subseteq K \subseteq \mathbb{N}$. Then $\mathcal{T}(\equiv_M, K) \subseteq \mathcal{T}(\equiv_M, L)$.
	\label{lem:tremoore:moore_less_thresh_is_subset}
\end{lem}

\begin{proof}
	%TODO
\end{proof}

\begin{lem}
	For all $q \in Q'$ with $c(q) \leq k$, $(\mathcal{A}, q) \equiv^\sim_\text{TM} (\mathcal{A}', q)$.
	\label{lem:tremoore:merge_keep_tm}
\end{lem}

\begin{proof}
	Representative merges never change priorities assigned to states. Together with Lemma \ref{lem:tremoore:merge_keep_sim}, Lemma \ref{lem:tremoore:merge_keep_tmoore}, and Lemma \ref{lem:tremoore:moore_less_thresh_is_subset} this already finishes the proof.
\end{proof}

\begin{lem}
	For all $p, q \in Q'$ with $\min \{c(p), c(q)\} \leq k$, $(\mathcal{A}, p) \equiv^\sim_\text{TM} (\mathcal{A}, q)$ iff \linebreak $(\mathcal{A}', p) \equiv^\sim_\text{TM} (\mathcal{A}', q)$.
	\label{lem:tremoore:merge_changes_only_higher}
\end{lem}

\begin{proof}
	If $c(p) = c(q) \leq k$: By Lemma \ref{lem:tremoore:merge_keep_tm}, we know that $(\mathcal{A}, p) \equiv^\sim_\text{TM} (\mathcal{A}', p)$ and $(\mathcal{A}, q) \equiv^\sim_\text{TM} (\mathcal{A}', q)$. If now $(\mathcal{A}, p) \equiv^\sim_\text{TM} (\mathcal{A}, q)$ holds, then also $(\mathcal{A}', p) \equiv^\sim_\text{TM} (\mathcal{A}', q)$ and vice versa.
	
	If $c(p) \neq c(q)$, then also $c'(p) \neq c'(q)$ and we have both $(\mathcal{A}, p) \not\equiv^\sim_\text{TM} (\mathcal{A}, q)$ and $(\mathcal{A}', p) \not\equiv^\sim_\text{TM} (\mathcal{A}', q)$.
\end{proof}

\newpage



\vspace{25pt}

We can notice that merging classes of $\equiv_\text{TM}^\sim$ in a specific order is a valid language-preserving operation. This is then used to define a fitting merger function. For this next part, let $\mathcal{A}$ be a DPA and let $\kappa \in \mathfrak{C}(\equiv^\sim_\text{TM})$ be a set of states in $\mathcal{A}$. By definition of $\equiv^\sim_\text{TM}$, all states in this class have a unique priority $k$. Let $\mathcal{A}'$ be a representative merge of $\mathcal{A}$ w.r.t. $\kappa$.

\begin{lem}
	For all $q \in Q'$, $(\mathcal{A}, q) \sim (\mathcal{A}', q)$.
	\label{lem:tremoore:merge_keep_sim}
\end{lem}

\begin{proof} 
	Let $\rho$ and $\rho'$ be the runs of $\mathcal{A}$ and $\mathcal{A}'$ on $\alpha \in \Sigma^\omega$ starting from $q$. We claim that $\rho$ is accepting iff $\rho'$ is accepting.
	
	By Lemma \ref{lem:general:cong_stays_in_merge}, $(\mathcal{A}, \rho(i)) \equiv_L (\mathcal{A}, \rho'(i))$ and $(\mathcal{A}, \rho(i)) \equiv_M^{\leq k} (\mathcal{A}, \rho'(i))$ for all $i$. Now there are two cases: if $c(\rho)$ sees infinitely many priorities of at most $k$, then $c(\rho')$ sees the same priorities at the same positions and thus $\min \text{Inf}(c(\rho)) = \min \text{Inf}(c(\rho'))$. By definition of a representative merge, $c(\rho') = c'(\rho')$.
	
	 Otherwise there is a position $n$ from which $c(\rho)$ only is greater than $k$ and therefore the same is true for $c(\rho')$. That means reading $\alpha[n,\omega]$ from $\rho'(n)$ in either $\mathcal{A}$ or $\mathcal{A}'$ yields the same run which has the same acceptance as $\rho$.
\end{proof}

\begin{lem}
	For all $q \in Q'$ with $c(q) \leq k$, $(\mathcal{A}, q) \equiv_M^{\leq k} (\mathcal{A}', q)$.
	\label{lem:tremoore:merge_keep_tmoore}
\end{lem}

\begin{proof} 
	Let $\rho$ and $\rho'$ be the runs of $\mathcal{A}$ and $\mathcal{A}'$ on $\alpha \in \Sigma^\omega$ starting from $q$. We claim that $\rho$ is accepting iff $\rho'$ is accepting.
	
	By Lemma \ref{lem:general:cong_stays_in_merge}, $(\mathcal{A}, \rho(i)) \equiv_M^{\leq k} (\mathcal{A}, \rho'(i))$ for all $i$ which especially means that for all $i$, $c(\rho(i)) =^{\leq k} c(\rho'(i))$. Since $c(\rho'(i)) = c'(\rho'(i))$, that also implies $c(\rho(i)) =^{\leq k} c'(\rho'(i))$ which means that $(\mathcal{A}, q) \equiv_M^{\leq k} (\mathcal{A}', q)$.
\end{proof}

\begin{lem}
	Let $L \subseteq K \subseteq \mathbb{N}$. Then $\mathcal{T}(\equiv_M, K) \subseteq \mathcal{T}(\equiv_M, L)$.
	\label{lem:tremoore:moore_less_thresh_is_subset}
\end{lem}

\begin{proof}
	%TODO
\end{proof}

\begin{lem}
	For all $q \in Q'$ with $c(q) \leq k$, $(\mathcal{A}, q) \equiv^\sim_\text{TM} (\mathcal{A}', q)$.
	\label{lem:tremoore:merge_keep_tm}
\end{lem}

\begin{proof}
	Representative merges never change priorities assigned to states. Together with Lemma \ref{lem:tremoore:merge_keep_sim}, Lemma \ref{lem:tremoore:merge_keep_tmoore}, and Lemma \ref{lem:tremoore:moore_less_thresh_is_subset} this already finishes the proof.
\end{proof}

\begin{lem}
	For all $p, q \in Q'$ with $\min \{c(p), c(q)\} \leq k$, $(\mathcal{A}, p) \equiv^\sim_\text{TM} (\mathcal{A}, q)$ iff \linebreak $(\mathcal{A}', p) \equiv^\sim_\text{TM} (\mathcal{A}', q)$.
	\label{lem:tremoore:merge_changes_only_higher}
\end{lem}

\begin{proof}
	If $c(p) = c(q) \leq k$: By Lemma \ref{lem:tremoore:merge_keep_tm}, we know that $(\mathcal{A}, p) \equiv^\sim_\text{TM} (\mathcal{A}', p)$ and $(\mathcal{A}, q) \equiv^\sim_\text{TM} (\mathcal{A}', q)$. If now $(\mathcal{A}, p) \equiv^\sim_\text{TM} (\mathcal{A}, q)$ holds, then also $(\mathcal{A}', p) \equiv^\sim_\text{TM} (\mathcal{A}', q)$ and vice versa.
	
	If $c(p) \neq c(q)$, then also $c'(p) \neq c'(q)$ and we have both $(\mathcal{A}, p) \not\equiv^\sim_\text{TM} (\mathcal{A}, q)$ and $(\mathcal{A}', p) \not\equiv^\sim_\text{TM} (\mathcal{A}', q)$.
\end{proof}

\vspace{10pt}

\begin{defn}
	Let $\mathcal{A}$ be a DPA and let $\sim$ be an equivalence relation that implies language equivalence. We define the \emph{Threshold Moore merger function} $\mu_\text{TM}^\sim := \mu_\div^{\equiv^\sim_\text{TM}}$.
\end{defn}

\begin{theorem}
	Let $\mathcal{A}$ be a DPA and let $\sim$ be an equivalence relation that implies language equivalence. Every representative merge of $\mathcal{A}$ w.r.t. $\mu_\text{TM}^\sim$ is language equivalent to $\mathcal{A}$.
\end{theorem}

\begin{proof}
	Let $\kappa_1, \dots, \kappa_m$ be an enumeration of $\mathfrak{C}(\equiv^\sim_\text{TM})$ such that the (unique) priority of states in $\kappa_i$ is $k_i$ and $k_1, \dots, k_m$ is a descending series.
	
	If we build a representative merge of $\mathcal{A}$ w.r.t. $\kappa_1$ by candidates $_1$, we obtain some DPA $\mathcal{A}_1$. By Lemma \ref{lem:tremoore:merge_changes_only_higher}, for all $i > 1$, the states in $\kappa_i$ are still pairwise $\equiv^\sim_\text{TM}$ equivalent. Moreover, $\kappa_i$ is an $\equiv^\sim_\text{TM}$ equivalence class in $\mathcal{A}_1$.
	
	That means we can continue building representative merges in order of the enumeration and our previous results apply. In the end, we obtain a DPA $\mathcal{A}'$ that is language equivalent to $\mathcal{A}$ by Lemma \ref{lem:tremoore:merge_keep_sim}.
\end{proof}

\newpage






















