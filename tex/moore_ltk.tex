\section{Threshold Moore}
\begin{defn}
	Let $x, y, n \in \mathbb{N}$. We write $x =^{\leq n} y$ if $x = y$ or $x, y > n$.

	Let $\mathcal{A} = (Q, \Sigma, q_0, \delta, c)$ be a DPA. For $k \in c(Q)$, we define $\equiv_M^{\leq k} \subseteq Q \times Q$ as a relation, such that $p \equiv_M^{\leq k} q$ if and only if for all $w \in \Sigma^*$, $c(\delta^*(p, w)) =^{\leq k} c(\delta^*(q, w))$. We call $\equiv_M^{\leq k}$ the \emph{$k$-threshold Moore equivalence}.
\end{defn}

\begin{lem}
	Let $\mathcal{A} = (Q, \Sigma, q_0, \delta, c)$ be a DPA and let $\mathcal{A}' = (Q, \Sigma, q_0, \delta, c')$ with $c'(q) = \min \{k+1, c(q)\}$. Then $\equiv_M^{\leq k}$ of $\mathcal{A}$ is equal to $\equiv_M$ of $\mathcal{A}'$.
	\label{lem:tremoore:mleq_is_actually_just_moore}
\end{lem}

\begin{proof}
	Follows directly from the definition of $=^{\leq k}$.
\end{proof}

\begin{cor}
	$\equiv_M^{\leq k}$ is a congruence relation.
\end{cor}

\begin{defn}
	Let $\mathcal{A}$ be a DPA and let $R$ be an equivalence relation on the state space that implies language equivalence. We define a relation $\equiv_\text{TM}^R$ such that $p \equiv_\text{TM} q$ if and only if all of the following are satisfied:
	\begin{enumerate}
		\item $c(p) = c(q)$
		\item $p \equiv_M^{\leq c(p)} q$
		\item $(p, q) \in R$
	\end{enumerate}
\end{defn}


\begin{theorem}
	Let $\mathcal{A}$ and $R$ as before and let $\mathcal{A}'$ be a representative merge of $\mathcal{A}$ w.r.t. an equivalence class $\lambda$ of $\equiv_\text{TM}^R$. Then $L(\mathcal{A}) = L(\mathcal{A}')$.
\end{theorem}

\begin{proof}
	Let $q \in Q'$ be a state in the representative merge and let $\alpha \in \Sigma^\omega$. Let $\rho$ and $\rho'$ be the runs of $\mathcal{A}$ and $\mathcal{A}'$ on $\alpha$ starting from $q$. We claim that $\rho$ is accepting iff $\rho'$ is accepting.
	
	By Lemma \ref{lem:tremoore:cong_stays_in_merge}, $\rho(i) \equiv_L \rho'(i)$ and $\rho(i) \equiv_M^{\leq k} \rho'(i)$ for all $i$. Now there are two cases: if $c(\rho)$ sees infinitely many priorities of at most $k$, then $c(\rho')$ sees the same priorities at the same positions and thus $\min \text{Inf}(c(\rho)) = \min \text{Inf}(c(\rho'))$. Otherwise there is a position $n$ from which $c(\rho)$ only is greater than $k$ and therefore the same is true for $c(\rho')$. That means reading $\alpha[n,\omega]$ from $\rho'(n)$ in either $\mathcal{A}$ or $\mathcal{A}'$ yields the same run which has the same acceptance as $\rho$.
\end{proof}

\begin{lem}
	Let $\mathcal{A}$ be a DPA and let $p$ and $q$ be two states with $p \equiv_M q$. We construct $\mathcal{A}'$ from $\mathcal{A}$ by redirecting all transitions to $p$ to $q$ instead. Then for all states $r \neq p$ and all words $w$, $c(\delta^*(r, w)) = c'(\delta^{\prime *}(r, w))$.
	\label{lem:tremoore:moore_redirect_is_ok}
\end{lem}

\begin{proof}
	Let $\rho$ and $\rho'$ be the runs of $\mathcal{A}$ and $\mathcal{A}'$ on $w$ starting in $r$. If $\rho$ never visits $p$, then $\rho = \rho'$ and the proof is done. Otherwise, let $n$ be the last position at which $\rho(n) = p$. Then $\rho'(n) = q$. Since $p \equiv_M q$, $c(\delta^*(p, u)) = c(\delta^*(q, u))$ for all $u \in \Sigma^*$ and especially for $u = w[n, |w|]$. Since $n$ was chosen as the last position where $p$ is visited, $\delta^*(q, u) = \delta^{\prime *}(q, u)$ and therefore $c(\delta^*(p, u)) = c'(\delta^{\prime *}(q, u))$ which finishes the proof.
\end{proof}

\begin{lem}
	Let $\mathcal{A}$ and $R$ as before and let $\mathcal{A}'$ be a representative merge of $\mathcal{A}$ w.r.t. an equivalence class $\lambda$ of $\equiv_\text{TM}^R$. Let $k$ be the priority of the states in $\lambda$ and let $\equiv_M^{\leq l}$ and $\equiv_M^{\unlhd l}$ be the $l$-threshold Moore equivalences of $\mathcal{A}$ and $\mathcal{A}'$. If $l \leq k$, then $\equiv_M^{\leq l}$ and $\equiv_M^{\unlhd l}$ are the same.
\end{lem}

\begin{proof}
	A representative merge w.r.t. $\lambda$ can be seen as a repeated redirection of transitions, meaning that Lemma \ref{lem:tremoore:moore_redirect_is_ok} applies. Together with Lemma \ref{lem:tremoore:mleq_is_actually_just_moore}, that already finishes our proof.
\end{proof}

On the other hand, figures \ref{} show that if $l > k$, the $l$-threshold Moore equivalence can both grow or shrink during the merge step. %TODO



























