
\section{Automata}
In this work, we focus ourselves on the model of parity automata (abbreviated PA) (also known as Rabin chain automata). We now give three different but similar definitions of this structure, the classical model and two variations better suited for the upcoming constructions. 


\begin{defn}
	A \emph{standard parity automaton} is a tuple $\mathcal{A} = (Q, \Sigma, q_0, \delta, c)$, where
	\begin{itemize}
		\item $Q$ is the finite, nonempty set of states.
		\item $\Sigma$ is the input alphabet.
		\item $q_0 \in Q$ is the initial state.
		\item $\delta : Q \times \Sigma \rightarrow \mathcal{P}(Q)$ is the transition function.
		\item $c : Q \rightarrow \mathbb{N}$ is the priority function.
	\end{itemize}
	$\mathcal{A}$ is said to be \emph{deterministic} if for all $q \in Q, a \in \Sigma$, the transition is uniquely defined, i.e. $|\delta(q, a)| \leq 1$. In that case, we write $\delta(q, a) = \emptyset$ or $\delta(q, a) = p \in Q$.
	
	A \emph{run} of $\mathcal{A}$ on an $\omega$-word $\alpha \in \Sigma^\omega$ is a word $\rho \in Q^\omega$, such that $\rho(0) = q_0$ and for all $i > 0: \rho(i) \in \delta(\rho(i-1), \alpha(i))$. \\
	The priorities of $\rho$ are $c(\rho) = \gamma \in \mathbb{N}^\omega$ such that $\gamma(i) = c(\rho(i))$. $\rho$ is \emph{accepting} if $\max \text{Inf}(c(\rho))$ is an even number.
	
	The language of $\mathcal{A}$, written $L(\mathcal{A}) \subseteq \Sigma^\omega$ is defined as the set of all $\omega$-words over $\Sigma$ which have an accepting run on $\mathcal{A}$. \\
	$\mathcal{A}$ \emph{recognizes} or \emph{accepts} a language $U \subseteq \Sigma^\omega$ if $L(\mathcal{A}) = U$.
\end{defn}


\begin{defn}
	A \emph{transition parity automaton} is a tuple $\mathcal{A} = (Q, \Sigma, q_0, \delta, c, F)$, where
	\begin{itemize}
		\item $Q$ is the finite, nonempty set of states.
		\item $\Sigma$ is the input alphabet.
		\item $q_0 \in Q$ is the initial state.
		\item $\delta : Q \times \Sigma \rightarrow \mathcal{P}(Q)$ is the transition function.
		\item $c : Q \times Q \rightarrow \mathbb{Z}$ is the priority function.
		\item $F \subseteq Q$ is the set of accepting states.
	\end{itemize}
	Determinism of $\mathcal{A}$ is defined as for standard parity automata.
	
	A \emph{run} of $\mathcal{A}$ on an $\infty$-word $w \in \Sigma^\infty$ is a word $\rho \in Q^{1 + |w|}$, such that $\rho(0) = q_0$ and for all $0 < i \leq |w|: \rho(i) \in \delta(\rho(i-1), \alpha(i))$. \\
	The priorities of $\rho$ are $c(\rho) = \gamma \in \mathbb{N}^{|w|}$ such that $\gamma(i) = c(\rho(i), \rho(i+1))$. For an $\omega$-word, $\rho$ is \emph{accepting} if $\max \text{Inf}(c(\rho))$ is an even number. For a finite word, $\rho$ is \emph{accepting} if $\rho(|w|) \in F$.
	
	The language of $\mathcal{A}$, written $L(\mathcal{A}) \subseteq \Sigma^\infty$ is defined as the set of all $\infty$-words over $\Sigma$ which have an accepting run on $\mathcal{A}$. We define $L_*(\mathcal{A}) = L(\mathcal{A}) \cap \Sigma^*$ and $L_\omega(\mathcal{A}) = L(\mathcal{A}) \cap \Sigma^\omega$. \\
	$\mathcal{A}$ \emph{recognizes} or \emph{accepts} a language $L \subseteq \Sigma^\infty$ if $L(\mathcal{A}) = L$.
	
	Regarding the transition function $\delta$, we also write $\delta(q, A) = \bigcup\limits_{a \in A} \delta(q, a)$ for some set of symbols $A \subseteq \Sigma$.
\end{defn}


\begin{defn}
	An \emph{expression parity automaton} is a tuple $\mathcal{A} = (Q, \Sigma, q_0, \delta, c)$, where
	\begin{itemize}
		\item $Q$ is the finite, nonempty set of states.
		\item $\Sigma$ is the input alphabet.
		\item $q_0 \in Q$ is the initial state.
		\item $\delta : Q \times Q \rightarrow \mathcal{R}eg_\infty[\Sigma]$ is the transition function, with $\infty$-regular expressions as labels.
		\item $c : Q \rightarrow \mathbb{N}$ is the priority function.
	\end{itemize}
	
	A \emph{run} of $\mathcal{A}$ on an $\infty$-word $w \in \Sigma^\infty$ is a word $\rho \in Q^{1 + |w|}$, such that $\rho(0) = q_0$ and there is a decomposition $w = \underset{i < |\rho|}{\circ} v_i$, with $v_i \in \begin{cases}\Sigma^* & \text{if } i+1 < |\rho| \\ \Sigma^\infty & \text{else} \end{cases}$, such that $v_i \in L(\delta(\rho(i), \rho(i+1)))$ for all  $i$. \\
	The priorities of $\rho$ are $c(\rho) = \gamma \in \mathbb{N}^{|\rho|}$ such that $\gamma(i) = c(\rho(i))$. $\rho$ is accepting if $C = \emptyset$ or $\max C$ is even, where $C = \text{Inf}(c(\rho))$.
	
	The language of $\mathcal{A}$, written $L(\mathcal{A}) \subseteq \Sigma^\infty$ is defined as the set of all $\infty$-words over $\Sigma$ which have an accepting run on $\mathcal{A}$. We define $L_*(\mathcal{A}) = L(\mathcal{A}) \cap \Sigma^*$ and $L_\omega(\mathcal{A}) = L(\mathcal{A}) \cap \Sigma^\omega$. \\
	$\mathcal{A}$ \emph{recognizes} or \emph{accepts} a language $L \subseteq \Sigma^\infty$ if $L(\mathcal{A}) = L$.
\end{defn}


Every $\omega$-regular language can be recognized by a standard parity automaton, even a deterministic one. (see \cite{Thomas96}). The other two models are exactly as powerful.

\begin{prop}
	Let $U \subseteq \Sigma^\omega$ be an $\omega$-language. The following statements are equivalent:
	\begin{itemize}
		\item $U$ is regular.
		\item There is a standard PA $\mathcal{A}$ with $L(\mathcal{A}) = U$.
		\item There is a transition PA $\mathcal{A}$ with $L_\omega(\mathcal{A}) = U$.
		\item There is an expression PA $\mathcal{A}$ with $L_\omega(\mathcal{A}) = U$.
	\end{itemize}
\end{prop}

\begin{prop}
	Let $L \subseteq \Sigma^\infty$ be an $\infty$-language. The following statements are equivalent:
	\begin{itemize}
		\item $L$ is regular.
		\item There is a transition PA $\mathcal{A}$ with $L(\mathcal{A}) = L$.
		\item There is an expression PA $\mathcal{A}$ with $L(\mathcal{A}) = L$.
	\end{itemize}
\end{prop}

Most of the constructions are trivial and require at most $|Q|$ many additional states. The only step not as clear is the question on how to transform an expression PA to one of the other models. That will be part of a later chapter, when we discuss the conversion from automata to regular expressions. 



\begin{defn}
	Let $\mathcal{A} = (Q, \Sigma, q_0, \delta, c, F)$ and $\mathcal{A}' = (Q', \Sigma, q'_0, \delta', c', F')$ be transition parity automata. A function $f : Q \rightarrow Q'$ is an \emph{isomorphism} from $\mathcal{A}$ to $\mathcal{A}'$, if
	\begin{itemize}
		\item $f$ is bijective.
		\item $f(q_0) = q'_0$.
		\item For all $q \in Q$, for all $a \in \Sigma$: $\delta'(f(q), a) = \{f(p) \mid p \in \delta(q, a))\}$.
		\item For all $q \in Q$: $q \in F$ if and only if $f(q) \in F'$.
		\item There is a bijective, monotone function $n : \mathbb{Z} \rightarrow \mathbb{Z}$ (i.e. $x < y$ implies $n(x) < n(y)$) for which $n(x)$ is even if and only if $x$ is even, such that for all $q \in Q$ and $p \in \text{succ}(q)$: $c(p, q) = n(c(f(p), f(q)))$. (In particular, this can be satisfied by the identity function.)
	\end{itemize}
	
	If an isomorphism from $\mathcal{A}$ to $\mathcal{A}'$ exists, $\mathcal{A}$ and $\mathcal{A}'$ are \emph{isomorphic} ($\mathcal{A} \cong \mathcal{A}'$).
\end{defn}


\begin{prop}
	Let $\mathcal{A}$ and $\mathcal{A}'$ be transition parity automata. If $\mathcal{A} \cong \mathcal{A}'$, then $L(\mathcal{A}) = L(\mathcal{A}')$.
\end{prop}

\begin{proof}
	$\mathcal{A} \cong \mathcal{A}'$, so let $f : Q \rightarrow Q'$ be an isomorphism from $\mathcal{A}$ to $\mathcal{A}'$. It is clear that $f^{-1}$ is also an isomorphism from $\mathcal{A}'$ to $\mathcal{A}$ and therefore, it suffices to show that $L(\mathcal{A}) \subseteq L(\mathcal{A}')$. Let $w \in L(\mathcal{A})$ and let $\rho$ be the run of $\mathcal{A}$ on $w$.
	
	We first show that $\rho' : |\rho| \rightarrow Q', i \mapsto f(\rho(i))$ is a valid run of $\mathcal{A}'$ on $w$. We do so using induction over $|w|$. For $|w| = 0$, this is clear because $\rho'(0) = q'_0 = f(q_0) = f(\rho(0))$. Otherwise, let $0 < |w|$. By the induction hypothesis, $\rho'[0, |w|-1]$ is a valid run of $\mathcal{A}'$ on $w[0, |w|-2]$. $\rho$ is a valid run of $\mathcal{A}$ on $w$, so $\rho(|w|) \in \delta(\rho(|w|-1), w(|w|-1))$. The definition of an isomorphism then implies that $\delta'(f(\rho(|w|-1)), w(|w|-1)) = \delta'(\rho'(|w|-1), w(|w|-1)) = \{f(p) \mid p \in \delta(\rho(|w|-1), w(|w|-1))\}$. In particular, $f(\rho(|w|)) = \rho'(|w|) \in \delta'(\rho'(|w|-1), w(|w|-1))$, which is what had to be shown.
	
	If $w \in \Sigma^*$, $\rho(|w|) \in F$. By definition, $\rho'(|w|) \in F'$, which means that $w \in L(\mathcal{A}')$.
	
	If $w \in \Sigma^\omega$, let $P$ be the priority set of $\rho$. $\max P$ must be even. By definition of the isomorphism, the priority set of $\rho'$ is $P' = \{n(x) \mid x \in P\}$ for a fitting function $n$. $n$ is monotone, so $(\max P) = \max P'$. Also, $\max P$ is even which means that $n(\max P) = \max P'$ must be even. Therefore, $w \in L(\mathcal{A}')$.
\end{proof}
