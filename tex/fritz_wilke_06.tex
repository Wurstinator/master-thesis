
\section{Fritz \& Wilke}
\label{sect:fritzwilke}

\subsection{Delayed Simulation Game}
In this section we consider delayed simulation games and variants thereof on DPAs. This approach is based on the paper \cite{FritzWilke06}, which considered the games for alternating parity automata. The DPAs we use are a special case of these APAs and therefore worth examining.

\begin{defn}
	For convenience, we define two orders for this chapter. First, we introduce $\checkmark$ as an \enquote{infinity} to the natural numbers and define the \textbf{obligation order} $\leq_\checkmark \subseteq (\mathbb{N} \cup \{\checkmark\}) \times (\mathbb{N} \cup \{\checkmark\})$ as $0 \leq_\checkmark 1 \leq_\checkmark 2 \leq_\checkmark \dots \leq_\checkmark \checkmark$.
	
	Second, we define an order of \enquote{goodness} on parity priorities $\preceq_\text{p} \subseteq \mathbb{N} \times \mathbb{N}$ as $0 \preceq_\text{p} 2 \preceq_\text{p} 4 \preceq_\text{p} \dots \preceq_\text{p} 5 \preceq_\text{p} 3 \preceq_\text{p} 1$.
\end{defn}

\begin{defn}
	Let $\mathcal{A} = (Q, \Sigma, \delta, c)$ be a DPA. We define the \emph{delayed simulation automaton} $\mathcal{A}_\text{de}(p, q) = (Q_\text{de}, \Sigma, \delta_\text{de}, F_\text{de})$ with starting state $q_0^\text{de}(p, q) = (p, q, \gamma(c(p), c(q), \checkmark))$, which is a deterministic Büchi automaton, as follows.
	
	\begin{itemize}
		\item $Q_\text{de} = Q \times Q \times (\text{img}(c) \cup \{ \checkmark \})$, i.e. the states are given as triples in which the first two components are states from $\mathcal{A}$ and the third component is either a priority from $\mathcal{A}$ or $\checkmark$.
		\item The alphabet remains $\Sigma$.
		\item $\delta_\text{de}( (p, q, k), a ) = ( p', q', \gamma(c(p'), c(q'), k)$, where $p' = \delta(p, a)$, $q' = \delta(q, a)$, and $\gamma$ is the same function as used in the initial state. The first two components behave like a regular product automaton.
		\item $F_\text{de} = Q \times Q \times \{ \checkmark \}$.
		\item The starting state is a triple $(p, q, \gamma(c(p), c(q), \checkmark))$, where $p, q \in Q$ are parameters given to the automaton, and $\gamma$ is defined below.
	\end{itemize}
	
	$\gamma : \mathbb{N} \times \mathbb{N} \times (\mathbb{N} \cup \{\checkmark\}) \rightarrow \mathbb{N} \cup \{\checkmark\}$ is the update function of the third component and defines the \enquote{obligations} as they are called in \cite{}. It is defined as 
	$$ \gamma(i, j, k) = \begin{cases}
		\checkmark & \text{if } i \text{ is odd and } i \leq_\checkmark k \text{ and } j \preceq_\text{p} i \\
		\checkmark & \text{if } j \text{ is even and } j \leq_\checkmark k \text{ and } j \preceq_\text{p} i \\
		\min_{\leq_\checkmark} \{ i,j,k \} & \text{else}
	\end{cases} $$
\end{defn}


\begin{defn}
	Let $\mathcal{A}$ be a DPA and let $\mathcal{A}_\text{de}$ be the delayed simulation automaton of $\mathcal{A}$. We say that a state $p$ $de$-simulates a state $q$ if $L(\mathcal{A}_\text{de}(p, q), q_0^\text{de}(p, q)) = \Sigma^\omega$. In that case we write $p \leq_\text{de} q$. If also $q \leq_\text{de} p$ holds, we write $p \equiv_\text{de} q$.
\end{defn}



\vspace{1cm}
\subsubsection*{$\boldsymbol{\equiv_\text{de}}$ is a congruence relation.}
Our overall goal is to use $\equiv_\text{de}$ to build a quotient automaton of our original DPA. The first step towards this goal is to show that the result is actually a well-defined DPA, by proving that the relation is a congruence.

\begin{lem}
\label{lem:fritzwilke:gamma_mono}
	$\gamma$ is monotonous in the third component, i.e. if $k \leq_\checkmark k'$, then $\gamma(i, j, k) \leq_\checkmark \gamma(i, j, k')$ for all $i, j \in \mathbb{N}$.
\end{lem}

\begin{proof}
	We consider each case in the definition of $\gamma$. If $i$ is odd, $i \leq_\checkmark k$ and $j \preceq_\text{p} i$, then also $i \leq_\checkmark k'$ and $\gamma(i, j, k) = \gamma(i, j, k') = \checkmark$.
	
	If $j$ is even, $j \leq_\checkmark k$ and $j \preceq_\text{p} i$, then also $j \leq_\checkmark k'$ and $\gamma(i, j, k) = \gamma(i, j, k') = \checkmark$.
	
	Otherwise, $\gamma(i, j, k) = \min \{i, j, k\}$ and $\gamma(i, j, k') = \min \{i, j, k'\}$. Since $k \leq_\checkmark k'$, $\gamma(i, j, k) \leq_\checkmark \gamma(i, j, k')$.
\end{proof}

\begin{lem}
\label{lem:fritzwilke:gamma_mono_ext}
	Let $\mathcal{A}$ be a DPA and let $p, q \in Q$, $k \in \mathbb{N} \cup \{\checkmark\}$. If the run of $\mathcal{A}_\text{de}$ starting at $(p, q, k)$ on some $\alpha \in \Sigma^\omega$ is accepting, then for all $k \leq_\checkmark k'$ also the run of $\mathcal{A}_\text{de}$ starting at $(p, q, k')$ on $\alpha$ is accepting.
\end{lem}

\begin{proof}
	Let $\rho$ be the run starting at $(p, q, k)$ and let $\rho'$ be the run starting at $(p, q, k')$. Further, let $p_i$, $q_i$, $k_i$, and $k'_i$ be the components of the states of those runs in the $i$-th step. Via induction we show that $k_i \leq_\checkmark k'_i$ for all $i$. Since $k_i$ is $\checkmark$ infinitely often, the same must be true for $k'_i$ and $\rho'$ is accepting.
	
	For $i = 0$, we have $k_0 = k \leq_\checkmark k' = k'_0$. Otherwise, we have $k_{i+1} = \gamma(c(p_{i+1}), c(q_{i+1}), k_i)$ and $k'_{i+1}$ analogously. The rest follows from Lemma \ref{lem:fritzwilke:gamma_mono}.
\end{proof}

\begin{lem}
\label{lem:fritzwilke:k_shrink}
	Let $\mathcal{A}$ be a DPA and $\rho \in Q_\text{de}^\omega$ be a run of $\mathcal{A}_\text{de}$ on some word. Let $k \in (\mathbb{N} \cup \{\checkmark\})^\omega$ be the third component during $\rho$. For all $i$, $k(i+1) \leq_\checkmark k(i)$ or $k(i+1) = \checkmark$.
\end{lem}

\begin{proof}
	Follows directly from the definition of $\gamma$.
\end{proof}

\begin{lem}
\label{lem:fritzwilke:n0_exists}
	Let $\mathcal{A}$ be a DPA with states $p, q \in Q$. For a word $\alpha \in \Sigma^\omega$, let $\rho : i \mapsto (p_i, q_i, k_i)$ be the run of $\mathcal{A}_\text{de}(p, q)$ on $\alpha$. If $\rho$ is not accepting, there is a position $n$ such that
	\begin{itemize}
		\item $\text{Occ}(\{p_i \mid i \geq n\}) = \text{Inf}(\{p_i \mid i \in \mathbb{N}\})$,
		\item $\text{Occ}(\{q_i \mid i \geq n\}) = \text{Inf}(\{q_i \mid i \in \mathbb{N}\})$,
		\item For all $i \geq j \geq n$, $k_i = k_j$ and $k_i \neq \checkmark$.
	\end{itemize}
	
	In other words, from $n$ on, $p$ and $q$ only see states that are seen infinitely often, and the obligations of $\rho$ do not change anymore.
\end{lem}

\begin{proof}
	The first two requirements are clear. Since states not in $\text{Inf}(\{p_i \mid i \in \mathbb{N}\})$ only occur finitely often, there must be positions $n_p$ and $n_q$ from which on they do not occur anymore at all.
	
	For the third requirement, we know that from some point on, the obligations $k$ never become $\checkmark$ anymore, as the run would be accepting otherwise. By Lemma \ref{lem:fritzwilke:k_shrink}, $k$ can only become lower from there on. As $\leq_\checkmark$ is a well-ordering, a minimum must be reached at some point $n_k$.
	
	The position $n_0 = \max \{n_p, n_q, n_k\}$ satisfies the statement.
\end{proof}

\begin{lem}
\label{lem:fritzwilke:run_goodness_implies_de}
	Let $\mathcal{A}$ be a DPA with two states $p, q \in Q$. Let $\alpha \in \Sigma^\omega$ be an $\omega$-word and let $\rho_p$ and $\rho_q$ be the respective runs of $\mathcal{A}$ on $\alpha$ starting in $p$ and $q$. If $\min \text{Inf}(c(\rho_q)) \preceq_p \min \text{Inf}(c(\rho_p))$, then $\alpha \in L(\mathcal{A}_\text{de}(p, q))$.
\end{lem}

\begin{proof}
	We write $l_q = \min \text{Inf}(c(\rho_q))$ and $l_p = \min \text{Inf}(c(\rho_p))$. Assume that the Lemma is false, so $l_q \preceq_p l_p$ but $\alpha \notin L(\mathcal{A}_\text{de}(p, q))$. Let $k_{13} \in (\mathbb{N} \cup \{\checkmark\})^\omega$ be the third component of the run of $\mathcal{A}_\text{de}(p, q)$ on $\alpha$. Let $n_0$ be a position as described in Lemma \ref{lem:fritzwilke:n0_exists} (for $\rho$). 
	
	\paragraph{Case 1: $l_q$ is even and $l_q \leq l_p$} We know $k_{13}(n_0) = l_q$, as that is the smaller value. Let $m > n_0$ be a position with $c(\rho_q(m)) = l_q$. Then $c(\rho_q(m))$ is even and $c(\rho_q(m)) = l_q \leq l_p \leq c(\rho_p(m))$, so $c(\rho_q(m)) \preceq_p c(\rho_p(m))$. Also we have $c(\rho_q(m)) \leq k_{13}(m-1) = l_q$ and therefore $k_{13}(m) = \checkmark$, which contradicts the choice of $n_0$.
	
	\paragraph{Case 2: $l_p$ is odd and $l_q \geq l_p$} We know $k_{13}(n_0) = l_p$. Let $m > n_0$ be a position with $c(\rho_p(m)) = l_p$. Then $c(\rho_p(m))$ is odd and $c(\rho_p(m)) = l_p \leq l_q \leq c(\rho_q(m))$, so $c(\rho_q(m)) \preceq_p c(\rho_p(m))$. By the same argumentation as above, we deduce $k_{13}(m) = \checkmark$.
\end{proof}

\begin{lem}
\label{lem:fritzwilke:de_order_reftran}
	Let $\mathcal{A}$ be a DPA. Then $\leq_\text{de}$ is reflexive and transitive.
\end{lem}

\begin{proof}
	For reflexivitiy, we need to show that $q \leq_\text{de} q$ for all states $q$. This is rather easy to see. For a word $\alpha \in \Sigma^\omega$, the third component of states in the run of $\mathcal{A}_\text{de}(q, q)$ on $\alpha$ is always $\checkmark$, as $\gamma(i, i, \checkmark) = \checkmark$.
	
	For transitivity, let $q_1 \leq_\text{de} q_2$ and $q_2 \leq_\text{de} q_3$. Assume towards a contradiction that $q_1 \not\leq_\text{de} q_3$, so there is a word $\alpha \notin L(\mathcal{A}_\text{de}(q_1, q_3))$. We consider the three runs $\rho_{12}$, $\rho_{23}$, and $\rho_{13}$ of $\mathcal{A}_\text{de}(q_1, q_2)$, $\mathcal{A}_\text{de}(q_2, q_3)$, and $\mathcal{A}_\text{de}(q_1, q_3)$ respectively on $\alpha$. Then $\rho_{12}$ and $\rho_{23}$ are accepting, whereas $\rho_{13}$ is not. 
	
	Moreover, we use the notation $q_1(i), q_2(i), q_3(i)$ for the states of the run and $k_{12}(i), k_{23}(i), k_{13}(i)$ for the obligations. More specifically for a run $\rho_{ij}$, it is true that $\rho_{ij}(n) = (q_i(n), q_j(n), k_{ij}(n))$.
	
	Let $n_0$ be a position as described in Lemma \ref{lem:fritzwilke:n0_exists} (for $\rho_{13}$) and let $l_j = \min \{ c(q_j(i)) \mid i \geq n_0 \}$ be the lowest priority that $q_j$ reaches after $n_0$.	This is equivalent to $l_j = \min \text{Inf}(\{ c(q_j(i)) \mid i \in \mathbb{N}\})$. We now show that $l_3 \preceq_p l_1$. By Lemma \ref{lem:fritzwilke:run_goodness_implies_de} this gives us $\alpha \in L(\mathcal{A}_\text{de}(q_1, q_3))$, letting us conclude in a contradiction.
	
	\paragraph{Case 1: $l_2$ is even.} We claim that $l_3$ is even and $l_3 \leq l_2$. 
	
	First, to show $l_3 \leq l_2$, let $m \geq n_0$ be a position with $c(q_2(m)) = l_2$ and let $n \geq m$ be the minimal position with $k_{23}(n) = \checkmark$. If $m = n$, then $c(q_3(n)) \preceq_p c(q_2(n)) = l_2$ and therefore $c(q_3(n)) \leq l_2$. Otherwise, from $m$ to $n-1$, $k_{23}$ only grows smaller and is at most $l_2$ (Lemma \ref{lem:fritzwilke:k_shrink}). As the priority of $q_2$ never becomes an odd number smaller than $l_2$, the only way for $k_{23}(m)$ to be $\checkmark$ is that $c(q_3(m))$ is even and $c(q_3(m)) \leq k_{23}(m-1) \leq l_2$.
	
	Second, assume that $l_3$ is odd and let $m$ be a position with $c(q_3(m)) = l_3$. As $l_2$ is even, we have $k_{23}(m) \leq l_3 < l_2$. At no future position can $c(q_3)$ both be even and smaller than $k_{23}$, so $k_{23}$ never becomes $\checkmark$ again. Thus, $\rho_{23}$ is not accepting.
	
	We claim that $l_1$ is odd or $l_1 \geq l_2$.
	
	Towards a contradiction assume the opposite, so $l_1 < l_2$ and $l_1$ is even. Let $m \geq n_0$ be a position with $c(q_1(m)) = l_1$. Then $c(q_2(m)) \not\preceq_p c(q_1(m))$ and therefore $k_{12}(m) = l_1$. At no position after $m$ can it happen that the conditions for $k_{12}$ to become $\checkmark$ again are satisfied. Thus, $\rho_{12}$ would not be accepting.
	
	If $l_1$ is odd and $l_3$ is even, $l_3 \preceq_p l_1$ follows. For the other case, $l_1$ and $l_3$ both being even with $l_3 \leq l_2 \leq l_1$, that also holds.
	
	\paragraph{Case 2: $l_2$ is odd.} We skip the details of this case as it works symmetrically to case 1. In particular, we first show that $l_1$ is odd and $l_1 \leq l_2$. We continue with $l_3$ being even or $l_3 \geq l_2$. From these two statements, $l_3 \preceq_p l_1$ again follows.
\end{proof}

\begin{theorem}
	Let $\mathcal{A}$ be a DPA. Then $\equiv_\text{de}$ is a congruence relation.
\end{theorem}

\begin{proof}
	The three properties that are required for $\equiv_\text{de}$ to be a equivalence relation are rather easy to see. Reflexivity and transitivity have been shown for $\leq_\text{de}$ already and symmetry follows from the definition. Congruence requires more elaboration.

	Let $p \equiv_\text{de} q$ be two equivalent states. Let $a \in \Sigma$ and $p' = \delta(p, a)$ and $q' = \delta(q, a)$. We have to show that also $p' \equiv_\text{de} q'$. Towards a contradiction, assume that $p' \not\leq_\text{de} q'$, so there is a word $\alpha \notin L(\mathcal{A}_\text{de}(p', q'))$. Let $(p', q', k) = \delta_\text{de}((p, q, \checkmark), a)$. By Lemma \ref{lem:fritzwilke:gamma_mono_ext}, the run of $\mathcal{A}_\text{de}$ on $\alpha$ from $(p', q', k)$ cannot be accepting; otherwise, the run of $\mathcal{A}_\text{de}$ from $(p', q', \checkmark)$ would be accepting and $\alpha \in L(\mathcal{A}_\text{de}(p', q'))$. Hence, $a \alpha \notin L(\mathcal{A}_\text{de}(p, q))$, which means that $p \not\equiv_\text{de} q$.
\end{proof}

We want to mention here that $\equiv_\text{de}$ is actually an equivalence relation on APAs as well, as was shown in the original paper. However, congruence is the key point at which deterministic automata diverge. Congruence requires something to be true for \emph{all} successors of a state; delayed simulation only requires there to be \emph{one} equivalent pair of successors. Only in deterministic automata is it that these two coincide.



\vspace{1cm}
\subsubsection*{Alternative characterization}
Having described several properties of $\leq_\text{de}$ and $\equiv_\text{de}$ in the previous part, we briefly want to give an alternative description in corollary \ref{cor:fritzwilke:equivde_alternative} of what it means for two states to be de-simulation equivalent. The intention is to give a more intuitive understanding as well as provide a framework to make future proofs more comfortable.

\begin{lem}
\label{lem:fritzwilke:preceq_alternative}
	Let $\mathcal{A}$ be a DPA with states $p, q \in Q$. Then $p \preceq_\text{de} q$ if and only if the following property holds for all $w \in \Sigma^*$: 
	
	Let $p' = \delta^*(p, w)$ and $q' = \delta^*(q, w)$. If $c(p')$ is even and $c(p') < c(q')$, then every path from $q'$ eventually reaches a priority at most $c(p')$. On the other hand, if $c(q')$ is odd and $c(q') < c(p')$, then every path from $p'$ eventually reaches a priority at most $c(q')$.
\end{lem}

\begin{proof}
	\textbf{If} We show the contrapositive. Let $p \not\preceq_\text{de} q$, so there is a word $\alpha \notin L(\mathcal{A}_\text{de}(p, q))$ with the run of $\mathcal{A}_\text{de}(p, q)$ being $(p_0, q_0, k_0) (p_1, q_1, k_1) \dots$. By Lemma \ref{lem:fritzwilke:n0_exists}, there is a position $n$ such that $k_i$ does not change anymore for $i \geq n$. We define $w = \alpha[0, n_0]$ and $\beta = \alpha[n_0+1, ]$ (so $\alpha = w\beta$) and claim that $w$ is a valid counterexample for the right-side property.
	
	The first case is: $c(p') < c(q')$ and $c(p')$ is even. \\
	Reading $\beta$ from $q'$ induces a run that never visits a priority less or equal to $c(p')$: Let $u \sqsubset \beta$ and assume that $c(\delta^*(q', u)) \leq c(p')$. By choice of $n$, we know that $k_{|wu|} = k_{|wu|+1} \neq \checkmark$. This can only happen if the \enquote{else} case of $\gamma$ is hit, meaning that $k_{|wu|+1} = \min \{ k_{|wu|}, c(\delta^*(p', u)), c(\delta^*(q', u)) \}$. Specifically, $c(\delta^*(q', u)) \geq k_{|wu|}$. By choice of $w$ we also have $k_{|wu|} = k_{|w|} \leq c(p')$, so $c(\delta^*(q', u)) = c(p')$.
	
	This, however, means that $c(\delta^*(q', u))$ is even, $c(\delta^*(q', u)) \leq_\checkmark k_{|wu|}$, and $c(\delta^*(q', u)) \preceq_p c(p')$ and thus $k_{|wu|+1} = \checkmark$ which is a contradiction.
	
	The second case, $c(q') < c(p')$ and $c(q')$ is odd, works almost identically so we omit the proof here.
	
	\paragraph{Only If} Again we show the contrapositive: There is a $w \in \Sigma^*$ such that the right-side property is violated. Let this $w$ now be chosen among all these words such that $\min \{ c(p'), c(q') \}$ becomes minimal. We now show that $p \not\preceq_\text{de} q$. 
	
	The first case is: $c(p') < c(q')$ and $c(p')$ is even. \\
	Let $\beta \in \Sigma^\omega$ be a word such that the respective run from $q'$ only sees priorities strictly greater than $c(p')$. Let $(p_0, q_0, k_0) (p_1, q_1, k_1) \dots$ be the run of $\mathcal{A}_\text{de}(p, q)$ on $\alpha = w \beta$. We claim that $k_i \neq \checkmark$ for all $i > |w|$. If that is true, then the run is rejecting and $\alpha \notin L(\mathcal{A}_\text{de}(p, q))$.
	
	Assume towards a contradiction that $k_i$ does become $\checkmark$ again at some point. Let $j \geq |w|$ be the minimal position with $k_{j+1} = \checkmark$. Then by definition of $\gamma$, $c(q_{j+1}) \leq k_j$ is even or $c(p_{j+1}) \leq k_j$ is odd. In the former case, we would have a contradiction to the choice of $\beta$. In the latter case, we would have a contradiction to the choice of $w$ as a word with minimal priority at $c(p')$: since $c(p')$ is even, $c(p_{j+1}) < c(p')$ and from $q_{j+1}$ there is a run that never reaches a smaller priority. Hence, $w \cdot \beta[0, j-|w|]$ would have been our choice for $w$ instead.
	
	The second case, $c(q') < c(p')$ and $c(q')$ is odd, works almost identically so we omit the proof here.
\end{proof}

While this characterization of $\preceq_\text{de}$ seems arbitrary, it allows for an easier definition of $\equiv_\text{de}$ as is seen in the following statement.

\begin{cor}
\label{cor:fritzwilke:equivde_alternative}
	Let $\mathcal{A}$ be a DPA with states $p, q \in Q$. Then $p \equiv_\text{de} q$ if and only if the following holds for all words $w \in \Sigma^*$:
	
	Let $p' = \delta^*(p, w)$ and $q' = \delta^*(q, w)$. Every run that starts in $p'$ or $q'$ eventually sees a priority less than or equal to $\min \{c(p'), c(q')\}$.
\end{cor}



\vspace{1cm}
\subsubsection*{Using delayed simulation as a merger}
In general, delayed simulation does not imply equal priorities. It is not obvious if a quotient automaton can be built by choosing an arbitrary representative of each equivalence class and in fact example %TODO
shows that this is not the case. However, the following theorem solves this issue.

\begin{lem}
\label{lem:fritzwilke:equiv_states_same_minpri}
	Let $\mathcal{A}$ be a DPA and let $\pi$ and $\rho$ be runs of $\mathcal{A}$ on the same word $\alpha$ but starting at different states. If $\pi(0) \equiv_\text{de} \rho(0)$, then $\min \text{Occ}(c(\pi)) = \min \text{Occ}(c(\rho))$.
\end{lem}

\begin{proof}
	Let $\pi(0) = p$ and $\rho(0) = q$. Assume towards a contradiction that $\min \text{Occ}(c(\pi)) < \min \text{Occ}(c(\rho))$. Let $k = \min \text{Occ}(c(\pi))$ and let $n$ be the first position at which $c(\pi(n)) = k$. Let $p' = \pi(n)$ and $q' = \rho(n)$. These two states are reachable from $p$ and $q$ respectively with the word $\alpha[0,n]$.
	
	By Corollary \ref{cor:fritzwilke:equivde_alternative}, the run $\rho[n,\omega]$ must eventually see a priority at most $k$ which contradicts the assumption.
\end{proof}


\begin{theorem} 
\label{thm:fritzwilke:combine_priorities}
	Let $\mathcal{A} = (Q, \Sigma, \delta, c)$ be a DPA and let $p, q \in Q$ with $p \equiv_\text{de} q$ and $c(p) < c(q)$. Define $\mathcal{A}' = (Q, \Sigma, \delta, c')$ with $c'(s) = \begin{cases} c(p) & \text{if } s = q \\ c(s) & \text{else} \end{cases}$. Then $\mathcal{A} \equiv_L \mathcal{A}'$.
\end{theorem}

\begin{proof}
	Let $q_0 \in Q$ be an arbitrary state. We show that $L(\mathcal{A}, q_0) = L(\mathcal{A}', q_0)$.
	
	First, consider the case that $c(p)$ is an even number. The parity of each state is at least as good in $\mathcal{A}'$ as it is in $\mathcal{A}$, so $L(\mathcal{A}, q_0) \subseteq L(\mathcal{A}', q_0)$. For the other direction, assume there is an $\alpha \in L(\mathcal{A}', q_0) \setminus L(\mathcal{A}, q_0)$, so the respective run $\rho \in Q^\omega$ is accepting in $\mathcal{A}'$ but not in $\mathcal{A}$. 
	
	For this to be true, $\rho$ must visit $q$ infinitely often and $c'(q)$ must be the lowest priority that occurs infinitely often; otherwise, the run would have the same acceptance in both automata. Thus, there is a finite word $w \in \Sigma^*$ such that from $q$, $\mathcal{A}$ reaches again $q$ via $w$ and inbetween only priorities greater than $c'(q)$ are seen.
	
	Now consider the word $w^\omega$ and the run $\pi_q$ of $\mathcal{A}$ on said word starting in $q$. With the argument above, we know that the minimal priority occuring in $c(\pi_q)$ is greater than $c'(q)$. If we take the run $\pi_p$ on $w^\omega$ starting at $p$ though, we find that this run sees priority $c(p) = c'(q)$ at the very beginning. This contradicts Lemma \ref{lem:fritzwilke:equiv_states_same_minpri}, as $p \equiv_\text{de} q$. Thus, the described $\alpha$ cannot exist. 
	
	If $c(p)$ is an odd number, a very similar argumentation can be applied with the roles of $\mathcal{A}$ and $\mathcal{A}'$ reversed. We omit this repetition.
\end{proof}

\vspace{5pt}

Together with Lemma \ref{lem:general:congrel_prio_implies_moore}, this allows for a merger function that preservers language.

\begin{defn}
	We define the \emph{delayed simulation merger} as $\mu_\text{de} : \mathfrak{C}(\equiv_\text{de}) \rightarrow 2^Q$ with $\mu_\text{de}(\kappa) = c^{-1}(\min c(\kappa))$.
\end{defn}

\begin{cor}
	Let $\mathcal{A}$ be a DPA. Then $\mathcal{A}$ is language equivalent to every representative merge w.r.t. $\mu_\text{de}$.
\end{cor}

\begin{proof}
	With Theorem \ref{thm:fritzwilke:combine_priorities}, we can construct a language equivalent $\mathcal{A}' = (Q, \Sigma, \delta, c')$ such that for all $\kappa \in \mathfrak{C}(\equiv_\text{de})$, all states in $\kappa$ have the same priority in $c'$. Every representative merge of $\mathcal{A}$ w.r.t. $\mu_\text{de}$ is a representative merge of $\mathcal{A}'$ w.r.t. $\mu_\text{de}$.
	
	In $\mathcal{A}'$, $\mu_\text{de}$ is the same as $\mu^{\equiv_\text{de}}_\div$. The fact that representative merges of $\mathcal{A}'$ w.r.t. $\mu_\text{de}$ preserve language follows from Lemma \ref{lem:general:congrel_prio_implies_moore}.
\end{proof}

\vspace{10pt}

Finally, we provide a quick analysis of the computational effort for delayed simulation.

\begin{theorem}
	For a given DPA $\mathcal{A}$, $\equiv_\text{de}$ can be computed in $\mathcal{O}(|Q|^2 \cdot |c(Q)|)$.
\end{theorem}

\begin{proof}
	$\mathcal{A}_\text{de}$ is of size $\mathcal{O}(|Q|^2 \cdot |c(Q)|)$. Building it is a straight-forward construction. Once the delayed simulation automaton is built, we have to solve the universal problem, i.e. find all states from which all words are accepted. For DBAs this can be done in linear time with depth search or similar algorithms.
\end{proof}






\vspace{10pt}

\subsubsection{Resetting obligations} 
In the delayed simulation automaton, \enquote{obligations} correspond to good priorities that the first state has accumulated or bad priorities that the second state has accumulated and the need for the respective other state to compensate in some way. The intuitive idea behind this concept is that an obligation that cannot be compensated, stands for an infinite run in which the acceptance differs between the two states that are being compared. The issue with the original definition is that obligations carry over, even if they can only be caused finitely often. This is demonstrated in figure \ref{fig:fritzwilke:reset_oblig_example}; the two states could be merged into one, but they are not $\equiv_\text{de}$-equivalent as can be seen in the delayed simulation automaton in figure \ref{fig:fritzwilke:reset_oblig_example_dea}.

\begin{figure}
\centering
\begin{tikzpicture}[shorten >=1pt,node distance=2cm,on grid,initial text=]
  \node[state,initial]   (0)                {0};
  \node[state]           (1) [right=of 0] {1};
  \path[->] (0) edge node [above] {a} (1)
            (1) edge [loop right] node {a} (1);
\end{tikzpicture}
\caption{Example automaton in which the states could be merged but delayed simulation separates them.}
\label{fig:fritzwilke:reset_oblig_example}
\end{figure}

\begin{figure}
\centering
\begin{tikzpicture}[shorten >=1pt,node distance=2cm,on grid,initial text=]
  \node[state]   (0)                {$0,1,0$};
  \node[state]   (1) [right=of 0] {$1,1,0$};
  \path[->] (0) edge node [above] {a} (1)
            (1) edge [loop right] node {a} (1);
\end{tikzpicture}
\caption{Delayed simulation automaton for \ref{fig:fritzwilke:reset_oblig_example}.}
\label{fig:fritzwilke:reset_oblig_example_dea}
\end{figure}

As a solution to this, we propose a simple change to the definition of the automaton which resets the obligations every time, either state moves to a new SCC. 

\begin{defn}
	Let $\mathcal{A} = (Q, \Sigma, q_0, \delta, c)$ be a DPA. We define the \emph{delayed simulation automaton with SCC resets} $\mathcal{A}_\text{deR}(p, q) = (Q_\text{de}, \Sigma, \delta_\text{deR}, F_\text{de})$ with $\delta_\text{deR}((p, q, k), a) = \delta_\text{de}((p, q, \text{reset}(p, q, k, a)), a)$. Except for the addition of the reset function, this automaton is the same as $\mathcal{A}_\text{de}$.
	
	If $p$ and $\delta(p, a)$ lie in the same SCC, as well as $q$ and $\delta(q, a)$, then we simply set $\text{reset}(p, q, k, a) =~k$. Otherwise, i.e. if any state changes its SCC, the reset comes into play and we set $\text{reset}(p, q, k, a) = \checkmark$.
	
	We write $p \leq_\text{deR} q$ if $L(\mathcal{A}_\text{deR}(p, q), q_0^\text{de}(p, q)) = \Sigma^\omega$. If also $q \leq_\text{deR} p$ holds, we write $p \equiv_\text{deR} q$.
\end{defn}

As the definition is so similar to the original delayed simulation, most results that we have already proven translate directly to the new relation. 

\begin{theorem}
	$\equiv_\text{deR}$ is a congruence relation.
\end{theorem}

\begin{lem}
	$\leq_\text{de} \;\subseteq\; \leq_\text{deR}$.
\end{lem}

\begin{proof} 
	Consider the two simulation automata $\mathcal{A}_\text{de}(p, q)$ and $\mathcal{A}_\text{deR}(p, q)$ and let $\alpha \in \Sigma^\omega$ be an arbitrary word. Let $(p_i, q_i, k_i)_{i \in \mathbb{N}}$ and $(p_i, q_i, l_i)_{i \in \mathbb{N}}$ be the runs of these two automata on $\alpha$. We claim that $k_i \leq_\checkmark l_i$ at every position. Then, since $L(\mathcal{A}_\text{de}(p, q), q_0^\text{de}(p, q)) = \Sigma^\omega$, both runs must be accepting.
	
	We know $k_0 = l_0$ by definition. For the sake of induction, we look at position $i+1$. If neither $p_i$ nor $q_i$ change their SCC in this step, the statement follows from lemma \ref{lem:fritzwilke:gamma_mono}. Otherwise, $l_{i+1} = \checkmark \geq_\checkmark k_{i+1}$.
\end{proof}

\vspace{5pt}

Unfortunately, Theorem \ref{thm:fritzwilke:combine_priorities} does not translate easily to $\equiv_\text{deR}$. In fact, we were not able to find a characterization for a merger function except for the option to only merge states within single SCCs. Then, however, using the skip merger in combination with normal delayed simulation yields a result just as good.




















