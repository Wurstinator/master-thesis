
\section{Congruence Path Refinement}

In this section we present an algorithm that uses an existing congruence relation and refines it to the point where equivalent states can be \enquote{merged}.

\begin{defn}
	Let $\mathcal{A} = (Q, \Sigma, q_0, \delta, c)$ be a DPA and let $\equiv \,\subseteq Q \times Q$ be an equivalence relation on the state set. For every equivalence class $\kappa \subseteq Q$, let $r_\kappa \in \kappa$ be an arbitrary representative of that class. For a DPA $\mathcal{A}' = (Q', \Sigma, q'_0, \delta', c')$, we say that $\mathcal{A}'$ is a \emph{representative merge of $\mathcal{A}$ w.r.t. $\equiv$} if it satisfies the following:
	\begin{itemize}
		\item $Q' = \{ r_{[q]_\equiv} \subseteq Q \mid q \in Q \}$
		\item $q'_0 = r_{[q_0]_\equiv}$
		\item For all $q \in Q'$ and $a \in \Sigma$: $\delta'(q, a) = r_{[\delta(q, a)]_\equiv}$
		\item $c' = c\upharpoonright_{Q'}$
	\end{itemize}
\end{defn}

\begin{defn}
	Let $\mathcal{A} = (Q, \Sigma, q_0, \delta, c)$ be a DPA and let $R \subseteq Q \times Q$ be a congruence relation on the state space. For any states $p, q \in Q$, let $L_{[p]_R \rightarrow [q]_R} \subseteq \Sigma^*$ be the set of words $w$ such that for any $u \sqsubseteq w$, $(\delta(p, u), q) \in R$ iff $u \in \{\varepsilon, w\}$. In other words, the set contains all minimal words by which the automaton reaches $[q]_R$ from $[p]_R$.
	
	Let $\kappa \subseteq Q$ be an equivalence class of $R$ and let $p, q \in \kappa$. We define $R_\kappa \subseteq \kappa \times \kappa$ as the largest set $(p, q) \in R_\kappa$ iff the following holds for all words $w \in L_{\kappa \rightarrow \kappa}$:
	\begin{itemize}
		\item $(\delta^*(p, w), \delta^*(q, w)) \in R_\kappa$
		\item $\min \{ c(\delta^*(p, u)) \mid u \sqsubset w \} = \min \{ c(\delta^*(q, u)) \mid u \sqsubset w \}$
	\end{itemize}
	
	Finally, we call $\equiv_\text{PR}^R \,= \bigcup_{q \in Q} R_{[q]_R}$ the \emph{path refinement of $R$}.
\end{defn}

\begin{lem}
	The path refinement is a well defined equivalence relation.
	\label{lem:pr:pr_well_def}
\end{lem}

\begin{proof}
	We have to consider the sets $L_{[p]_R \rightarrow [q]_R}$ and the sets $R_\kappa$. For $L_{[p]_R \rightarrow [q]_R}$, the definition works because $R$ has the congruence property. 
	
	For $R_\kappa$, consider the following function $f : 2^{Q \times Q} \rightarrow 2^{Q \times Q}$: 
	$$ f(X) = \{ (p, q) \in X \mid \text{for all } w \in L_{\kappa \rightarrow \kappa}: (\delta^*(p, w), \delta^*(q, w)) \in X \}$$
	$$ Y_\kappa = \{ (p, q) \in Q \times Q \mid \text{for all } w \in L_{\kappa \rightarrow \kappa}: \min \{ c(\delta^*(p, u)) \mid u \sqsubset w \} = \min \{ c(\delta^*(q, u)) \mid u \sqsubset w \} \} $$

	Now Let $X_0 = Y_\kappa$ and $X_{i+1} = f(X_i)$. $f$ is monotone w.r.t. $\subseteq$, so there must be a fixed point $X_\infty$. By Kleene's fixed point theorem and from the definition of $R_\kappa$, we have $X_\infty = \text{gfp}(f) = R_\kappa$.
	
	Every $X_i$ is an equivalence relation on $\kappa$: for $i = 0$, every state is only equivalent to itself, and for $i > 0$, the three properties can easily be verified via induction. Hence, $X_\infty = R_\kappa$ is also an equivalence relation. All $R_\kappa$ are disjoint and thus $\equiv_\text{PR}^R$ has to be an equivalence relation as well.
\end{proof}

\begin{theorem}
	Let $\mathcal{A} = (Q, \Sigma, q_0, \delta, c)$ be a DPA and let $R \subseteq Q \times Q$ be a congruence relation that implies language equivalence. Let $\mathcal{B}$ be a representative merge of $\mathcal{A}$ w.r.t. $\equiv_\text{PR}^R$. Then $L(\mathcal{A}) = L(\mathcal{B})$.
\end{theorem}

\begin{proof}
	Let $n$ be the number of non-trivial equivalence classes in $\equiv_\text{PR}^R$, i.e. classes with size greater than 1. If $n = 0$, then $p \equiv_\text{PR}^R q$ iff $p = q$ and therefore $\mathcal{B} = \mathcal{A}$. 
	
	Now assume for an argument of induction that the statement is true for $n$ and we want to show that it still holds for $n+1$ classes. Let $\kappa \subseteq Q$ be an arbitrary non-trivial equivalence class of $\equiv_\text{PR}^R$. Let $\mathcal{A}' = (Q', \Sigma, q'_0, \delta', c')$ be the representative merge of $\mathcal{A}$ w.r.t. $\equiv_\text{PR}^R \upharpoonright_\kappa$ with the same representative $r_\kappa$ as in $\mathcal{B}$. The path refinement equivalence of $\mathcal{A}'$ then is equal to $\equiv_\text{PR}^R \upharpoonright_{Q'}$ and has $n$ non-trivial equivalence classes (as $\kappa$ was merged into a single state). By induction, $L(\mathcal{A}') = L(\mathcal{B})$. It remains to be proven that $L(\mathcal{A}) = L(\mathcal{A}')$.
	
	Let $\alpha \in \Sigma^\omega$ be a word with runs $\rho \in Q^\omega$ and $\rho' \in (Q')^\omega$ of $\mathcal{A}$ and $\mathcal{A}'$ respectively. Let $\lambda \subseteq Q$ be the equivalence class of $R$ from which $\kappa$ was extracted.

	\vspace{5pt}
	\textbf{Claim 1}: At every position $i$, $\rho(i) \in \kappa$ iff $\rho'(i) \in \kappa$. \\
	Let $k_0$ be the first position at which $\rho(k_0) \in \kappa$ is true. For all $i < k_0$, we have $\rho(i) = \rho'(i)$, and at $k_0$ we have $\rho(k_0) \equiv_\text{PR}^R r_\kappa = \rho'(k_0)$.
	
	Now assume that the claim holds for all $i \leq k$, where $k$ is a position at which $\rho(k) \in \kappa$. Let $l > k$ be the next position at which $\rho(l) \in \lambda$. If $l$ does not exist, then neither $\rho(i)$ nor $\rho'(i)$ are elements of $\kappa$ for any $i > k$.
	
	Let $w = \alpha[k, l]$. Since $\kappa \subseteq \lambda$, $w \in L_{\lambda \rightarrow \lambda}$. By definition of $\equiv_\text{PR}^R$, that means $\delta^*(\rho(k), w) = \rho(l) \equiv_\text{PR}^R \delta^*(\rho'(k), w)$. Between $k$ and $l-1$, no redirected edge is used in $\rho'$, so $\delta^*(\rho'(k), \alpha[k, l-1]) = \rho'(l-1)$. Finally, $\rho'(l) = \delta'(\rho'(l-1), \alpha(l)) = r_{[\delta(\rho'(l-1), \alpha(l)]_{\equiv_\text{PR}^R}} \equiv_\text{PR}^R \delta(\rho'(l-1), \alpha(l)) = \delta^*(\rho'(k), w)$. Thus, $\rho(l) \equiv_\text{PR}^R \rho'(l)$.
	
	Now, if $\rho(l) \in \kappa$, then $\rho(l+1) \in \kappa$ and our proof of induction is complete. If $\rho(l) \notin \kappa$, then $\rho'(l) = \rho(l)$, so the runs visit the same states in all positions until $\kappa$ is reached again. This also completes the proof of our claim.
	
	\vspace{5pt}
	\textbf{Claim 2}: If $\kappa$ only occurs finitely often in $\rho$ and $\rho'$, then $\rho$ is accepting iff $\rho'$ is accepting.
	
	Let $k \in \mathbb{N}$ be the last position at which $\rho(k)$ and $\rho'(k)$ are in $\kappa$. From this point on, $\rho'[k, \omega]$ is also a valid run of $\mathcal{A}$ on $\alpha[k, \omega]$. $\rho(k) \equiv_\text{PR}^R \rho'(k)$, so $(\rho(k), \rho'(k)) \in R$. As $R$ implies language equivalence, reading $\alpha[k, \omega]$ from either state in $\mathcal{A}$ leads to the same acceptance status. This also means that $\rho'(k)$ has the same acceptance status as $\rho(k)$.
	
	\vspace{5pt}
	\textbf{Claim 3}: If $\kappa$ occurs infinitely often in $\rho$ and $\rho'$, then $\rho$ is accepting iff $\rho'$ is accepting.
	
	Let $(k_i)_{i \in \mathbb{N}}$ be all positions at which $\kappa$ is visited. For each $k_i$, let $l_i > k_i$ be the minimal position at which $\rho(l_i) \in \lambda$. In two steps, we first show that $c(\rho[l_i, k_{i+1}]) = c'(\rho'[l_i, k_{i+1}])$ and second that $\min \text{Occ}(c(\rho[k_i, l_i])) = \min \text{Occ}(c'(\rho'[k_i, l_i]))$. Together, these results mean that the minimal priority that is seen infinitely often in the two runs is the same.
	
	First, observe that at every $l_i$, we either have $l_i = k_{i+1}$ (if $\rho(l_i) \in \kappa$) or $\rho(l_i) = \rho'(l_i)$. In the first case, $\rho[l_i, k_{i+1}]$ is empty, so $c(\varepsilon) = c'(\varepsilon)$ is true. In the second case, $\rho[l_i, k_{i+1}] = \rho'[l_i, k_{i+1}]$ and therefore $c(\rho[l_i, k_{i+1}]) = c'(\rho'[l_i, k_{i+1}])$.
	
	Second, let $w_i = \alpha[k_i, l_i]$. Then $\alpha \in L_{\lambda \rightarrow \lambda}$ and $\min \text{Occ}(c(\rho[k_i, l_i])) = \min \text{Occ}(c'(\rho'[k_i, l_i]))$ holds directly by definition of $\equiv_\text{PR}^R$.
\end{proof}



\subsection{Algorithmic Definition}
The definition of path refinement that we introduced is useful for the proofs of correctness. It however does not provide one with a way to actually compute the relation. That is why we now provide an alternative definition that yields the same results but is more algorithmic in nature.

\begin{defn}
	Let $\mathcal{A} = (Q, \Sigma, q_0, \delta, c)$ be a DPA and let $R \subseteq Q \times Q$ be a congruence relation. For each equivalence class $\lambda$ of $R$, we define the \emph{path refinement automaton} $\mathcal{G}_\text{PR}^{R,\lambda}(p, q) = (Q_\text{PR}, \Sigma, q_{0, \text{PR}}^{p,q}, \delta^\lambda_\text{PR}, F_\text{PR})$, which is a DFA.
	
	\begin{itemize}
		\item $Q_\text{PR} = (Q \times Q \times c(Q) \times \{<, >, =\}) \cup \{ \perp \}$
		\item $q_{0, \text{PR}}^{p,q} = (p, q, \eta_k(c(p), c(q), \checkmark), \eta_x(c(p), c(q), \checkmark, =))$
		\item $\delta^\lambda_\text{PR}((p, q, k, x), a) = \begin{cases}
			(p', q', \eta_k(c(p'), c(q'), k), \eta_x(c(p'), c(q'), k, x)) & \text{if } p' \notin \lambda \\
			q_{0,\text{PR}}^{p',q'} & \text{if } p' \in \lambda \text{ and } (x =\, =) \\
			\perp & \text{else}
		\end{cases}$ \\
			where $p' = \delta(p, a)$ and $q' = \delta(q, a)$. \\
			$\eta_k(k_p, k_q, k) = \min_{\leq_\checkmark} \{k_p, k_q, k\}$ \\
			$\eta_x(k_p, k_q, k, x) = \begin{cases}
				< & \text{if } (k_p <_\checkmark k_q \text{ and } k_p <_\checkmark k) \text{ or } (k < k_q \text{ and } (x =\, <)) \\
				> & \text{if } (k_p >_\checkmark k_q \text{ and } k >_\checkmark k_q) \text{ or } (k_p > k \text{ and } (x =\, >)) \\
				= & \text{else}
			\end{cases}$ 
		\item $F_\text{PR} = Q_\text{PR} \setminus \{\perp\}$
	\end{itemize}
\end{defn}

\begin{lem}
	Let $\mathcal{A}$ be a DPA with a congruence relation $R$. Let $\lambda$ be an equivalence class of $R$, $p, q \in \lambda$, and $w \in L_{\lambda \rightarrow \lambda}$. For every $v \sqsubset w$ and $\oplus \in \{<, >, =\}$, the fourth component of $(\delta_\text{PR}^\lambda)^*(q_{0,\text{PR}}, v)$ is $\oplus$ if and only if $\min \{ c(\delta^*(p, u)) \mid u \sqsubseteq v \} \oplus \min \{ c(\delta^*(q, u)) \mid u \sqsubseteq v \}$.
	\label{lem:pr:pr_game_nx}
\end{lem}

	The proof of this Lemma is a very formal analysis of every case in the relations between the different priorities that occur and making sure that the definition of $\eta_x$ covers these correctly. No great insight is gained, which is why we omit the proof at this point.

\begin{theorem}
	Let $\mathcal{A}$ be a DPA with a congruence relation $R$. Let $\lambda$ be an equivalence class of $R$ and $p, q \in \lambda$. Then $p \equiv_\text{PR}^R q$ iff $L(\mathcal{G}_\text{PR}^{R,\lambda}(p, q)) = \Sigma^*$.
\end{theorem}

\begin{proof}
	\textbf{If } Let $p \not\equiv_\text{PR}^R q$. Similarly to the proof of Lemma \ref{lem:pr:pr_well_def}, we use the inductive definition of $R_\kappa \subseteq\, \equiv_\text{PR}^R$ using $f$ and the sets $X_i$ here. Let $m$ be the smallest index at which $(p, q) \notin X_m$. Let $\rho = (p_i, q_i, k_i, x_i)_{0 \leq i \leq |w|}$ be the run of $\mathcal{G}_\text{PR}^{R,\lambda}(p, q)$ on $w$. We prove that $\rho(|w|) = \perp$ and therefore $\rho$ is not accepting by induction on $m$.
	
	If $m = 0$, then $(p, q) \notin Y_\lambda$, meaning that there is a word $w$ such that $\min \{ c(\delta^*(p, u)) \mid u \sqsubset w \} \neq \min \{ c(\delta^*(q, u)) \mid u \sqsubset w \}$. Without loss of generality, assume $\min \{ c(\delta^*(p, u)) \mid u \sqsubset w \} < \min \{ c(\delta^*(q, u)) \mid u \sqsubset w \}$. By Lemma \ref{lem:pr:pr_game_nx}, $x_{|w|-1} =\, <$. Furthermore, $\delta(p_{|w|-1}, w_{|w|-1}) \in \lambda$, as $w \in L_{\lambda \rightarrow \lambda}$. Thus, $\rho(|w|) =\, \perp$ and the run is rejecting.
	
	Now consider $m+1 > 1$. Since $(p, q) \in X_m \setminus f(X_m)$, there must be a word $w \in L_{\lambda \rightarrow \lambda}$ such that $(p', q') \notin X_m$, where $p' = \delta^*(p, w)$ and $q' = \delta^*(q, w)$. As $R_\kappa \subseteq X_m$, $(p', q') \notin R_\kappa$ and therefore $p' \not\equiv_\text{PR}^R q'$. By induction, $w \notin L(\mathcal{G}_\text{PR}^{R,\lambda}(p', q'))$; since that run is a suffix of $\rho$, $\rho$ itself is also a rejecting run.
	
	\paragraph{Only If} Let $L(\mathcal{G}_\text{PR}^{R,\lambda}(p, q)) \neq \Sigma^*$. Since $\varepsilon$ is always accepted, there is a word $w \in \Sigma^+ \setminus L(\mathcal{G}_\text{PR}^{R,\lambda}(p, q))$, meaning that $\delta_\text{PR}^*(q_{0,\text{PR}}, w) = \perp$. Split $w$ into sub-words $w = u_1 \cdots u_m$ such that $u_1, \dots, u_m \in L_{\lambda \rightarrow \lambda}$. Note that this partition is unique. We show $p \not\equiv_\text{PR}^R q$ by induction on $m$. Let $\rho = (p_i, q_i, k_i, x_i)_{0 \leq i < |w|}$ be the run of $\mathcal{G}_\text{PR}^{R,\lambda}(p, q)$ on $w$.
	
	If $m = 1$, then $w \in L_{\lambda \rightarrow \lambda}$. Since $\rho(|w|) =\, \perp$, it must be true that $x_{|w|-1} \neq\, =$. Without loss of generality, assume $x_{|w|-1} =\, <$. By Lemma \ref{lem:pr:pr_game_nx}, $\min \{ c(\delta^*(p, u)) \mid u \sqsubset w \} < \min \{ c(\delta^*(q, u)) \mid u \sqsubset w \}$. Therefore, $p \not\equiv_\text{PR}^R q$.
	
	Now consider $m+1 > 1$. Let $p' = \delta^*(p, u_1)$ and $q' = \delta^*(q, u_1)$. By induction on the word $u_2 \cdots u_m$, $p' \not\equiv_\text{PR}^R q'$. Since $u_1 \in L_{\lambda \rightarrow \lambda}$, that also means $p \not\equiv_\text{PR}^R q$.
\end{proof}

The differences between different $\mathcal{G}_\text{PR}^{R,\lambda}$ for different $\lambda$ are minor and the question whether the accepted language is universal boils down to a simple question of reachability. Thus, $\equiv_\text{PR}^R$ can be computed in $\mathcal{O}(|\mathcal{G}_\text{PR}^{R,\lambda}|)$ which is $\mathcal{O}(|Q|^2 \cdot |c(Q)|)$.






















