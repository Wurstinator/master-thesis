
\section{Skip Merger}
\label{sect:skipper}

\begin{defn}
	Let $\mathcal{A} = (Q, \Sigma, \delta, c)$ be a DPA and let $\sim \,\subseteq Q \times Q$ be a congruence relation on $\mathcal{A}$. Let $\preceq \,\subseteq Q \times Q$ be a total extension of $\preceq_\text{reach}^\mathcal{A}$. 
	
	We define the \emph{skip merger function} $\mu_\text{skip}^\sim : D \rightarrow 2^Q$ as follows: for each equivalence class $\kappa \in \mathfrak{C}(\sim)$, let $C_\kappa \subseteq \kappa$ be the of $\preceq$-maximal elements in $\kappa$. Let $M_\kappa = \kappa \setminus C_\kappa$. Then we have $D = \{ M_\kappa \mid \kappa \in \mathfrak{C}(\sim) \}$ and $\mu_\text{skip}^\sim(M_\kappa) = C_\kappa$.
\end{defn}

The idea behind the skip merger is that whenever an equivalence class of $\sim$ is reached, the transition is redirected to another element of the same equivalence class that lies as \enquote{deep} in the automaton as possible.

\begin{lem}
\label{lem:skip:skip_aut_linear_time}
	For a given $\mathcal{A}$ and $\sim$, $\mu_\text{skip}^\sim$ can be computed in $\mathcal{O}(|\mathcal{A}|)$.
\end{lem}

\begin{proof}
	As seen in Lemma \ref{lem:general:reach_topo_lintime}, $\preceq$ can be computed in linear time. Assuming that $\sim$ is given by a suitable data structure, each equivalence class can easily be accessed and $\preceq$-maximal elements can be found in linear time.
\end{proof}

\vspace{5pt}

Now that we have established the definition and possible computation of the skip merger function, we want to analyze its structure and prove its correctness. For the rest of this section, we use $\mathcal{A} = (Q, \Sigma, \delta, c)$ as a DPA, $\sim$ as a congruence relation, and $\mathcal{B} = (Q_\mathcal{B}, \Sigma, \delta_\mathcal{B}, c_\mathcal{B})$ as a representative merge of $\mathcal{A}$ w.r.t. $\mu_\text{skip}^\sim$.

\begin{lem}
\label{lem:skip:run_growing}
	Let $\rho$ be a run on $\alpha$ in $\mathcal{B}$. Then for all $i$, $\rho(i) \preceq \rho(i+1)$.
	Furthermore, we have $\rho(i) \prec \rho(i+1)$ if and only if $\rho(i) \prec r_{[\delta(\rho(i), \alpha(i))]_\sim}$.
\end{lem}

\begin{proof}
	Let $i$ be an arbitrary index of the run. If $\rho(i)$ to $\rho(i+1)$ is also a transition in $\mathcal{A}$, then $\rho(i+1)$ is reachable from $\rho(i)$ in $\mathcal{A}$ and hence $\rho(i) \preceq \rho(i+1)$ by definition of the preorder. Otherwise the transition used was redirected in the construction. The way the redirection is defined, this implies $\rho(i) \prec \rho(i+1)$.
	
	We move on to the second part of the lemma. If $\rho(i) \prec r_{[\delta_\mathcal{A}(\rho(i), \alpha(i))]_\sim}$, then the transition is redirected to $\rho(i+1) = r_{[\delta_\mathcal{A}(\rho(i), \alpha(i))]_\sim}$ and the statement holds. 
	
	For the other direction, let $\rho(i) \prec \rho(i+1)$ and assume towards a contradiction that $\rho(i) \not\prec r_{[\delta_\mathcal{A}(\rho(i), \alpha(i))]_\sim}$. This means that the transition was not redirected and $\rho(i+1) = \delta_\mathcal{A}(\rho(i), \alpha(i))$. Since $\preceq$ is total, we have $r_{[\delta_\mathcal{A}(\rho(i), \alpha(i))]_\sim} = r_{[\rho(i+1)]_\sim} \preceq \rho(i) \prec \rho(i+1)$ which contradicts the $\preceq$-maximality of representatives.
\end{proof}

\begin{lem}
\label{lem:skip:equiv_same_scc}
	Let $p, q \in Q_\mathcal{B}$. If $p \sim q$, then $p$ and $q$ lie in the same SCC. 
\end{lem}

\begin{proof}
	It suffices to restrict ourselves to $q = r_{[q]_\sim} = r_{[p]_\sim}$. If we can prove the Lemma for this case, then the general statement follows by transitivity.
	
	Let $p_0$ be a state from which both $p$ and $q$ are reachable. Let $p_0 \cdots p_n$ be a minimal run of $\mathcal{B}$ that reaches $p$. By Lemma \ref{lem:skip:run_growing}, we have $p_0 \preceq \dots \preceq p_n$. Whenever $p_i \prec p_{i+1}$, a redirected transition to the representative $r_{[p_{i+1}]_\sim} = p_{i+1}$ is taken. 
	
	Let $k$ be the first position after which no redirected transition is taken anymore. For the first case, assume that $k < n$. Then $p_i \simeq r_{[p_{i+1}]_\sim}$ for all $i \geq k$. In particular, $p_{n-1} \simeq q$. Since $p_{n-1} \preceq p_n$, we also have $q \preceq p_n$. The representatives are chosen $\preceq$-maximal in their $\sim$-class, so $q \simeq p_n$.
	
	The second case is $k = n$. In that case, the transition from $p_{n-1}$ to $p_n$ is redirected and $p_n = r_{[p_n]_\sim} = q$.
\end{proof}


\begin{lem}
\label{lem:skip:run_suffix}
	Let $\rho \in Q^\omega$ be an infinite run in $\mathcal{B}$ starting at a reachable state. Then $\rho$ has a suffix that is a run in $\mathcal{A}$.
\end{lem} 

\begin{proof}
	We show that only finitely often a redirected transition is used in $\rho$. Then, from some point on, only transitions that also exist in $\mathcal{A}$ are used. The suffix starting at this point is the run that we are looking for.
	
	Let $\rho = p_0 p_1 \cdots$. By Lemma \ref{lem:skip:run_growing}, we have $p_i \preceq p_{i+1}$ for all $i$ and $p_i \prec p_{i+1}$ whenever a redirected transition is taken. As $Q$ is finite, we can only move up in the order finitely often. This proves our claim.
\end{proof}


\begin{theorem}
	Let $\sim \,\subseteq\, \equiv_L$. Then $\mathcal{A}$ and $\mathcal{B}$ are language equivalent.
\end{theorem}

\begin{proof}
	Let $\alpha \in \Sigma^\omega$ be a word and let $\rho$ be of $\mathcal{B}$ starting in $q_0$ on $\alpha$. By Lemma \ref{lem:skip:run_suffix}, $\rho$ has a suffix $\pi$ which is a run segment of $\mathcal{A}$ on some suffix $\beta$ of $\alpha$. The acceptance condition of DPAs is prefix independent, so $\alpha \in L(\mathcal{B}, q_0)$ iff $\rho$ is an accepting run iff $\pi$ is an accepting run. Since the priorities do not change during the construction, $\pi$ is accepting in $\mathcal{B}$ iff it is accepting in $\mathcal{A}$.
	
	Let $w \in \Sigma^*$ be the prefix of $\alpha$ with $\alpha = w \beta$. By Lemma \ref{lem:general:cong_stays_in_merge}, we know that $\delta^*(q_0, w) \sim \delta^*_\mathcal{B}(q_0, w)$. Since every state is $\sim$-equivalent to its representative and $\sim$ is a congruence relation, we also know $\delta^*_\mathcal{B}(q_0, w) \sim \delta^*_\mathcal{B}(r_{[q_0]_\sim}, w)$. From $\delta^*_\mathcal{B}(r_{[q_0]_\sim}, w)$, the run $\pi$ accepts $\beta$ iff $\alpha \in L(\mathcal{B}, q_0)$. As $\sim$ implies language equivalence, the same must hold for $\delta^*_\mathcal{A}(q_0, w)$. Therefore, $\alpha \in L(\mathcal{A}, q_0)$ iff $\alpha \in L(\mathcal{B}, q_0)$.
\end{proof}


