
\section{Schewe}
\label{sect:schewe}

This section is based heavily on \cite{Schewe2010} and mostly adapts their results from B\"uchi to parity automata.

\begin{defn}
	Let $\mathcal{A} = (Q, \Sigma, \delta, c)$ be a DPA and let $\sim \,\subseteq Q \times Q$ be a congruence relation on $\mathcal{A}$. Let $\preceq \,\subseteq Q \times Q$ be a total extension of $\preceq_\text{reach}^\mathcal{A}$. 
	
	We define the \emph{Schewe merger function} $\mu_\text{Sch}^\sim : D \rightarrow 2^Q$ as follows: for each equivalence class $\kappa \in \mathfrak{C}(\sim)$, let $C_\kappa \subseteq \kappa$ be the of $\preceq$-maximal elements in $\kappa$. Let $M_\kappa = \kappa \setminus C_\kappa$. Then we have $D = \{ M_\kappa \mid \kappa \in \mathfrak{C}(\sim) \}$ and $\mu_\text{Sch}^\sim(M_\kappa) = C_\kappa$.
	
	We call a representative merge of $\mathcal{A}$ w.r.t. $\mu_\text{Sch}^\sim$ a \emph{Schewe automaton}.
\end{defn}

The idea behind the Schewe merger is that whenever an equivalence class of $\sim$ is reached, the transition is redirected to another element of the same equivalence class that lies as \enquote{deep} in the automaton as possible.

\begin{lem}
\label{lem:schewe:schewe_aut_linear_time}
	For a given $\mathcal{A}$ and $\sim$, $\mu_\text{Sch}^\sim$ can be computed in $\mathcal{O}(|\mathcal{A}|)$.
\end{lem}

\begin{proof}
	As seen in Lemma \ref{lem:general:reach_topo_lintime}, $\preceq$ can be computed in linear time. Assuming that $\sim$ is given by a suitable data structure, each equivalence class can easily be accessed and $\preceq$-maximal elements can be found in linear time.
\end{proof}

\vspace{10pt}

Now that we have established the definition and possible computation of the Schewe merger function, we want to analyze its structure and prove its correctness. For the rest of this section, we use $\mathcal{A} = (Q, \Sigma, \delta, c)$ as a DPA, $\sim$ as a congruence relation, and $\mathcal{S} = (Q_\mathcal{S}, \Sigma, \delta_\mathcal{S}, c_\mathcal{S})$ as a Schewe automaton with $\mu_\text{Sch}^\sim$.

\begin{lem}
\label{lem:schewe:run_growing}
	Let $\rho$ be a run on $\alpha$ in $\mathcal{S}$. Then for all $i$, $\rho(i) \preceq \rho(i+1)$.
	Furthermore, we have $\rho(i) \prec \rho(i+1)$ if and only if $\rho(i) \prec r_{[\delta(\rho(i), \alpha(i))]_\sim}$.
\end{lem}

\begin{proof}
	Let $i$ be an arbitrary index of the run. If $\rho(i)$ to $\rho(i+1)$ is also a transition in $\mathcal{A}$, then $\rho(i+1)$ is reachable from $\rho(i)$ in $\mathcal{A}$ and hence $\rho(i) \preceq \rho(i+1)$ by definition of the preorder. Otherwise the transition used was redirected in the construction. The way the redirection is defined, this implies $\rho(i) \prec \rho(i+1)$.
	
	We move on to the second part of the lemma. If $\rho(i) \prec r_{[\delta_\mathcal{A}(\rho(i), \alpha(i))]_\sim}$, then the transition is redirected to $\rho(i+1) = r_{[\delta_\mathcal{A}(\rho(i), \alpha(i))]_\sim}$ and the statement holds. 
	
	For the other direction, let $\rho(i) \prec \rho(i+1)$ and assume towards a contradiction that $\rho(i) \not\prec r_{[\delta_\mathcal{A}(\rho(i), \alpha(i))]_\sim}$. This means that the transition was not redirected and $\rho(i+1) = \delta_\mathcal{A}(\rho(i), \alpha(i))$. Since $\preceq$ is total, we have $r_{[\delta_\mathcal{A}(\rho(i), \alpha(i))]_\sim} = r_{[\rho(i+1)]_\sim} \preceq \rho(i) \prec \rho(i+1)$ which contradicts the $\preceq$-maximality of representatives.
\end{proof}

\begin{lem}
\label{lem:schewe:equiv_same_scc}
	Let $p, q \in Q_\mathcal{S}$. If $p \sim q$, then $p$ and $q$ lie in the same SCC. 
\end{lem}

\begin{proof}
	It suffices to restrict ourselves to $q = r_{[q]_\sim} = r_{[p]_\sim}$. If we can prove the Lemma for this case, then the general statement follows by transitivity.
	
	Let $p_0$ be a state from which both $p$ and $q$ are reachable. Let $p_0 \cdots p_n$ be a minimal run of $\mathcal{S}$ that reaches $p$. By Lemma \ref{lem:schewe:run_growing}, we have $p_0 \preceq \dots \preceq p_n$. Whenever $p_i \prec p_{i+1}$, a redirected transition to the representative $r_{[p_{i+1}]_\sim} = p_{i+1}$ is taken. 
	
	Let $k$ be the first position after which no redirected transition is taken anymore. For the first case, assume that $k < n$. Then $p_i \simeq r_{[p_{i+1}]_\sim}$ for all $i \geq k$. In particular, $p_{n-1} \simeq q$. Since $p_{n-1} \preceq p_n$, we also have $q \preceq p_n$. The representatives are chosen $\preceq$-maximal in their $\sim$-class, so $q \simeq p_n$.
	
	The second case is $k = n$. In that case, the transition from $p_{n-1}$ to $p_n$ is redirected and $p_n = r_{[p_n]_\sim} = q$.
\end{proof}


\begin{lem}
\label{lem:schewe:run_suffix}
	Let $\rho \in Q^\omega$ be an infinite run in $\mathcal{S}$ starting at a reachable state. Then $\rho$ has a suffix that is a run in $\mathcal{A}$.
\end{lem} 

\begin{proof}
	We show that only finitely often a redirected transition is used in $\rho$. Then, from some point on, only transitions that also exist in $\mathcal{A}$ are used. The suffix starting at this point is the run that we are looking for.
	
	Let $\rho = p_0 p_1 \cdots$. By Lemma \ref{lem:schewe:run_growing}, we have $p_i \preceq p_{i+1}$ for all $i$ and $p_i \prec p_{i+1}$ whenever a redirected transition is taken. As $Q$ is finite, we can only move up in the order finitely often. This proves our claim.
\end{proof}


\begin{theorem}
	Let $\sim \,\subseteq\, \equiv_L$. Then $\mathcal{A}$ and $\mathcal{S}$ are language equivalent.
\end{theorem}

\begin{proof}
	Let $\alpha \in \Sigma^\omega$ be a word and let $\rho$ be of $\mathcal{S}$ starting in $q_0$ on $\alpha$. By Lemma \ref{lem:schewe:run_suffix}, $\rho$ has a suffix $\pi$ which is a run segment of $\mathcal{A}$ on some suffix $\beta$ of $\alpha$. The acceptance condition of DPAs is prefix independent, so $\alpha \in L(\mathcal{S})$ iff $\rho$ is an accepting run iff $\pi$ is an accepting run. Since the priorities do not change during the construction, $\pi$ is accepting in $\mathcal{S}$ iff it is accepting in $\mathcal{A}$.
	
	Let $w \in \Sigma^*$ be the prefix of $\alpha$ with $\alpha = w \beta$. By Lemma \ref{lem:general:cong_stays_in_merge}, we know that $\delta^*(q_0, w) \sim \delta^*_\mathcal{S}(q_0, w)$. Since every state is $\sim$-equivalent to its representative and $\sim$ is a congruence relation, we also know $\delta^*_\mathcal{S}(q_0, w) \sim \delta^*_\mathcal{S}(r_{[q_0]_\sim}, w)$. From $\delta^*_\mathcal{S}(r_{[q_0]_\sim}, w)$, the run $\pi$ accepts $\beta$ iff $\alpha \in L(\mathcal{S})$. As $\sim$ implies language equivalence, the same must hold for $\delta^*_\mathcal{A}(q_0, w)$. Therefore, $\alpha \in L(\mathcal{A})$ iff $\alpha \in L(\mathcal{S})$.
\end{proof}

\vspace{5pt}

We have proven that the Schewe merger function can be used to refine congruence relations (such as $\equiv_\text{Ankh}$) that, by themselves are not strong enough criteria to allow for a merging of states, to the point that they can be used for state space reduction. A final result regarding this technique is adapted from \cite{Schewe2010} and shows a relation between this algorithm and priority almost equivalence.



\begin{lem}
	Let $\sim \,=\, \equiv_\text{\Ankh}$ and let $\mathcal{S}'$ be a representative merge of $\mathcal{S}$ w.r.t. $\equiv_M$. There is no smaller DPA than $\mathcal{S}'$ that is priority almost equivalent to $\mathcal{A}$.
\end{lem} %TODO

\begin{proof}
	Let $\mathcal{B}$ be a DPA that is smaller than $\mathcal{S}'$. Our goal is to construct a word on which their priorities differ infinitely often.
	
	First of all, $\mathcal{B}$ must have \enquote{the same} $\sim$-equivalence classes as $\mathcal{S}'$, i.e. for all states $p$ in $\mathcal{S}'$, there is a state $q$ in $\mathcal{B}$ s.t. $\mathcal{B}_q$ and $\mathcal{S}'_p$ are priority almost-equivalent. If this would not be the case, there is a $p$ for which no such $q$ exists. As $\mathcal{S}'$ was minimized, $p$ is reachable by some word $w$. Whatever state $q$ is reached in $\mathcal{B}$ via $w$, $\mathcal{B}_q$ and $\mathcal{S}'_p$ are not priority almost-equivalent and there is some witness $\alpha$ for that. Hence, $w \alpha$ is a witness that $\mathcal{B}$ and $\mathcal{S}'$ are not priority almost-equivalent.
	
	We define $f$ as a function that maps every $\sim$-equivalence class in $\mathcal{S}'$ to its respective class in $\mathcal{B}$. $\mathcal{B}$ has at least as many $\sim$-equivalence classes but less states than $\mathcal{S}'$, so there is a class $\mathfrak{c}$ in $\mathcal{S}'$ such that $f(\mathfrak{c})$ contains less elements than $\mathfrak{c}$.
	
	By Lemma \ref{lem:schewe:equiv_same_scc}, there is a unique SCC $C$ in $\mathcal{S}'$ in which all states in $\mathfrak{c}$ are contained. The same must be true for some SCC $D$ in $\mathcal{B}$ for the states in $f(\mathfrak{c})$. Otherwise we could apply the Schewe automaton construction to $\mathcal{B}$ which does not increase the number of states, is priority almost-equivalent to $\mathcal{B}$, and has this property.
	
	$C$ and $D$ must be non-trivial SCCs. If $C$ would be trivial, then $\mathfrak{c}$ would only contain one element and $f(\mathfrak{c})$ would be empty. Hence, one can force multiple visits to a state in $\mathfrak{c}$ in $\mathcal{S}'$. If $D$ would be trivial, this would not be possible and we could again find a seperating witness of the two automata.
	
	We claim: There is a state $p \in \mathfrak{c}$ s.t. for all $q \in f(\mathfrak{c})$, there is a word $w_q \in \Sigma^*$ that is a witness for non-Moore equivalence of $\mathcal{B}_q$ and $\mathcal{S}'_r$ and $\mathcal{S}'$ does not leave $C$ when reading $w_q$ from $p$. Assume the opposite, i.e. for all $p \in \mathfrak{c}$, there is a $q_p \in f(\mathfrak{c})$ which does not satisfy said property. 
	
	As $|\mathfrak{c}| < |f(\mathfrak{c})|$, there are two states $p_1$ and $p_2$ such that $q_{p_1} = q_{p_2} =: q$. For both $i \in \{1,2\}$ and for each word $w \in \Sigma^*$, we have $\lambda_\mathcal{B}(q, w) = \lambda_{\mathcal{S}'}(p_i, w)$ or $\mathcal{S}'$ leaves the SCC $C$ when reading $w$ from $p_i$. If for both $i$ and all words $w$ the first case would apply, then $p_1$ and $p_2$ would be Moore-equivalent and would have been merged in the minimization process of $\mathcal{S}'$. Hence, there are $i$ (which we assume to be $i=1$ wlog) and $w$ such that $\lambda_\mathcal{B}(q, w) \neq \lambda_{\mathcal{S}'}(p_i, w)$ but $\mathcal{S}'_{p_i}$ leaves $C$ when reading $w$. Let this $w$ be minimal in length.
	
	Let $\rho$ and $\pi$ be the runs of $\mathcal{S}'$ from $p_1$ and $p_2$ via $w$. At some position $k$, $\rho$ leaves $C$. By Lemma \ref{lem:schewe:run_growing}, this means that a redirected transition was taken to $\rho(k+1) = r_{[\rho(k+1)]_\sim}$. $\sim$ is a congruence relation and $p_1 \sim p_2$, so $\pi(k+1) \sim \rho(k+1)$. As $p_1 \simeq p_2$ and $p_1 \prec r_{[\rho(k+1)]_\sim}$, we also have $p_2 \prec r_{[\rho(k+1)]_\sim}$ and $\pi(k+1) = \rho(k+1)$. As $w$ was chosen minimal in length, the priorities of the runs are equal everywhere except for $\delta^*_{\mathcal{S}'}(p_1, w)$ and $\delta^*_{\mathcal{S}'}(p_2, w)$. However, the runs converge at $k+1$ which means that they visit the same states from that point on. In particular, $\delta^*_{\mathcal{S}'}(p_1, w) = \delta^*_{\mathcal{S}'}(p_2, w)$.
	
	We have thus proven that the described state $p \in \mathfrak{c}$ and the words $w_q \in \Sigma^*$ exist. We can use these to finally construct our witness $\alpha$. We define a sequence $(\alpha_n)_{n \in \mathbb{N}} \in \Sigma^*$ such that on $\alpha_n$, the runs of $\mathcal{S}'$ and $\mathcal{B}$ differ at least $n$ times in priority. Then $\alpha := \bigcup_n \alpha_n$ satisfies our requirements. Furthermore, we make sure that after reading $\alpha_n$, $\mathcal{S}'$ always stops in $p$.
	
	Every state in $\mathcal{S}'$ is reachable, so let $\alpha_0$ be a word that reaches $p$ from the initial state. Now assume that $\alpha_n$ was already defined. Let $q = \delta^*_\mathcal{B}(q_0^\mathcal{B}, \alpha_n)$. As $p \in \mathfrak{c}$, we can use the same argument as earlier to find $q \in f(\mathfrak{c})$. There is a suitable word $w_q$ that is a witness for Moore non-equivalence while staying in $C$. As we stay in $C$, there is also a word $u$ such that $\delta^*_{\mathcal{S}'}(p, w_q u) = p$. We set $\alpha_{n+1} := \alpha_n w_q u$. By choice of $w_q$, there is a position during the segment on $w_q u$ at which runs of $\mathcal{S}'$ and $\mathcal{B}$ differ in priority. By induction, $\alpha_{n+1}$ satisfies all our required properties.
\end{proof}





