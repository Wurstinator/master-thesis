
\section{Schewe}
\label{sect:schewe}

This section is based heavily on \cite{Schewe2010} and partially adapts their results from B\"uchi to parity automata.

\begin{defn}
		Let $\mathcal{A} = (Q, \Sigma, \delta, c)$ be a DPA and let $\emptyset \neq C \subseteq M \subseteq Q$. Let $\mathcal{A}' = (Q', \Sigma, \delta', c')$ be another DPA. We call $\mathcal{A}'$ a \emph{Schewe merge of $\mathcal{A}$ w.r.t. $M$ by candidates $C$} if it satisfies the following:
	\begin{itemize}
		\item There is a state $r_M \in C$ such that $Q' = (Q \setminus M) \cup \{r_M\}$.
		\item $c' = c\upharpoonright_{Q'}$.
		\item Let $p \in Q'$ and $\delta(p, a) = q$. If $q \in M$ or if $q \in C$ and $p$ is not reachable from $q$, then $\delta'(p, a) = r_M$. Otherwise, $\delta'(p, a) = q$. 
	\end{itemize}
\end{defn}

The definition of a Schewe merge is almost identical to that of a representative merge. The only difference lies therein that some additional transitions are redirected to the representative: when a transition leads to a candidate that is not in $M$ while also moving to a different SCC.

More redirection allows for better state space reduction in theory. 
%TODO




\begin{defn}
	Let $\mathcal{A} = (Q, \Sigma, \delta, c)$ be a DPA and let $\sim \,\subseteq Q \times Q$ be a congruence relation on $\mathcal{A}$. Let $\preceq \,\subseteq Q \times Q$ be a total extension of $\preceq_\text{reach}^\mathcal{A}$. 
	
	We define the \emph{Schewe merger function} $\mu_\text{Sch}^\sim : D \rightarrow 2^Q$ as follows: for each equivalence class $\kappa \in \mathfrak{C}(\sim)$, let $C_\kappa \subseteq \kappa$ be the of $\preceq$-maximal elements in $\kappa$. Let $M_\kappa = \kappa \setminus C_\kappa$. Then we have $D = \{ M_\kappa \mid \kappa \in \mathfrak{C}(\sim) \}$ and $\mu_\text{Sch}^\sim(M_\kappa) = C_\kappa$.
	
	We call a representative merge of $\mathcal{A}$ w.r.t. $\mu_\text{Sch}^\sim$ a \emph{Schewe automaton}.
\end{defn}

The idea behind the Schewe merger is that whenever an equivalence class of $\sim$ is reached, the transition is redirected to another element of the same equivalence class that lies as \enquote{deep} in the automaton as possible.

\begin{lem}
\label{lem:schewe:schewe_aut_linear_time}
	For a given $\mathcal{A}$ and $\sim$, $\mu_\text{Sch}^\sim$ can be computed in $\mathcal{O}(|\mathcal{A}|)$.
\end{lem}

\begin{proof}
	As seen in Lemma \ref{lem:general:reach_topo_lintime}, $\preceq$ can be computed in linear time. Assuming that $\sim$ is given by a suitable data structure, each equivalence class can easily be accessed and $\preceq$-maximal elements can be found in linear time.
\end{proof}

\vspace{10pt}

Now that we have established the definition and possible computation of the Schewe merger function, we want to analyze its structure and prove its correctness. For the rest of this section, we use $\mathcal{A} = (Q, \Sigma, \delta, c)$ as a DPA, $\sim$ as a congruence relation, and $\mathcal{S} = (Q_\mathcal{S}, \Sigma, \delta_\mathcal{S}, c_\mathcal{S})$ as a Schewe automaton with $\mu_\text{Sch}^\sim$.

\begin{lem}
\label{lem:schewe:run_growing}
	Let $\rho$ be a run on $\alpha$ in $\mathcal{S}$. Then for all $i$, $\rho(i) \preceq \rho(i+1)$.
	Furthermore, we have $\rho(i) \prec \rho(i+1)$ if and only if $\rho(i) \prec r_{[\delta(\rho(i), \alpha(i))]_\sim}$.
\end{lem}

\begin{proof}
	Let $i$ be an arbitrary index of the run. If $\rho(i)$ to $\rho(i+1)$ is also a transition in $\mathcal{A}$, then $\rho(i+1)$ is reachable from $\rho(i)$ in $\mathcal{A}$ and hence $\rho(i) \preceq \rho(i+1)$ by definition of the preorder. Otherwise the transition used was redirected in the construction. The way the redirection is defined, this implies $\rho(i) \prec \rho(i+1)$.
	
	We move on to the second part of the lemma. If $\rho(i) \prec r_{[\delta_\mathcal{A}(\rho(i), \alpha(i))]_\sim}$, then the transition is redirected to $\rho(i+1) = r_{[\delta_\mathcal{A}(\rho(i), \alpha(i))]_\sim}$ and the statement holds. 
	
	For the other direction, let $\rho(i) \prec \rho(i+1)$ and assume towards a contradiction that $\rho(i) \not\prec r_{[\delta_\mathcal{A}(\rho(i), \alpha(i))]_\sim}$. This means that the transition was not redirected and $\rho(i+1) = \delta_\mathcal{A}(\rho(i), \alpha(i))$. Since $\preceq$ is total, we have $r_{[\delta_\mathcal{A}(\rho(i), \alpha(i))]_\sim} = r_{[\rho(i+1)]_\sim} \preceq \rho(i) \prec \rho(i+1)$ which contradicts the $\preceq$-maximality of representatives.
\end{proof}

\begin{lem}
\label{lem:schewe:equiv_same_scc}
	Let $p, q \in Q_\mathcal{S}$. If $p \sim q$, then $p$ and $q$ lie in the same SCC. 
\end{lem}

\begin{proof}
	It suffices to restrict ourselves to $q = r_{[q]_\sim} = r_{[p]_\sim}$. If we can prove the Lemma for this case, then the general statement follows by transitivity.
	
	Let $p_0$ be a state from which both $p$ and $q$ are reachable. Let $p_0 \cdots p_n$ be a minimal run of $\mathcal{S}$ that reaches $p$. By Lemma \ref{lem:schewe:run_growing}, we have $p_0 \preceq \dots \preceq p_n$. Whenever $p_i \prec p_{i+1}$, a redirected transition to the representative $r_{[p_{i+1}]_\sim} = p_{i+1}$ is taken. 
	
	Let $k$ be the first position after which no redirected transition is taken anymore. For the first case, assume that $k < n$. Then $p_i \simeq r_{[p_{i+1}]_\sim}$ for all $i \geq k$. In particular, $p_{n-1} \simeq q$. Since $p_{n-1} \preceq p_n$, we also have $q \preceq p_n$. The representatives are chosen $\preceq$-maximal in their $\sim$-class, so $q \simeq p_n$.
	
	The second case is $k = n$. In that case, the transition from $p_{n-1}$ to $p_n$ is redirected and $p_n = r_{[p_n]_\sim} = q$.
\end{proof}


\begin{lem}
\label{lem:schewe:run_suffix}
	Let $\rho \in Q^\omega$ be an infinite run in $\mathcal{S}$ starting at a reachable state. Then $\rho$ has a suffix that is a run in $\mathcal{A}$.
\end{lem} 

\begin{proof}
	We show that only finitely often a redirected transition is used in $\rho$. Then, from some point on, only transitions that also exist in $\mathcal{A}$ are used. The suffix starting at this point is the run that we are looking for.
	
	Let $\rho = p_0 p_1 \cdots$. By Lemma \ref{lem:schewe:run_growing}, we have $p_i \preceq p_{i+1}$ for all $i$ and $p_i \prec p_{i+1}$ whenever a redirected transition is taken. As $Q$ is finite, we can only move up in the order finitely often. This proves our claim.
\end{proof}


\begin{theorem}
	Let $\sim \,\subseteq\, \equiv_L$. Then $\mathcal{A}$ and $\mathcal{S}$ are language equivalent.
\end{theorem}

\begin{proof}
	Let $\alpha \in \Sigma^\omega$ be a word and let $\rho$ be of $\mathcal{S}$ starting in $q_0$ on $\alpha$. By Lemma \ref{lem:schewe:run_suffix}, $\rho$ has a suffix $\pi$ which is a run segment of $\mathcal{A}$ on some suffix $\beta$ of $\alpha$. The acceptance condition of DPAs is prefix independent, so $\alpha \in L(\mathcal{S})$ iff $\rho$ is an accepting run iff $\pi$ is an accepting run. Since the priorities do not change during the construction, $\pi$ is accepting in $\mathcal{S}$ iff it is accepting in $\mathcal{A}$.
	
	Let $w \in \Sigma^*$ be the prefix of $\alpha$ with $\alpha = w \beta$. By Lemma \ref{lem:general:cong_stays_in_merge}, we know that $\delta^*(q_0, w) \sim \delta^*_\mathcal{S}(q_0, w)$. Since every state is $\sim$-equivalent to its representative and $\sim$ is a congruence relation, we also know $\delta^*_\mathcal{S}(q_0, w) \sim \delta^*_\mathcal{S}(r_{[q_0]_\sim}, w)$. From $\delta^*_\mathcal{S}(r_{[q_0]_\sim}, w)$, the run $\pi$ accepts $\beta$ iff $\alpha \in L(\mathcal{S})$. As $\sim$ implies language equivalence, the same must hold for $\delta^*_\mathcal{A}(q_0, w)$. Therefore, $\alpha \in L(\mathcal{A})$ iff $\alpha \in L(\mathcal{S})$.
\end{proof}

\vspace{5pt}

We have proven that the Schewe merger function can be used to refine congruence relations (such as $\equiv_\text{\Ankh}$) that, by themselves are not strong enough criteria to allow for a merging of states, to the point that they can be used for state space reduction. A final result regarding this technique is adapted from \cite{Schewe2010} and shows a relation between this algorithm and priority almost equivalence.

\begin{lem}
	Let $\sim \,=\, \equiv_\text{\Ankh}$ and let $\mathcal{S}'$ be a representative merge of $\mathcal{S}$ w.r.t. $\mu_M$. There is no smaller DPA than $\mathcal{S}'$ that is priority almost equivalent to $\mathcal{A}$.
\end{lem} 

\begin{proof}
	Let $\mathcal{B}$ be a DPA that is smaller than $\mathcal{S}'$. Our goal is to show that $\mathcal{A} \not\equiv_\text{\Ankh} \mathcal{B}$.
	
	At first observe that $\mathcal{A} \equiv_\text{\Ankh} \mathcal{S}'$: $\mathcal{A}$ and $\mathcal{S}$ are priority almost equivalent as for every state in $\mathcal{A}$, there is an equivalent representative in $\mathcal{S}$. $\mathcal{S}$ and $\mathcal{S}'$ are Moore equivalent by Lemma \ref{lem:general:moore_merge_keeps_mooreequiv} and thus they are also priority almost equivalent by Lemma \ref{lem:general:M_subs_Ankh_subs_L}.

	Assume that $\mathcal{S}' \equiv_\text{\Ankh} \mathcal{B}$ holds, so for all states in $\mathcal{S}'$, there is a a priority almost equivalent state in $\mathcal{B}$. We define a function $f$ that maps to each equivalence class of $\equiv_\text{\Ankh}$ in $\mathcal{S}'$ all states in $\mathcal{B}$ that are equivalent to it, i.e. if $\kappa \subseteq Q_{\mathcal{S}'}$ is an equivalence class, then $f(\kappa) = \{q \in Q_\mathcal{B} \mid \exists p \in \kappa: (\mathcal{S}', p) \equiv_\text{\Ankh} (\mathcal{B}, q) \}$. Note that $f(\kappa)$ can never be empty by the assumption of $\mathcal{S}' \equiv_\text{\Ankh} \mathcal{B}$.
	
	As $\mathcal{B}$ is smaller than $\mathcal{S}'$, the pigeonhole principle applies and we can fix $\kappa$ to be one equivalence class such that $|f(\kappa)| < |\kappa|$.
	
	By Lemma \ref{lem:schewe:equiv_same_scc}, there is an SCC $C$ in $\mathcal{S}'$ that contains all states in $\kappa$. Without loss of generality we can assume that likewise there is an SCC $D$ in $\mathcal{B}$ that contains $f(\kappa)$. If no such SCC would exist, we could simply apply the Schewe merger to $\mathcal{B}$ to find an automaton that is smaller than $\mathcal{S}'$ and does have this property.
	
	$C$ and $D$ must be non-trivial SCCs: if $C$ would be trivial, $\kappa$ would contain only one element and $f(\kappa)$ would be empty. If $D$ would be trivial, $f(\kappa) = \{q\}$ would contain of only one state. Since $\mathcal{S}' \equiv_\text{\Ankh} \mathcal{B}$, there is a state $p \in \kappa \subseteq C$ in $\mathcal{S}'$ with $(\mathcal{S}', p) \equiv_\text{\Ankh} (\mathcal{B}, q)$. Since $C$ is not trivial, there is a word $w$ from which $\mathcal{S}'$ moves from $p$ back to $p$. $\mathcal{B}$ however, leaves $f(\kappa)$ with that word, as $D$ is trivial, so $q$ cannot reach itself again. This is a contradiction, as $\equiv_\text{\Ankh}$ is a congruence relation.
	
	\vspace{5pt}
	
	We claim that there is a state $p \in \kappa$ such that there is a family of words $(w_q)_{q \in Q_{\mathcal{B}}}$ such that $\mathcal{S}'$ does not leave $C$ reading these words from $p$ and $c_{\mathcal{S}'}(\delta_{\mathcal{S}'}^*(p, w)) \neq c_{\mathcal{B}}(\delta_{\mathcal{B}}^*(q, w))$. (in other words, $w_q$ is a witness for $p$ and $q$ being not Moore-equivalent.)
	
	Towards a contradiction, assume that the claim is false and that for every $p \in \kappa$, there is a state $q_p \in f(\kappa)$ that does not satisfy the property. As $|\kappa| > |f(\kappa)|$ we can again use the pigeonhole principle and obtain two states $p_1, p_2 \in \kappa$ such that $q_{p_1} = q_{p_2}$. We call this state $q := q_{p_1}$.
	
	For each word $w \in \Sigma^*$, $c_{\mathcal{S}'}^*(p_1, w) = c_{\mathcal{B}}^*(q, w)$ or $\mathcal{S}'$ leaves $C$ while reading $w$ from $p_1$. The same holds for $p_2$. If for all $w$ the first case would apply, then $c_{\mathcal{S}'}^*(p_1, w) = c_{\mathcal{S}'}^*(p_2, w)$ for all $w$ and thus $p_1 \equiv_M p_2$. This is impossible, as the two states would have been merged in the construction of $\mathcal{S}'$. Without loss of generality, we assume that $p_1$ breaks the pattern and there is a $w$ such that $c_{\mathcal{S}'}^*(p_1, w) \neq c_{\mathcal{B}}^*(q, w)$ but $\mathcal{S}'$ leaves $C$ when reading $w$ from $p_1$. Let this $w$ have minimal length.
	
	Let $\rho_1$ and $\rho_2$ be the respective runs of $\mathcal{S}$ (note that we are using the automaton here that is not yet Moore-minimized) on $w$ from $p_1$ and $p_2$. At some point $k$, $\rho_1$ leaves $C$.
	%TODO
	
	\vspace{5pt}
	
	We can use the claim that he have just shown to finish our overall proof. Fix an arbitrary $q_0 \in f(\kappa)$. We define a sequence of finite words $(\alpha_n)_{n \in \mathbb{N}}$ such that every $\alpha_n$ is a prefix of $\alpha_{n+1}$ and the runs of $\mathcal{S}'$ and $\mathcal{B}$ from $p$ and $q_0$ respectively differ in priority at least $n$ times. Then $\alpha := \bigcup_n \alpha_n$ is an $\omega$-word that is a witness for $(\mathcal{S}', p)$ and $(\mathcal{B}, q_0)$ not being priority almost equivalent.
	
	We make sure that after reading any $\alpha_n$ from $p$, $\mathcal{S}'$ moves back to $p$. Let $\alpha_0 := \varepsilon$ and assume for induction that $\alpha_n$ has already been defined. After reading $\alpha_0$, $\mathcal{S}'$ reaches $p$ and $\mathcal{B}$ reaches some $q$. If $q \notin f(\kappa)$, i.e. $(\mathcal{S}', p) \not\equiv_\text{\Ankh} (\mathcal{B}, q)$, there is a witness $\beta$ and we can simply set $\alpha := \alpha_n \beta$. Otherwise, $q \in f(\kappa)$. With the claim we have proven, we can find a word $w_q$ such that reading the word from $p$ and $q$ leads to states $p'$ and $q'$ that have different priorities but $p'$ is still in $C$. Thus, there is a word $u$ that leads back from $p'$ to $p$. We then set $\alpha_{n+1} := \alpha_n w_q u$. This finishes our proof.
\end{proof}





