\section{Labeled SCC Filter}
\label{sect:lsf}

\begin{defn}
	Let $\mathcal{A} = (Q_1, \Sigma, \delta_1, c_1)$ and $\mathcal{B} = (Q_2, \Sigma, \delta_2, c_2)$ be DPAs, $k \in \mathbb{N}$, and $\sim$ be an equivalence relation that implies language equivalence. We define a relation $\equiv_\text{LSF}^{k,\sim}$ such that $(\mathcal{A}, p) \equiv_\text{LSF}^{k,\sim} (\mathcal{B}, q)$ holds if and only if $(\mathcal{A}, p) \equiv_M^{\leq k} (\mathcal{B}, q)$ and $(\mathcal{A}, p) \sim (\mathcal{B}, q)$.
\end{defn}

\begin{lem}
	$\equiv_\text{LSF}^{k,\sim}$ is an equivalence relation.
\end{lem}

\begin{proof}
	$\equiv_\text{LSF}^{k,\sim}$ is the intersection of two equivalence relations and therefore one itself.
\end{proof}

\begin{lem}
	If $l \leq k$ and $\sim \,\subseteq\, \approx$, then $\equiv_\text{LSF}^{k,\sim} \,\subseteq\, \equiv_\text{LSF}^{l,\approx}$.
\end{lem}

\begin{proof}
	%TODO
\end{proof}

\begin{defn}
	Let $\mathcal{A} = (Q, \Sigma, \delta, c)$ be a DPA. We define $\mathcal{A}\upharpoonright^c_{> k} := \mathcal{A}\upharpoonright_P$ with $P = \{q \in Q \mid c(q) > k\}$ and set $\preceq_k$ to be a total extension of the reachability preorder in $\mathcal{A}\upharpoonright^c_{> k}$.
\end{defn}

\begin{defn}
	Let $\kappa \in \mathfrak{C}(\equiv_\text{LSF}^{k,\sim}, \mathcal{A})$ be an equivalence class. We define $C_\kappa^k = \{ r \in \kappa \mid c(r) > k \text{ and } r \text{ is } \preceq_k \text{-maximal among } \kappa \}$ and $M^k_\kappa = \kappa \setminus C_\kappa$.
\end{defn}


\vspace{10pt}

\begin{lem}
	Let $\mathcal{A}$ be a DPA and $\kappa \in \mathfrak{C}(\equiv_\text{LSF}^{k,\sim})$. Let $\mathcal{A}'$ be a representative merge of $\mathcal{A}$ w.r.t. $M_\kappa$ by candidates $C_\kappa$. Then $(\mathcal{A}, p) \equiv_L (\mathcal{A}', p)$ for all $p$.
	\label{lem:lsf:preserve_language}
\end{lem}

\begin{proof}
	Let $r$ be the representative that is used in the merge. Let $\rho$ and $\rho'$ be the runs of $\mathcal{A}$ and $\mathcal{A}'$ on some $\alpha$ starting in $p$. We claim that $\rho$ is accepting iff $\rho'$ is accepting.
	
	By Lemma \ref{lem:general:cong_stays_in_merge}, we know that $(\mathcal{A}, \rho(i)) \sim (\mathcal{A}, \rho'(i))$ and $(\mathcal{A}, \rho(i)) \equiv_M^{\leq k} (\mathcal{A}, \rho'(i))$ for all $i$. If there is a position $n$ from which on $\rho'[n,\omega]$ is both a valid run in $\mathcal{A}$ and $\mathcal{A}'$, then we know that $\rho$ is accepting if and only if $\rho'$ is accepting since $(\mathcal{A}, \rho(n)) \sim (\mathcal{A}, \rho'(n))$ and therefore $(\mathcal{A}, \rho(n)) \equiv_L (\mathcal{A}, \rho'(n))$.
	
	If $\rho'$ visits infinitely many states with priority equal to or less than $k$, then $\rho$ and $\rho'$ share the same minimal priority that is visited infinitely often ($(\mathcal{A}, \rho(i)) \equiv_M^{\leq k} (\mathcal{A}, \rho'(i))$) and thus have the same acceptance.
	
	For the last case, assume that $\rho'$ uses infinitely many redirected edges but from some point $n_1$ on stays in $\mathcal{A}\upharpoonright^c_{>k}$. Let $n_3 > n_2 > n_1$ be the next two positions at which $\rho'$ uses a redirected edge, i.e. $\delta(\rho'(n_2), \alpha(n_2)) \neq \delta'(\rho'(n_2), \alpha(n_2))$ and analogous for $n_3$. Note that $\delta'(\rho'(n_2), \alpha(n_2)) = \delta'(\rho'(n_3), \alpha(n_3)) = r$, since all redirected transition target the representative state. We call $\delta(\rho'(n_3), \alpha(n_3)) = q$. Since between $n_2$ and $n_3$ no redirected transition is taken, $\rho'[n_2 + 1, n_3 + 1]$ is a valid path in $\mathcal{A}$, so we have $r \preceq_k q$ by choice of $n_1$. The fact that transitions to $q$ are redirected to $r$ however requires that $q \in M_\kappa$ and therefore $q$ being not $\preceq_k$-maximal. Thus, there is a $q'$ with $q \prec_k q'$ and therefore $r \prec_k q'$ which contradicts the choice of $r$ from the set of candidates.
\end{proof}

\begin{lem}
	Let $\mathcal{A}$ be a DPA and $\kappa \in \mathfrak{C}(\equiv_\text{LSF}^{k,\sim})$. Let $\mathcal{A}'$ be a representative merge of $\mathcal{A}$ w.r.t. $M_\kappa$ by candidates $C_\kappa$. Then $(\mathcal{A}, p) \equiv_M^{\leq k} (\mathcal{A}', p)$ for all $p$.
	\label{lem:lsf:preserve_kmoore}
\end{lem}

\begin{proof}
	Let $\rho$ and $\rho'$ be the runs of $\mathcal{A}$ and $\mathcal{A}'$ on $\alpha \in \Sigma^\omega$ starting from $q$. We claim that $c(\rho(i)) =^{\leq k} c'(\rho'(i))$ for all $i$ which then proves the Lemma.
	
	By Lemma \ref{lem:general:cong_stays_in_merge}, $(\mathcal{A}, \rho(i)) \equiv_M^{\leq k} (\mathcal{A}, \rho'(i))$ for all $i$ which especially means that for all $i$, $c(\rho(i)) =^{\leq k} c(\rho'(i))$. Since $c(\rho'(i)) = c'(\rho'(i))$, that also implies $c(\rho(i)) =^{\leq k} c'(\rho'(i))$.
\end{proof}

\vspace{10pt}

While these are already good results, to go further requires us to restrict $\sim$ to $\equiv_L$.

\begin{lem} 
	Let $\mathcal{A}$ be a DPA and $\kappa \in \mathfrak{C}(\equiv_\text{LSF}^{k,\sim})$. Let $\mathcal{A}'$ be a representative merge of $\mathcal{A}$ w.r.t. $M_\kappa$ by candidates $C_\kappa$. Then $(\mathcal{A}, p) \equiv_\text{LSF}^{k,\equiv_L} (\mathcal{A}', p)$ for all $p$.
	\label{lem:lsf:preserve_lsf}
\end{lem}

\begin{proof}
	Representative merges never change priorities assigned to states. Together with Lemma \ref{lem:lsf:preserve_language} and Lemma \ref{lem:lsf:preserve_kmoore} this already finishes the proof.
\end{proof}


\vspace{10pt}

We can now define a merger function using the LSF method and prove its correctness.

\begin{defn}
	We define the \emph{LSF merger function}: $$\mu_\text{LSF}^{k,\sim} : \{ M^k_\kappa \mid \kappa \in \mathfrak{C}(\equiv_\text{LSF}^{k,\sim}, \mathcal{A}) \} \rightarrow 2^Q , M^k_\kappa \mapsto C^k_\kappa $$
\end{defn}

\begin{theorem}
	Let $\mathcal{A}'$ be a representative merge of a DPA $\mathcal{A}$ w.r.t. $\equiv_\text{LSF}^{k,\equiv_L}$. Then $(\mathcal{A}, p) \equiv_\text{LSF}^{k,\equiv_L} (\mathcal{A}', p)$ for all $p$.
\end{theorem}

\begin{proof}
		Building a representative merge of $\mathcal{A}$ w.r.t. $\mu_\text{LSF}^{k,\equiv_L}$ implicitly builds multiple representative merges $\mathcal{A}_0, \dots, \mathcal{A}_m$, where $\mathcal{A}_{i+1}$ is a representative merge of $\mathcal{A}_i$ w.r.t. some $M^k_\kappa$ by candidates $C^k_\kappa$. By Lemma \ref{lem:lsf:preserve_lsf}, for all $j > i$, the states in $\kappa_j$ are still pairwise $\mu_\text{LSF}^{k,\equiv_L}$ equivalent. Moreover, $\kappa_{i+1}$ is an $\mu_\text{LSF}^{k,\equiv_L}$ equivalence class in $\mathcal{A}_i$. That means we can continue building representative merges in order of the enumeration and our previous results apply. In the end, we obtain a DPA $\mathcal{A}'$ that, by Lemma \ref{lem:lsf:preserve_lsf}, satisfies $(\mathcal{A}, p) \equiv_\text{LSF}^{k,\equiv_L} (\mathcal{A}', p)$.
\end{proof}

\begin{cor}
	Every representative merge of a DPA $\mathcal{A}$ w.r.t. $\mu_\text{LSF}^{k,\equiv_L}$ is language equivalent to $\mathcal{A}$.
\end{cor}

\vspace{10pt}

A final result to show how versatile the LSF merge is is captured by the following Theorem.

\begin{theorem}
	$\mu_\text{LSF}^{-1,\sim} \,=\, \mu_\text{skip}^\sim$
\end{theorem}

\begin{proof}
	%TODO
\end{proof}
















