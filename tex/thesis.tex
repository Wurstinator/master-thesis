\documentclass[oneside]{book}

% PACKAGES
\usepackage[german,english]{babel}
\usepackage[utf8]{inputenc}
\usepackage{amsmath}
\usepackage{amsfonts}
\usepackage{amssymb}
\usepackage{amsthm}
\usepackage{marvosym}
\usepackage{tikz}
\usetikzlibrary{automata,arrows,positioning}
\usepackage{hyperref}
\usepackage{fancyhdr}
\usepackage[nottoc]{tocbibind}
\usepackage{environ}
\usepackage{algorithm}
\usepackage{algpseudocode}
\usepackage{pdfpages}
\usepackage{subcaption}
\usepackage[toc,page]{appendix}

\usepackage{geometry}

% COMMANDS
\newcommand{\enquote}[1]{``#1''}

% Quotient classes
\newcommand{\bigslant}[2]{{\raisebox{.2em}{$#1$}\left/\raisebox{-.2em}{$#2$}\right.}}
%\bigslant{A}{B}


% THEOREM STYLES
\newtheoremstyle{slplain}% name
  {.6\baselineskip\@plus.2\baselineskip\@minus.2\baselineskip}% Space above
  {0}% Space below
  {\slshape}% Body font
  {}%Indent amount (empty = no indent, \parindent = para indent)
  {\bfseries}%  Thm head font
  {.}%       Punctuation after thm head
  { }%      Space after thm head: " " = normal interword space;
        %       \newline = linebreak
  {}%       Thm head spec

\theoremstyle{slplain}
\newtheorem{theorem}{Theorem}[section]
\newtheorem{lem}[theorem]{Lemma}
\newtheorem{appendixlem}{Lemma}[chapter]
\newtheorem{prop}[theorem]{Proposition}
\newtheorem{cor}[theorem]{Corollary}

\theoremstyle{definition}
\newtheorem{defn}{Definition}[section]
\newtheorem{conj}{Conjecture}
\newtheorem{exmp}{Example}[chapter]

\theoremstyle{remark}
\newtheorem*{rem}{Remark}
\newtheorem*{note}{Note}
\newtheorem{case}{Case}

%FIX HYPERREF
\AtBeginDocument{\let\textlabel\label}

% REMOVE PROOFS
\NewEnviron{killcontents}{}
%\let\proof\killcontents
%\let\endproof\endkillcontents


% DISALLOW LINE BREAKS
\relpenalty=9999
\binoppenalty=9999


\begin{document}

\tableofcontents

\section*{Introduction}

Finite automata are a long established computation model that dates back to sources such as \cite{McCulloch1990} and \cite{RabinScott1959}. A known problem for finite automata is state space reduction, referring to the search of a language-equivalent automaton which uses fewer states than the original object. For deterministic finite automata (DFA), not just reduction but minimization was solved in \cite{Hopcroft1971}. Regarding nondeterministic finite automata (NFA), \cite{JianRavikumar1991} proved the PSPACE-completeness of the minimization problem, which is why reduction algorithms such as \cite{ChamparnaudCoulon2004} and \cite{BonchiPous2013} are a popular alternative.

In his prominent work \cite{Buchi1966}, B\"uchi introduced the model of B\"uchi automata (BA) as an extension of finite automata to read words of one-sided infinite length. As these $\omega$-automata tend to have higher levels of complexity in comparison to standard finite automata, the potential gain of state space reduction is even greater. Similar to NFAs, exact minimization for deterministic B\"uchi automata was shown to be NP-complete in \cite{Schewe2010} and spawned heuristic approaches such as \cite{Schewe2010}, \cite{MayrClemente2012}, or \cite{EtessamiWilkeSchuller2001}. 

As \cite{Thomas1991} displays, deterministic B\"uchi automata are a strictly weaker model than nondeterministic Büchi automata. It is therefore interesting to consider different models of $\omega$-automata in which determinism is possible while maintaining enough power to describe all $\omega$-regular languages. Parity automata (PA) are one such model, a mixture of B\"uchi automata and Moore automata (\cite{Moore56}), that use a parity function rather than the usual acceptance set. \cite{Piterman2007} showed that deterministic parity automata are in fact sufficient to recognize all $\omega$-regular languages. As for DBAs, the exact minimization problem for DPAs is NP-complete (\cite{Schewe2010}).

Our goal in this publication is to develop new algorithms for state space reduction of DPAs, partially adapted from existing algorithms for B\"uchi or Moore automata. We perform theoretical analysis of the algorithms in the form of proofs of correctness and analysis of run time complexity, as well as practical implementation of the algorithms in code to provide empirical data for or against their actual efficiency.

%TODO


\chapter{Basics}
In the first chapter, we define basics notations and conventions and present some data of our empirical testing environment used to collect practical data of each approach.
In the second chapter, we establish some theoretical ground work that will be used by the rest of the thesis.
After that, we present several algorithms for state space reduction, split into one chapter for each such procedure. Each such chapter is made up of at least three sections. The first section describes the idea and definition of the algorithm and proofs well behaving properties. The second section covers how to actually compute the reduction and provides a short run time analysis. The third section shows practical data to analyze \enquote{real world} usefulness. Potential other sections contain variants or extensions of the original procedure as well as potential open questions.

\chapter{Basic Definitions}

The first chapter defines fundamentals of this thesis and notation used later on.

\section{General Mathematical Terms}
As our main focus is $\omega$-words, we will require a small extension of natural numbers into the transfinite realm.

\subsection{Sets and Functions}
\begin{defn}
	The \emph{natural numbers} $\mathbb{N} = \{0, 1, 2, \dots\}$ are the set of all non-negative integers. We define $0 := \emptyset$, $1 := \{0\}$, $2 := \{0, 1\}$, and so forth.
	
	The value $\omega$ denotes the \enquote{smallest} infinity, $\omega := \mathbb{N}$. For all natural numbers, we write $n < \omega$ and $\omega \not < \omega$. Also, we sometimes use the convention $n + \omega = \omega$.
	
	We denote the set $\mathbb{N} \cup \{\omega\}$ by $\mathbb{N}_\omega$.
\end{defn}

\begin{defn}
	Let $X$ and $Y$ be two sets. We use the usual definition of union ($\cup$), intersection ($\cap$), and set difference ($\setminus$). If some domain ($X \subseteq D$) is clear in the context, we write $X^\complement = D \setminus X$.
	
	We use the cartesian product $X \times Y = \{ (x, y) \mid x \in X, y \in Y \}$.
	
	We write $X^Y$ for the set of all functions with domain $Y$ and range $X$. If we have a function $f : D \rightarrow \{0, 1\}$, then we sometimes implicitly use it as a set $X \subseteq D$ with $x \in X$ iff $f(x) = 1$. In particular, $2^Y$ is the powerset of $Y$.
\end{defn}

\begin{defn}
	Let $f : D \rightarrow R$ be a function and let $X \subseteq D$ and $Y \subseteq R$. We describe by $f(X) = \{ f(x) \in R \mid x \in X\}$ and $f^{-1}(Y) = \{ x \in D \mid \exists y \in Y: f(x) = y \}$.
\end{defn}

\begin{defn}
	Let $X \subseteq D$ be a set. For $D' \subseteq D$, we define $X \upharpoonright_{D'} = X \cap D$. In particular, we use this notation for relations, e.g. $R \subseteq \mathbb{N} \times \mathbb{N}$ and $R \upharpoonright_{\{0\} \times \mathbb{N}}$.
	
	For a function $f : D \rightarrow R$, we write $f \upharpoonright_{D'}$ for the function $f' : D' \rightarrow R, x \mapsto f(x)$.
\end{defn}


\subsection{Relations and Orders}
\begin{defn}
	Let $X$ be a set. We call a set $R \subseteq X \times X$ a \emph{relation} over $X$. $R$ is
	\begin{itemize}
		\item \emph{reflexive}, if for all $x \in X$, $(x, x) \in R$.
		\item \emph{irreflexive}, if for all $x \in X$, $(x, x) \notin R$.
		\item \emph{symmetric}, if for all $(x, y) \in R$, also $(y, x) \in R$.
		\item \emph{asymmetric}, if for all $(x, y) \in R$, $(y, x) \notin R$.
		\item \emph{transitive}, if for all $(x, y), (y, z) \in R$, also $(x, z) \in R$.
		\item \emph{total}, if for all $x, y \in X$, $(x, y) \in R$ or $(y, x) \in R$ is true.
	\end{itemize}
	
	We call $R$ 
	\begin{itemize}
		\item a \emph{partial order}, if it is irreflexive, asymmetric, and transitive.
		\item a \emph{total order}, if it is a partial order and total.
		\item a \emph{preorder}, if it is reflexive and transitive.
		\item a \emph{total preorder}, if it is a preorder and total.
		\item an \emph{equivalence relation}, if it is a preorder and symmetric.
	\end{itemize}
	
	If $R$ is a partial order or a preorder, we call an element $x \in X$ \emph{minimal} (w.r.t. $R$), if for all $y \in X$, $(y, x) \in R$ implies $(x, y) \in R$. Similarly, we call it \emph{maximal}, if for all $y \in X$, $(x, y) \in R$ implies $(y, x) \in R$. 
	
	We call $x$ the \emph{minimum} of $R$ if for all $y \neq x$, $(y, x) \in R$. We write $x = \min_R X$.
\end{defn}

\begin{defn}
	Let $R$ be a partial order over $X$. We call a set $S \subseteq Y$ an \emph{extension of $R$ to $Y$} if $X \subseteq Y$, $R \subseteq S$, and $S$ is a partial order over $Y$. We use the same notation for total orders, preorders, and total preorders.
\end{defn}

\begin{defn}
	Let $R$ be an equivalence relation over $X$. $R$ implicitly forms a partition of $X$ into \emph{equivalence classes}. For an element $x \in X$, we call $[x]_R := \{ y \in X \mid (x, y) \in R \}$ the equivalence class of $x$. We denote the set of equivalence classes by $\mathfrak{C}(R) = \{ [x]_R \mid x \in R \}$.
\end{defn}





\section{Words and Languages}
\begin{defn}
	A non-empty set of symbols can be called an \emph{alphabet}, which we will denote by a variable $\Sigma$ most of the time. As symbols, we usually use lower case letters, i.e. $a$ or $b$.
	
	A \emph{finite word}, usually denoted by $u$, $v$, or $w$, over an alphabet $\Sigma$ is a function $w : n \rightarrow \Sigma$ for some $n$. We call $n$ the \emph{length} of $w$ and write $|w| = n$. The unique word of length $0$ is called \emph{empty word} and is written as $\varepsilon$.
	
	Given $\Sigma^n = \{ w \mid w \text{ is a word of length } n \text{ over } \Sigma \}$, we define $\Sigma^* = \bigcup\limits_{n \in \mathbb{N}} \Sigma^n$ as the set of all finite words over $\Sigma$. 
\end{defn}

\begin{defn}
	An \emph{$\omega$-word}, usually denoted by $\alpha$ or $\beta$, over an alphabet $\Sigma$ is a function $\alpha : \omega \rightarrow \Sigma$. $\omega$ is the length of $\alpha$ and we write $|\alpha| = \omega$. The set $\Sigma^\omega$ then describes the set of all $\omega$-words over $\Sigma$. 
\end{defn}

\begin{defn}
	A \emph{language} over an alphabet $\Sigma$ is a set of words $L \subseteq \Sigma^* \cup \Sigma^\omega$. In the context we use it should always be clear whether we are using finite words or $\omega$-words.
\end{defn}

\begin{defn}
	Let $v, w \in \Sigma^*$ and $w_i \in \Sigma^*$ for all $i \in \mathbb{N}$ be words over $\Sigma$ and $\alpha \in \Sigma^\omega$ be an $\omega$-word over $\Sigma$.
	
	The \emph{concatenation} of $v$ and $w$ (denoted by $v \cdot w$) is a word $u$ such that:
	\begin{align*}
	u : |v|+|w| \allowbreak \rightarrow \Sigma, i \mapsto 
	\begin{cases}
		v(i) & \text{if } i < |v| \\
		w(i-|v|) & \text{else}
	\end{cases}
	\end{align*}

	The \emph{concatenation} of $w$ and $\alpha$ (denoted by $w \cdot \alpha$) is an $\omega$-word $\beta$ such that:
	\begin{align*}
	\beta : \mathbb{N} \rightarrow \Sigma, i \mapsto 
	\begin{cases}
		w(i) & \text{if } i < |w| \\
		\alpha(i-|w|) & \text{else}
	\end{cases}
	\end{align*}
	
	For some $n \in \mathbb{N}$, the \emph{$n$-iteration} of $w$ (denoted by $w^n$) is a word $u$ such that:
	\begin{align*}
		u : |w|^n \rightarrow \Sigma, i \mapsto w(i \mod |w|)
	\end{align*}
	
	The \emph{$\omega$-iteration} of $w$ (denoted by $w^\omega$) is an $\omega$-word $\alpha$ such that:
	\begin{align*}
		\beta : \mathbb{N} \rightarrow \Sigma, i \mapsto w(i \mod |w|)
	\end{align*}
\end{defn}

\vspace{20pt}
For the purpose of easier notation and readability, we write singular symbols as words, i.e. for an $a \in \Sigma$ we write $a$ for the word $w_a : \{0\} \rightarrow \Sigma, i \mapsto a$.\\
We also abbreviate $v \cdot w$ to $vw$ and $w \cdot \alpha$ to $w \alpha$. Further, we use $\alpha \cdot \varepsilon = \alpha$ for $\alpha \in \Sigma^\omega$.
\vspace{10pt}

\begin{defn}
	Let $L, K \subseteq \Sigma^*$ be a language and $U \subseteq \Sigma^\omega$ be an $\omega$-language.\\
	The \emph{concatenation} of $L$ and $K$ is $L \cdot K = \{ u \in \Sigma^* \mid \text{There are } v \in L \text{ and } \allowbreak w \in K \text{ such that } u = v \cdot w\}$.\\
	The \emph{concatenation} of $L$ and $U$ is $L \cdot U = \{ \alpha \in \Sigma^\omega \mid \text{There are } w \in L \text{ and } \allowbreak \beta \in U \text{ such that } \alpha = w \cdot \beta\}$.\\
	For some $n \in \mathbb{N}$, the \emph{$n$-iteration} of $L$ is $L^n = \{ w \in \Sigma^* \mid \text{There is } v \in L \text{ such that } w = v^n\}$.\\
	The \emph{Kleene closure} of $L$ is $L^* = \bigcup\limits_{n \in \mathbb{N}} L^n$.\\
\end{defn}

\begin{defn}
	Let $w \in \Sigma^* \cup \Sigma^\omega$ be a word. We define a substring or subword of $w$ for some $n \leq m \leq |w|$ as $w[n, m] = w(n) \cdot w(n+1) \cdots w(m-1)$. In the case that $m = |w| = \omega$, it is simply $w[n, m] = w(n) \cdot w(n+1) \cdots$. Note that for $n = m$, we have $w[n, m] = \varepsilon$.
\end{defn}

\begin{defn}
	Let $v, w \in \Sigma^* \cup \Sigma^\omega$ be words. We call $v$ 
	\begin{itemize}
		\item a \emph{prefix} of $w$, if there is an $n \in \mathbb{N}_\omega$ with $v = w[0, n]$.
		\item a \emph{suffix} of $w$, if there is an $n \in \mathbb{N}_\omega$ with $v = w[n, |w|]$.
		\item an \emph{infix} of $w$, if there are $n, m \in \mathbb{N}_\omega$ with $v = w[n, m]$.
	\end{itemize}
\end{defn}

\begin{defn}
	The \emph{occurrence set} of a word $w \in \Sigma^* \cup \Sigma^\omega$ is the set of symbols which occur at least once in $w$. 
	$$\text{Occ}(w) = \{ a \in \Sigma \mid \text{There is an } n \in |w| \text{ such that } w(n) = a \text{.}\}$$
	
	The \emph{infinity set} of a word $w \in \Sigma^\omega$ is the set of symbols which occur infinitely often in $w$.
	$$\text{Inf}(w) = \{ a \in \Sigma \mid \text{For every } n \in \mathbb{N} \text{ there is a } m > n \text{ such that } w(m) = a \text{.}\}$$
\end{defn}


\section{Automata}

\begin{defn}
	Let $Q$ be a set, $\Sigma$ an alphabet, and $\delta : Q \times \Sigma \rightarrow Q$ a function. We call $\mathcal{S} = (Q, \Sigma, \delta)$ a \emph{deterministic transition structure}. We call $Q$ the states or state space.
	
	For $q \in Q$ and a word $w \in \Sigma^* \cup \Sigma^\omega$, we call $\rho \in Q^{1+|w|}$ the \emph{run} of $\mathcal{S}$ on $w$ starting in $q$ if $\rho(0) = q$ and for all $i$, $\rho(i+1) = \delta(\rho(i), w(i))$.
\end{defn}

\begin{defn}
	Let $\mathcal{S} = (Q, \Sigma, \delta)$ be a deterministic transition structure. For a set $\Omega \subseteq Q^* \cup Q^\omega$, we say that $\mathcal{S}$ has acceptance condition $\Omega$.
	
	We say that a run $\rho$ of $\mathcal{A}$ on some $w \in \Sigma^*$ is \emph{accepting}, if $\rho \in \Omega$; otherwise, the run is \emph{rejecting}. In either case, we say that $\mathcal{A}$ accepts or rejects $w$. 
	
	The \emph{language} of $\mathcal{A}$ with $\Omega$ from $q \in Q$ is the set of all words and $\omega$-words that are accepted by $\mathcal{A}$ from $q$.
\end{defn}

\begin{defn}
	Let $\mathcal{S} = (Q, \Sigma, \delta)$ be a deterministic transition structure. A \emph{strongly connected component}, or SCC, is a set $S \subseteq Q$ such that for all $p, q \in S$, there is a $w \in \Sigma^*$ with $\delta^*(p, w) = q$.
	
	An SCC $S$ is \emph{trivial} if it contains only one state $q$ and $\delta(q, a) \neq q$ for all $a \in \Sigma$.
\end{defn}

\begin{defn}
	A \emph{deterministic finite automaton} (or DFA) is a tuple $\mathcal{A} = (Q, \Sigma, \delta, F)$, where $F \subseteq Q$, such that $(Q, \Sigma, \delta)$ is a deterministic transition structure and has acceptance condition $\Omega = \{ \rho \in Q^* \mid \rho(|\rho|+1) \in F \}$. For the language of $(Q, \Sigma, \delta)$ with $\Omega$ from $q$, we write $L(\mathcal{A}, q)$.
\end{defn}

\begin{defn}
	A \emph{deterministic parity automaton} (or DPA) is a tuple $\mathcal{A} = (Q, \Sigma, \delta, c)$, where $c : Q \rightarrow \mathbb{N}$, such that $(Q, \Sigma, \delta)$ is a deterministic transition structure and has acceptance condition $\Omega = \{ \rho \in Q^* \mid \min \text{Inf}(c(\rho)) \text{ is even} \}$. For the language of $(Q, \Sigma, \delta)$ with $\Omega$ from $q$, we write $L(\mathcal{A}, q)$.
	
	We call the DPA a \emph{B\"uchi automaton} (or DBA) if $c(Q) \subseteq \{0, 1\}$. In that case, we use $F$ instead of $c$.
\end{defn}

\begin{defn}
	Let $\mathcal{S} = (Q, \Sigma, \delta)$ be a deterministic transition structure. We define $\delta^* : Q \times \Sigma^* \rightarrow Q$ as $\delta^*(q, \varepsilon) = q$ and $\delta^*(q, w \cdot a) = \delta(\delta^*(q, w), a)$.
\end{defn}

\begin{defn}
	Let $\mathcal{A} = (Q, \Sigma, \delta, c)$ be a DPA. We define $c^* : Q \times (\Sigma^* \cup \Sigma^\omega) \rightarrow (\mathbb{N}^* \cup \mathbb{N}^\omega)$ as $c^*(q, w) : 1+|w| \rightarrow \mathbb{N}, i \mapsto c(\delta^*(q, w[0, i]))$.
\end{defn}

\begin{defn}
	Let $\mathcal{S} = (Q, \Sigma, \delta)$ be a deterministic transition structure and $P \subseteq Q$. We define $\mathcal{S} \upharpoonright_P = (P, \Sigma, \delta')$ as a transition structure in which $\delta'$ might only be a partial function. We set $\delta'(p, a) = \delta(p, a)$ if $\delta(p, a) \in P$, or undefined otherwise.
\end{defn}


\vspace{1cm}




\section{Test Automata}
Several automata generated randomly using different parameters were used in the testing process. Three major different techniques of generation were used:

\begin{enumerate}
	\item Use Spot (\cite{}) to generate a random DPA. (called \emph{gendet})
	\item Use Spot to generate a random non-deterministic B\"uchi automaton and use Spot again to convert it to a DPA. (called \emph{detspot})
	\item Use Spot to generate a random non-deterministic B\"uchi automaton and use nbautils (\cite{}) to convert it to a DPA. (called \emph{detnbaut})
\end{enumerate}

Figures \ref{exp:fig:rawstats_size}, \ref{exp:fig:rawstats_prios}, \ref{exp:fig:rawstats_sccs}, and \ref{exp:fig:rawstats_langclas} present some information about the automata. Regarding the number of states we stopped the generation at about 100 states, as most algorithms become too slow at that point. For the detnbaut set, the path refinement procedure can be sped up which is why the upper limit is higher. We elaborate this point in the relevant section.

The number of priorities is rather small in general. For gendet, we intentionally limited the number to 4 to mimic real world behavior that can be found in the detnbautils set. In comparison to nbautils, Spot does not perform priority reduction on the determinization result which explains why the detspot automata use more priorities in general.

The number of SCCs is consistently small among all three sets. detspot and gendet contain a few examples of automata with up to 60 SCCs but these are extreme outliers. This low number of SCCs is important to consider, as multiple of the algorithms such as the skip merger perform better the less connected the automaton is.

Finally, the average size of equivalence classes $\mathfrak{C}(\equiv_L)$. Again, this is of relevance to some of the reduction algorithms such as LSF as only states which are language equivalent can be merged. We can observe that the gendet set almost entirely consists of trivial classes (i.e. classes of size 1) while detspot and detnbaut show more promise in that regard.


\begin{figure}
	\centering
	\begin{minipage}{0.49\textwidth}
		\includegraphics[page=1,height=.3\textheight]{../data/analysis/rawstats_gendet.pdf} 
		\includegraphics[page=1,height=.3\textheight]{../data/analysis/rawstats_detspot.pdf} 
		\includegraphics[page=1,height=.3\textheight]{../data/analysis/rawstats_detnbaut.pdf}
		\caption{Sizes of the automata in the testing environment.}
		\label{exp:fig:rawstats_size}
	\end{minipage}
	\hfill
	\begin{minipage}{0.49\textwidth}
		\includegraphics[page=2,height=.3\textheight]{../data/analysis/rawstats_gendet.pdf} 
		\includegraphics[page=2,height=.3\textheight]{../data/analysis/rawstats_detspot.pdf} 
		\includegraphics[page=2,height=.3\textheight]{../data/analysis/rawstats_detnbaut.pdf}
		\caption{Number of priorities in the automata in the testing environment.}
		\label{exp:fig:rawstats_prios}
	\end{minipage}
\end{figure}

\begin{figure}
	\centering
	\begin{minipage}{0.49\textwidth}
		\includegraphics[page=1,height=.3\textheight]{../data/analysis/rawstats_gendet.pdf} 
		\includegraphics[page=1,height=.3\textheight]{../data/analysis/rawstats_detspot.pdf} 
		\includegraphics[page=1,height=.3\textheight]{../data/analysis/rawstats_detnbaut.pdf}
		\caption{Number of SCCS in the automata in the testing environment.}
		\label{exp:fig:rawstats_sccs}
	\end{minipage}
	\hfill
	\begin{minipage}{0.49\textwidth}
		\includegraphics[page=2,height=.3\textheight]{../data/analysis/rawstats_gendet.pdf} 
		\includegraphics[page=2,height=.3\textheight]{../data/analysis/rawstats_detspot.pdf} 
		\includegraphics[page=2,height=.3\textheight]{../data/analysis/rawstats_detnbaut.pdf}
		\caption{Average size of $\equiv_L$-classes of the automata in the testing environment.}
		\label{exp:fig:rawstats_langclas}
	\end{minipage}
\end{figure}


 

\chapter{General Theory}
\section{General Results}
We first use this section to establish some general results that are used multiple times in the upcoming proofs.

\subsection{Equivalence Relations}

In general, we use the symbol $\equiv$ to denote equivalence relations, mostly between states of an automata. In general, we have automata $\mathcal{A}$ and $\mathcal{B}$ with states $p$ and $q$ from there respective state spaces. Our relations are then defined on $(\mathcal{A}, p) \equiv (\mathcal{B}, q)$.

\begin{defn}
	Assuming that $\mathcal{A}$ is a fixed automaton that is obvious in context and $p$ and $q$ are both states in $\mathcal{A}$, we shorten $(\mathcal{A}, p) \equiv (\mathcal{A}, q)$ to $p \equiv q$.
	
	Furthermore, we write $\mathcal{A} \equiv \mathcal{B}$ if for every $p$ in $\mathcal{A}$ there is a $q$ in $\mathcal{B}$ such that $(\mathcal{A}, p) \equiv (\mathcal{B}, q)$; and the same holds with $\mathcal{A}$ and $\mathcal{B}$ exchanged.
\end{defn}

\begin{defn}
	Let $\mathcal{A} = (Q_1, \Sigma, \delta_1)$ and $\mathcal{B} = (Q_2, \Sigma, \delta_2)$ be deterministic transition structures and let $\sim \,\subseteq (\{\mathcal{A}\} \times Q_1) \times (\{\mathcal{B}\} \times Q_2)$ be an equivalence relation. We call $R$ a \emph{congruence relation} if for all $(\mathcal{A}, p) \sim (\mathcal{B}, q)$ and all $a \in \Sigma$, also $(\mathcal{A}, \delta_1(p, a)) \sim (\mathcal{B}, \delta_2(q, a))$.
\end{defn}


\vspace{5pt}
The following is a comprehensive list of all relevant equivalence relations that we use.

\begin{itemize}
	\item Language equivalence, $\equiv_L$. Defined below.
	\item Moore equivalence, $\equiv_M$. Defined below.
	\item Priority almost equivalence, $\equiv_\text{\Ankh}$. Defined below.
	\item Delayed simulation equivalence, $\equiv_\text{de}$. Defined in %TODO
	\item Path refinement equivalence, $\equiv_\text{PR}$. Defined in %TODO
	\item Threshold Moore equivalence, $\equiv_\text{TM}$. Defined in %TODO
	\item Labeled SCC filter equivalence, $\equiv_\text{LSF}$. Defined in %TODO
\end{itemize}

Immediately we define the three first of these relations and show that they are computable.


\subsubsection{Language Equivalence}

\begin{defn}
	Let $\mathcal{A}$ and $\mathcal{B}$ be $\omega$-automata. We define \emph{language equivalence} as $(\mathcal{A}, p) \equiv_L (\mathcal{B}, q)$ if and only if for all words $\alpha \in \Sigma^\omega$, $\mathcal{A}$ accepts $\alpha$ from $p$ iff $\mathcal{B}$ accepts $\alpha$ from $q$.
\end{defn}

\begin{lem}
	$\equiv_L$ is a congruence relation.
	\label{lem:general:L_congruence}
\end{lem}

\begin{proof}
	It is obvious that $\equiv_L$ is an equivalence relation. For two states $(\mathcal{A}, p) \equiv_L (\mathcal{B}, q)$ and some successors $p' = \delta_1(p, a)$ and $q' = \delta_2(q, a)$, it must be true that $(\mathcal{A}, p') \equiv_L (\mathcal{B}, q')$. Otherwise there is a word $\alpha \in \Sigma^\omega$ that is accepted from $p'$ and rejected from $q'$ (or vice-versa). Then $a \cdot \alpha$ is rejected from $p$ and accepted from $q$ and thus $p \not\equiv_L q$.
\end{proof}

\begin{lem}
	Language equivalence of a given DPA can be computed in $\mathcal{O}(|Q|^2 \cdot |c(Q)|^2)$.
\end{lem}

\begin{proof}
	The algorithm is based partially on \cite{HenzingerTelle1996}.
	
	Let $\mathcal{A} = (Q, \Sigma, \delta, c)$ be the DPA that we want to compute $\equiv_L$ on. We construct a labeled deterministic transition structure $\mathcal{B} = (Q \times Q, \Sigma, \delta', d)$ with $\delta'((p_1, p_2), a) = (\delta(p_1, a), \delta(p_2, a))$ and $d((p_1, p_2)) = (c(p_1), c(p_2)) \in \mathbb{N}^2$. Then, for every $i, j \in c(Q)$, let $\mathcal{B}_{i,j} = \mathcal{B} \upharpoonright_{Q_{i,j}}$ with $Q_{i,j} = \{ (p_1, p_2) \in Q \times Q \mid c(p_1) \geq i, c(p_2) \geq j \}$, i.e. remove all states which have first priority less than $i$ or second priority less than $j$.
	
	For each $i$ and $j$, let $S_{i,j} \subseteq 2^{Q \times Q}$ be the set of all SCCs in $\mathcal{B}_{i,j}$ and let $S = \bigcup_{i,j} S_{i,j}$. From this set $S$, remove all SCCs $s \subseteq Q \times Q$ in which the parity of the smallest priority in the first component differs from the parity of the smallest priority in the second component. The \enquote{filtered} set we call $S'$. For any two states $p, q \in Q$, $p \not\equiv_L q$ iff there is a pair $(p', q') \in \bigcup S'$ that is reachable from $(p, q)$ in $\mathcal{B}$.
	
	We omit the correctness proof of the algorithm here. Regarding the runtime, observe that $\mathcal{B}$ has size $\mathcal{O}(|Q|^2)$ and we create $\mathcal{O}(|c(Q)|^2)$ copies of it. All other steps like computing the SCCs can then be done in linear time in the size of the automata, which brings the total to $\mathcal{O}(|Q|^2 \cdot |c(Q)|^2)$.
\end{proof}


\subsubsection{Priority Almost Equivalence}

\begin{defn}
	Let $\mathcal{A} = (Q_1, \Sigma, \delta_1, c_1)$ and $\mathcal{B} = (Q_2, \Sigma, \delta_2, c_2)$ be DPAs. We define \emph{priority almost equivalence} as $(\mathcal{A}, p) \equiv_\text{\Ankh} (\mathcal{B}, q)$ if and only if for all words $\alpha \in \Sigma^\omega$, $c_1^*(p, \alpha)$ and $c_2^*(q, \alpha)$ differ at only finitely many positions.
\end{defn}

\begin{lem}
	Priority almost equivalence is a congruence relation.
	\label{lem:general:Ankh_congruence}
\end{lem}

\begin{proof} 
	It is obvious that $\equiv_\text{\Ankh}$ is an equivalence relation. For two states $(\mathcal{A}, p) \equiv_\text{\Ankh} (\mathcal{B}, q)$ and some successors $p' = \delta(p, a)$ and $q' = \delta(q, a)$, it must be true that $(\mathcal{A}, p') \equiv_\text{\Ankh} (\mathcal{B}, q')$. Otherwise there is a word $\alpha \in \Sigma^\omega$ such that $c_1^*(p', \alpha)$ and $c_2^*(q', \alpha)$ differ at infinitely many positions. Then $c_1^*(p, a \alpha)$ and $c_2^*(q, a \alpha)$ also differ at infinitely many positions and thus $(\mathcal{A}, p) \not\equiv_\text{\Ankh} (\mathcal{B}, q)$.
\end{proof}

The following definition is used as an intermediate step on the way to computing $\equiv_\text{\Ankh}$.

\begin{defn}
	Let $\mathcal{A} = (Q_1, \Sigma, \delta_1, c_1)$ and $\mathcal{B} = (Q_2, \Sigma, \delta_2, c_2)$ be DPAs. We define the deterministic Büchi automaton $\mathcal{A} \intercal \mathcal{B} = (Q_1 \times Q_2, \Sigma, \delta_\intercal, F_\intercal)$ with $\delta_\intercal((q_1, q_2), a) = (\delta_1(q_1, a), \delta_2(q_2, a))$. The transition structure is a common product automaton.
	
	The final states are $F_\intercal = \{ (p, q) \in Q_1 \times Q_2 \mid c_1(p) \neq c_2(q) \}$, i.e. every pair of states at which the priorities differ. 
\end{defn}

\begin{lem}
	$\mathcal{A} \intercal \mathcal{B}$ can be computed in time $\mathcal{O}(|\mathcal{A}| \cdot |\mathcal{B}|)$.
	\label{lem:general:intercal_runtime}
\end{lem}

\begin{proof}
	The definition already provides a rather straightforward description of how to compute $\mathcal{A} \intercal \mathcal{B}$. Each state only requires constant time (assuming that $\delta$ and $c$ can be evaluated in such) and has $|\mathcal{A}| \cdot |\mathcal{B}|$ many states.
\end{proof}

\begin{lem}
	Let $\mathcal{A} = (Q_1, \Sigma, \delta_1, c_1)$ and $\mathcal{B} = (Q_2, \Sigma, \delta_2, c_2)$ be DPAs. $(\mathcal{A}, p) \equiv_\text{\Ankh} (\mathcal{B}, q)$ iff $L(\mathcal{A} \intercal \mathcal{B}, (p, q)) = \emptyset$. 
	\label{lem:general:intercal_prioalmostequiv}
\end{lem}

\begin{proof}
	For the first direction of implication, let $L(\mathcal{A} \intercal \mathcal{B}, (p_0, q_0)) \neq \emptyset$, so there is a word $\alpha$ accepted by that automaton. Let $(p, q) (p_1, q_1) (p_2, q_2) \cdots$ be the accepting run on $\alpha$. Then $p p_1 \cdots$ and $q q_1 \cdots$ are the runs of $\mathcal{A}$ and $\mathcal{B}$ on $\alpha$ respectively. Whenever $(p_i, q_i) \in F_\intercal$, $p_i$ and $q_i$ have different priorities. As the run of the product automaton vists infinitely many accepting states, $\alpha$ is a witness for $p$ and $q$ being not priority almost-equivalent.
	
	For the second direction, let $p$ and $q$ be not priority almost-equivalent, so there is a witness $\alpha$ at which infinitely many positions differ in priority. Analogously to the first direction, this means that the run of $\mathcal{A} \intercal \mathcal{B}$ on the same word is accepting and therefore the language is not empty.
\end{proof}

\begin{cor}
	Priority almost equivalence of a given DPA can be computed in quadratic time.
\end{cor}

\begin{proof}
	By Lemma \ref{lem:general:intercal_runtime}, we can compute $\mathcal{A} \intercal \mathcal{A}$ in quadratic time. The emptiness problem for deterministic B\"uchi automata is solvable in linear time by checking reachability of loops that contain a state in $F$. 
\end{proof}


\subsubsection{Moore Equivalence}

\begin{defn}
	Let $\mathcal{A} = (Q_1, \Sigma, \delta_1, c_1)$ and $\mathcal{B} = (Q_2, \Sigma, \delta_2, c_2)$ be DPAs. We define \emph{Moore equivalence} as $(\mathcal{A}, p) \equiv_M (\mathcal{B}, q)$ if and only if for all words $w \in \Sigma^*$, $c_1(\delta^*(p, w)) = c_2(\delta^*(q, w))$.
\end{defn}

\begin{lem}
	$\equiv_M$ is a congruence relation.
	\label{lem:general:M_congruence}
\end{lem}

\begin{proof}
	It is obvious that $\equiv_M$ is an equivalence relation. For two states $(\mathcal{A}, p) \equiv_M (\mathcal{B}, q)$ and some successors $p' = \delta(p, a)$ and $q' = \delta(q, a)$, it must be true that $(\mathcal{A}, p') \equiv_M (\mathcal{B}, q')$. Otherwise there is a word $w \in \Sigma^*$ such that $c_1(\delta_1^*(p', w)) \neq c_2(\delta_2^*(q', w))$. Then $c_1(\delta_1^*(p, aw)) \neq c_2(\delta_2^*(q, aw))$ and thus $(\mathcal{A}, p) \not\equiv_M (\mathcal{B}, q)$.
\end{proof}

\begin{lem}
	Moore equivalence of a given DPA can be computed in log-linear time.
\end{lem}

\begin{proof}
	We refer to \cite{Hopcroft1971}. The given algorithm can be adapted to Moore automata without changing the complexity.
\end{proof}


\vspace{10pt}

\begin{lem}
	$\equiv_M \,\subseteq\, \equiv_\text{\Ankh} \,\subseteq\, \equiv_L$
	\label{lem:general:M_subs_Ankh_subs_L}
\end{lem}

\begin{proof}
	Let $\mathcal{A} = (Q_\mathcal{A}, \Sigma, q_0^\mathcal{A}, \delta_\mathcal{A}, c_\mathcal{A})$ and $\mathcal{B} = (Q_\mathcal{B}, \Sigma, q_0^\mathcal{B}, \delta_\mathcal{B}, c_\mathcal{B})$ be two DPA that are priority almost-equivalent and assume towards a contradiction that they are not language equivalent. Due to symmetry we can assume that there is a $w \in L(\mathcal{A}) \setminus L(\mathcal{B})$. 
	
	Consider $\alpha = \lambda_\mathcal{A}(q_0^\mathcal{A}, w)$ and $\beta = \lambda_\mathcal{B}(q_0^\mathcal{B}, w)$, the priority outputs of the automata on $w$. By choice of $w$, we know that $a := \max \text{Inf}(\alpha)$ is even and $b := \max \text{Inf}(\beta)$ is odd. Without loss of generality, assume $a > b$. That means $a$ is seen only finitely often in $\beta$ but infinitely often in $a$. Hence, $\alpha$ and $\beta$ differ at infinitely many positions where $a$ occurs in $\alpha$. That would mean $w$ is a witness that the two automata are not priority almost-equivalent, contradicting our assumption.
	
	%TODO
\end{proof}






\subsection{Representative Merge}

\begin{defn}
	Let $\mathcal{A} = (Q, \Sigma, \delta, c)$ be a DPA and let $\emptyset \neq C \subseteq M \subseteq Q$. Let $\mathcal{A}' = (Q', \Sigma, \delta', c')$ be another DPA. We call $\mathcal{A}'$ a \emph{representative merge of $\mathcal{A}$ w.r.t. $M$ by candidates $C$} if it satisfies the following:
	\begin{itemize}
		\item There is a state $r_M \in C$ such that $Q' = (Q \setminus M) \cup \{r_M\}$.
		\item $c' = c\upharpoonright_{Q'}$.
		\item Let $p \in Q'$ and $\delta(p, a) = q$. If $q \in M $, then $\delta'(p, a) = r_M$. Otherwise, $\delta'(p, a) = q$. 
	\end{itemize}
	
	We call $r_M$ the \emph{representative} of $M$ in the merge. We might omit $C$ and implicitly assume $C = M$.
\end{defn}

\begin{defn}
	Let $\mathcal{A} = (Q, \Sigma, \delta, c)$ be a DPA and let $\mu : D \rightarrow (2^\mathcal{Q} \setminus \emptyset)$ be a function for some $D \subseteq 2^Q$. If all sets in $D$ are pairwise disjoint and for all $X \in D$, $\mu(X) \subseteq X$, we call $\mu$ a \emph{merger function}. 
	
	A DPA $\mathcal{A}'$ is a representative merge of $\mathcal{A}$ w.r.t. $\mu$ if there is an enumeration $X_1, \dots, X_{|D|}$ of $D$ and a sequence of automata $\mathcal{A}_0, \dots, \mathcal{A}_{|D|}$ such that $\mathcal{A}_0 = \mathcal{A}$, $\mathcal{A}_{|D|} = \mathcal{A}'$ and every $\mathcal{A}_{i+1}$ is a representative merge of $\mathcal{A}_i$ w.r.t. $X_{i+1}$ by candidates $\mu(X_{i+1})$.
\end{defn}

A common special case of this are quotient automata that are often used in state space reduction. Given a congruence relation $\sim$, the quotient automaton w.r.t. $\sim$ is equivalent to a representative merge w.r.t. $\mu : \mathfrak{C}(\sim) \rightarrow 2^Q, \kappa \mapsto \kappa$.

\begin{lem}
	Let $\mathcal{A} = (Q, \Sigma, \delta, c)$ be a DPA and let $\sim$ be an equivalence relation. A representative merge of $\mathcal{A}$ w.r.t. $\sim$ is a representative merge of $\mathcal{A}$ w.r.t. $\mu : \mathfrak{C}(\sim) \rightarrow 2^Q, \kappa \mapsto \kappa$.
\end{lem}

\vspace{5pt}

The following Lemma formally proofs that this definition actually makes sense, as building representative merges is commutative if the merge sets are disjoint.

\begin{lem}
	Let $\mathcal{A} = (Q, \Sigma, \delta, c)$ be a DPA and let $M_1, M_2 \subseteq Q$. Let $\mathcal{A}_1$ be a representative merge of $\mathcal{A}$ w.r.t. $M_1$ by some candidates $C_1$. Let $\mathcal{A}_{12}$ be a representative merge of $\mathcal{A}_1$ w.r.t. $M_2$ by some candidates $C_2$. If $M_1$ and $M_2$ are disjoint, then there is a representative merge $\mathcal{A}_2$ of $\mathcal{A}$ w.r.t. $M_2$ by candidates $C_2$ such that $\mathcal{A}_{12}$ is a representative merge of $\mathcal{A}_2$ w.r.t $M_1$ by candidates $C_1$.
\end{lem}

\begin{proof}
	By choosing the same representative $r_{M_1}$ and $r_{M_2}$ in the merges, this is a simple application of the definition.
\end{proof}

The following Lemma, while simple to prove, is interesting and will find use in multiple proofs of correctness later on.

\begin{lem}
	Let $\mathcal{A}$ be a DPA. Let $\sim$ be a congruence relation on $Q$ and let $M \subseteq Q$ such that for all $x, y \in M$, $x \sim y$. Let $\mathcal{A}'$ be a representative merge of $\mathcal{A}$ w.r.t. $M$ by candidates $C$. Let $\rho$ and $\rho'$ be runs of $\mathcal{A}$ and $\mathcal{A}'$ on some $\alpha$. Then for all $i$, $\rho(i) \equiv \rho'(i)$.
	\label{lem:general:cong_stays_in_merge}
\end{lem}

\begin{proof}
	We use a proof by induction. For $i = 0$, we have $\rho(0) = q_0$ for some $q_0 \in Q$ and $\rho'(0) = r_{[q_0]_M}$. By choice of the representative, $q_0 \in M$ and $r_{[q_0]_M} \in M$ and thus $q_0 \sim r_{[q_0]_M}$.
	
	Now consider some $i+1 > 0$. Then $\rho'(i+1) = r_{[q]_M}$ for $q = \delta(\rho'(i), \alpha(i))$. By induction we know that $\rho(i) \sim \rho'(i)$ and thus $\delta(\rho(i), \alpha(i)) = \rho(i+1) \sim q$. Further, we know $q \sim r_{[q]_M}$ by the same argument as before. Together this lets us conclude in $\rho(i+1) \sim q \sim \rho'(i+1)$.
\end{proof}

The following is a comprehensive list of all relevant merger functions that we use.

\begin{itemize}
	\item Moore merger $\mu_M$. Defined below.
	\item Skip merger, $\mu_\text{skip}^\sim$. Defined in section \ref{sect:skipper}.
\end{itemize}


\subsubsection{Moore merger}
\begin{defn}
	Let $\mathcal{A}$ be a DPA. The Moore merger $\mu_M$ is defined as $\mu_M : \mathfrak{C}(\equiv_M) \rightarrow 2^Q, \kappa \mapsto \kappa$.
\end{defn}

\begin{lem}
	Let $\mathcal{A}$ be a DPA and let $\mathcal{A}'$ be a representative merge of $\mathcal{A}$ w.r.t. $\mu_M$. Then $\mathcal{A}$ and $\mathcal{A}'$ are Moore equivalent.
	\label{lem:general:moore_merge_keeps_mooreequiv}
\end{lem}

\begin{proof}
	%TODO
\end{proof}

\begin{cor}
	Let $\mathcal{A}$ be a DPA and let $\mathcal{A}'$ be a representative merge of $\mathcal{A}$ w.r.t. $\mu_M$. Then $\mathcal{A}$ and $\mathcal{A}'$ are language equivalent.
\end{cor}





\subsection{Reachability}

\begin{defn}
	Let $\mathcal{S} = (Q, \Sigma, \delta)$ be a deterministic transition structure. We define the \emph{reachability order} $\preceq_\text{reach}^\mathcal{S}$ as $p \preceq_\text{reach}^\mathcal{S} q$ if and only if $q$ is reachable from $p$. 
\end{defn}

We want to note here that we always assume for all automata to only have one connected component, i.e. for all states $p$ and $q$, there is a state $r$ such that $p$ and $q$ are both reachable from $r$. In practice, most automata have an predefined initial state and a simple depth first search can be used to eliminate all unreachable states.

\begin{lem}
	$\preceq_\text{reach}^\mathcal{S}$ is a preorder.
\end{lem}

\begin{defn}
	Let $\mathcal{S} = (Q, \Sigma, \delta)$ be a deterministic transition structure. We call a relation $\preceq$ a \emph{total extension of reachability} if it is a minimal superset of $\preceq_\text{reach}^\mathcal{S}$ that is also a total preorder.
	
	For $p \preceq q$ and $q \preceq p$, we write $p \simeq q$.
\end{defn}

\begin{lem}
	For a given deterministic transition structure $\mathcal{S}$, a total extension of reachability is computable in $\mathcal{O}(|\mathcal{S}|)$.
	\label{lem:general:reach_topo_lintime}
\end{lem}

\begin{proof}
	Using e.g. Kosaraju's algorithm \ref{}, the SCCs of $\mathcal{A}$ can be computed in linear time. We can now build a DAG from $\mathcal{A}$ by merging all states in an SCC into a single state; iterate over all transitions $(p, a, q)$ and add an $a$-transition from the merged representative of $p$ to that of $q$. Assuming efficient data structures for the computed SCCs, this DAG can be computed in $O(|\mathcal{A}|)$ time.
	
	To finish the computation of $\preceq$, we look for a topological order on that DAG. This is a total preorder on the SCCs that is compatible with reachability. All that is left to be done is to extend that order to all states.
\end{proof}




















%
\section{Skip Merger}
\label{sect:skipper}

\begin{defn}
	Let $\mathcal{A} = (Q, \Sigma, \delta, c)$ be a DPA and let $\sim \,\subseteq Q \times Q$ be a congruence relation on $\mathcal{A}$. Let $\preceq \,\subseteq Q \times Q$ be a total extension of $\preceq_\text{reach}^\mathcal{A}$. 
	
	We define the \emph{skip merger function} $\mu_\text{skip}^\sim : D \rightarrow 2^Q$ as follows: for each equivalence class $\kappa \in \mathfrak{C}(\sim)$, let $C_\kappa \subseteq \kappa$ be the of $\preceq$-maximal elements in $\kappa$. Let $M_\kappa = \kappa \setminus C_\kappa$. Then we have $D = \{ M_\kappa \mid \kappa \in \mathfrak{C}(\sim) \}$ and $\mu_\text{skip}^\sim(M_\kappa) = C_\kappa$.
\end{defn}

The idea behind the skip merger is that whenever an equivalence class of $\sim$ is reached, the transition is redirected to another element of the same equivalence class that lies as \enquote{deep} in the automaton as possible.

\begin{lem}
\label{lem:skip:skip_aut_linear_time}
	For a given $\mathcal{A}$ and $\sim$, $\mu_\text{skip}^\sim$ can be computed in $\mathcal{O}(|\mathcal{A}|)$.
\end{lem}

\begin{proof}
	As seen in Lemma \ref{lem:general:reach_topo_lintime}, $\preceq$ can be computed in linear time. Assuming that $\sim$ is given by a suitable data structure, each equivalence class can easily be accessed and $\preceq$-maximal elements can be found in linear time.
\end{proof}

\vspace{5pt}

Now that we have established the definition and possible computation of the skip merger function, we want to analyze its structure and prove its correctness. For the rest of this section, we use $\mathcal{A} = (Q, \Sigma, \delta, c)$ as a DPA, $\sim$ as a congruence relation, and $\mathcal{B} = (Q_\mathcal{B}, \Sigma, \delta_\mathcal{B}, c_\mathcal{B})$ as a representative merge of $\mathcal{A}$ w.r.t. $\mu_\text{skip}^\sim$.

\begin{lem}
\label{lem:skip:run_growing}
	Let $\rho$ be a run on $\alpha$ in $\mathcal{B}$. Then for all $i$, $\rho(i) \preceq \rho(i+1)$.
	Furthermore, we have $\rho(i) \prec \rho(i+1)$ if and only if $\rho(i) \prec r_{[\delta(\rho(i), \alpha(i))]_\sim}$.
\end{lem}

\begin{proof}
	Let $i$ be an arbitrary index of the run. If $\rho(i)$ to $\rho(i+1)$ is also a transition in $\mathcal{A}$, then $\rho(i+1)$ is reachable from $\rho(i)$ in $\mathcal{A}$ and hence $\rho(i) \preceq \rho(i+1)$ by definition of the preorder. Otherwise the transition used was redirected in the construction. The way the redirection is defined, this implies $\rho(i) \prec \rho(i+1)$.
	
	We move on to the second part of the lemma. If $\rho(i) \prec r_{[\delta_\mathcal{A}(\rho(i), \alpha(i))]_\sim}$, then the transition is redirected to $\rho(i+1) = r_{[\delta_\mathcal{A}(\rho(i), \alpha(i))]_\sim}$ and the statement holds. 
	
	For the other direction, let $\rho(i) \prec \rho(i+1)$ and assume towards a contradiction that $\rho(i) \not\prec r_{[\delta_\mathcal{A}(\rho(i), \alpha(i))]_\sim}$. This means that the transition was not redirected and $\rho(i+1) = \delta_\mathcal{A}(\rho(i), \alpha(i))$. Since $\preceq$ is total, we have $r_{[\delta_\mathcal{A}(\rho(i), \alpha(i))]_\sim} = r_{[\rho(i+1)]_\sim} \preceq \rho(i) \prec \rho(i+1)$ which contradicts the $\preceq$-maximality of representatives.
\end{proof}

\begin{lem}
\label{lem:skip:equiv_same_scc}
	Let $p, q \in Q_\mathcal{B}$. If $p \sim q$, then $p$ and $q$ lie in the same SCC. 
\end{lem}

\begin{proof}
	It suffices to restrict ourselves to $q = r_{[q]_\sim} = r_{[p]_\sim}$. If we can prove the Lemma for this case, then the general statement follows by transitivity.
	
	Let $p_0$ be a state from which both $p$ and $q$ are reachable. Let $p_0 \cdots p_n$ be a minimal run of $\mathcal{B}$ that reaches $p$. By Lemma \ref{lem:skip:run_growing}, we have $p_0 \preceq \dots \preceq p_n$. Whenever $p_i \prec p_{i+1}$, a redirected transition to the representative $r_{[p_{i+1}]_\sim} = p_{i+1}$ is taken. 
	
	Let $k$ be the first position after which no redirected transition is taken anymore. For the first case, assume that $k < n$. Then $p_i \simeq r_{[p_{i+1}]_\sim}$ for all $i \geq k$. In particular, $p_{n-1} \simeq q$. Since $p_{n-1} \preceq p_n$, we also have $q \preceq p_n$. The representatives are chosen $\preceq$-maximal in their $\sim$-class, so $q \simeq p_n$.
	
	The second case is $k = n$. In that case, the transition from $p_{n-1}$ to $p_n$ is redirected and $p_n = r_{[p_n]_\sim} = q$.
\end{proof}


\begin{lem}
\label{lem:skip:run_suffix}
	Let $\rho \in Q^\omega$ be an infinite run in $\mathcal{B}$ starting at a reachable state. Then $\rho$ has a suffix that is a run in $\mathcal{A}$.
\end{lem} 

\begin{proof}
	We show that only finitely often a redirected transition is used in $\rho$. Then, from some point on, only transitions that also exist in $\mathcal{A}$ are used. The suffix starting at this point is the run that we are looking for.
	
	Let $\rho = p_0 p_1 \cdots$. By Lemma \ref{lem:skip:run_growing}, we have $p_i \preceq p_{i+1}$ for all $i$ and $p_i \prec p_{i+1}$ whenever a redirected transition is taken. As $Q$ is finite, we can only move up in the order finitely often. This proves our claim.
\end{proof}


\begin{theorem}
	Let $\sim \,\subseteq\, \equiv_L$. Then $\mathcal{A}$ and $\mathcal{B}$ are language equivalent.
\end{theorem}

\begin{proof}
	Let $\alpha \in \Sigma^\omega$ be a word and let $\rho$ be of $\mathcal{B}$ starting in $q_0$ on $\alpha$. By Lemma \ref{lem:skip:run_suffix}, $\rho$ has a suffix $\pi$ which is a run segment of $\mathcal{A}$ on some suffix $\beta$ of $\alpha$. The acceptance condition of DPAs is prefix independent, so $\alpha \in L(\mathcal{B}, q_0)$ iff $\rho$ is an accepting run iff $\pi$ is an accepting run. Since the priorities do not change during the construction, $\pi$ is accepting in $\mathcal{B}$ iff it is accepting in $\mathcal{A}$.
	
	Let $w \in \Sigma^*$ be the prefix of $\alpha$ with $\alpha = w \beta$. By Lemma \ref{lem:general:cong_stays_in_merge}, we know that $\delta^*(q_0, w) \sim \delta^*_\mathcal{B}(q_0, w)$. Since every state is $\sim$-equivalent to its representative and $\sim$ is a congruence relation, we also know $\delta^*_\mathcal{B}(q_0, w) \sim \delta^*_\mathcal{B}(r_{[q_0]_\sim}, w)$. From $\delta^*_\mathcal{B}(r_{[q_0]_\sim}, w)$, the run $\pi$ accepts $\beta$ iff $\alpha \in L(\mathcal{B}, q_0)$. As $\sim$ implies language equivalence, the same must hold for $\delta^*_\mathcal{A}(q_0, w)$. Therefore, $\alpha \in L(\mathcal{A}, q_0)$ iff $\alpha \in L(\mathcal{B}, q_0)$.
\end{proof}





\section{Fritz \& Wilke}

\subsection{Delayed Simulation Game}
In this section we consider delayed simulation games and variants thereof on DPAs. This approach is based on the paper \cite{} which considered the games for alternating parity automata. The DPAs we use are a special case of these APAs and therefore worth examining.

\begin{defn}
	We define $\leq_\checkmark \subseteq (\mathbb{N} \cup \{\checkmark\}) \times (\mathbb{N} \cup \{\checkmark\})$ as follows:
	\begin{itemize}
		\item For $i, j \in \mathbb{N}$, we set $i \leq_\checkmark j$ iff $i \leq j$.
		\item For all $i \in \mathbb{N}$, we have $i \leq_\checkmark \checkmark$ and $\checkmark \not\leq_\checkmark i$.
		\item $\checkmark \leq_\checkmark \checkmark$.
	\end{itemize}
	
	Further, we define an order of \enquote{goodness} on parity priorities as $\preceq_\text{p} \subseteq \mathbb{N} \times \mathbb{N}$ as $0 \preceq_\text{p} 2 \preceq_\text{p} 4 \preceq_\text{p} \dots \preceq_\text{p} 5 \preceq_\text{p} 3 \preceq_\text{p} 1$.
\end{defn}

\begin{defn}
	Let $\mathcal{A} = (Q, \Sigma, q_0, \delta, c)$ be a DPA. We define the \emph{delayed simulation automaton} $\mathcal{A}_\text{de}(p, q) = (Q_\text{de}, \Sigma, (p, q, \gamma(c(p), c(q), \checkmark)), \delta_\text{de}, F_\text{de})$, which is a deterministic Büchi automaton, as follows.
	
	\begin{itemize}
		\item $Q_\text{de} = Q \times Q \times (\text{img}(c) \cup \{ \checkmark \})$, i.e. the states are given as triples in which the first two components are states from $\mathcal{A}$ and the third component is either a priority from $\mathcal{A}$ or $\checkmark$.
		\item The alphabet remains $\Sigma$.
		\item The starting state is a triple $(p, q, \gamma(c(p), c(q), \checkmark))$, where $p, q \in Q$ are parameters given to the automaton, and $\gamma$ is defined below.
		\item $\delta_\text{de}( (p, q, i), a ) = ( p', q', \gamma(i, c(p'), c(q'))$, where $p' = \delta(p, a)$, $q' = \delta(q, a)$, and $\gamma$ is the same function as used in the initial state. The first two components behave like a regular product automaton.
		\item $F_\text{de} = Q \times Q \times \{ \checkmark \}$.
	\end{itemize}
	
	$\gamma : \mathbb{N} \times \mathbb{N} \times (\mathbb{N} \cup \{\checkmark\}) \rightarrow \mathbb{N} \cup \{\checkmark\}$ is the update function of the third component and defines the \enquote{obligations} as they are called in \cite{}. It is defined as 
	$$ \gamma(i, j, k) = \begin{cases}
		\checkmark & \text{if } i \text{ is odd and } i \leq_\checkmark k \text{ and } j \preceq_\text{p} i \\
		\checkmark & \text{if } j \text{ is even and } j \leq_\checkmark k \text{ and } j \preceq_\text{p} i \\
		\min_{\leq_\checkmark} \{ i,j,k \} & \text{else}
	\end{cases} $$
\end{defn}


\begin{defn}
	Let $\mathcal{A}$ be a DPA and let $\mathcal{A}_\text{de}$ be the delayed simulation automaton of $\mathcal{A}$. We say that a state $p$ $de$-simulates a state $q$ if $L(\mathcal{A}_\text{de}(p, q)) = \Sigma^\omega$. In that case we write $p \leq_\text{de} q$. If also $q \leq_\text{de} p$ holds, we write $p \equiv_\text{de} q$.
\end{defn}



\subsubsection*{$\boldsymbol{\equiv_\text{de}}$ is a congruence relation.}
Our overall goal is to use $\equiv_\text{de}$ to build a quotient automaton of our original DPA. The first step towards this goal is to show that the result is actually a well-defined DPA, by proving that the relation is a congruence.

\begin{lem}
	$\gamma$ is monotonous in the third component, i.e. if $k \leq_\checkmark k'$, then $\gamma(i, j, k) \leq_\checkmark \gamma(i, j, k')$ for all $i, j \in \mathbb{N}$.
\end{lem}

\begin{proof}
	We consider each case in the definition of $\gamma$. If $i$ is odd, $i \leq_\checkmark k$ and $j \preceq_\text{p} i$, then also $i \leq_\checkmark k'$ and $\gamma(i, j, k) = \gamma(i, j, k') = \checkmark$.
	
	If $j$ is even, $j \leq_\checkmark k$ and $j \preceq_\text{p} i$, then also $j \leq_\checkmark k'$ and $\gamma(i, j, k) = \gamma(i, j, k') = \checkmark$.
	
	Otherwise, $\gamma(i, j, k) = \min \{i, j, k\}$ and $\gamma(i, j, k') = \min \{i, j, k'\}$. Since $k \leq_\checkmark k'$, $\gamma(i, j, k) \leq_\checkmark \gamma(i, j, k')$.
\end{proof}

\begin{lem}
	Let $\mathcal{A}$ be a DPA and let $p, q \in Q$, $k \in \mathbb{N} \cup \{\checkmark\}$. If the run of $\mathcal{A}_\text{de}$ starting at $(p, q, k)$ on some $\alpha \in \Sigma^\omega$ is accepting, then for all $k \leq_\checkmark k'$ also the run of $\mathcal{A}_\text{de}$ starting at $(p, q, k')$ on $\alpha$ is accepting.
\end{lem}

\begin{proof}
	Let $\rho$ be the run starting at $(p, q, k)$ and let $\rho'$ be the run starting at $(p, q, k')$. Further, let $p_i$, $q_i$, $k_i$, and $k'_i$ be the components of the states of those runs in the $i$-th step. Via induction we show that $k_i \leq_\checkmark k'_i$ for all $i$. Since $k_i$ is $\checkmark$ infinitely often, the same must be true for $k'_i$ and $\rho'$ is accepting.
	
	For $i = 0$, we have $k_0 = k \leq_\checkmark k' = k'_0$. Otherwise, we have $k_{i+1} = \gamma(c(p_{i+1}), c(q_{i+1}), k_i)$ and $k'_{i+1}$ analogously. The rest follows from Lemma \ref{}.
\end{proof}

\begin{lem}
	Let $\mathcal{A}$ be a DPA and $\rho \in Q_\text{de}^\omega$ be a run of $\mathcal{A}_\text{de}$ on some word, where the third component is $k \in (\mathbb{N} \cup \{\checkmark\})^\omega$. For all $i$, $k(i+1) \leq_\checkmark k(i)$ or $k(i+1) = \checkmark$.
\end{lem}

\begin{proof}
	Follows directly from the definition of $\gamma$.
\end{proof}

\begin{lem}
	Let $\mathcal{A}$ be a DPA and $\rho$ be a run of $\mathcal{A}_\text{de}$ on some word $\alpha$ with $\rho(i) = (p(i), q(i), k(i))$. Let $m < n$ be two distinct positions such that $k(m) = k(n) = \checkmark$ and at no positition inbetween does $k$ become $\checkmark$. Then for all $m < i \leq n$, $c(q(n)) \preceq_p c(p(i))$ and $c(q(n)) \preceq_p c(q(i))$.
\end{lem}

\begin{proof}
	%TODO
\end{proof}

\begin{lem}
	Let $\mathcal{A}$ be a DPA. Then $\leq_\text{de}$ is reflexive and transitive.
\end{lem}

\begin{proof}
	For reflexivitiy, we need to show that $q \leq_\text{de} q$ for all states $q$. This is rather easy to see. For a word $\alpha \in \Sigma^\omega$, the third component of states in the run of $\mathcal{A}_\text{de}(q, q)$ on $\alpha$ is always $\checkmark$, as $\gamma(i, i, \checkmark) = \checkmark$.
	
	For transitivity, let $q_1 \leq_\text{de} q_2$ and $q_2 \leq_\text{de} q_3$. Assume towards a contradiction that $q_1 \not\leq_\text{de} q_3$, so there is a word $\alpha \notin L(\mathcal{A}_\text{de}(q_1, q_3))$. We consider the three runs $\rho_{12}$, $\rho_{23}$, and $\rho_{13}$ of $\mathcal{A}_\text{de}(q_1, q_2)$, $\mathcal{A}_\text{de}(q_2, q_3)$, anbd $\mathcal{A}_\text{de}(q_1, q_3)$ respectively on $\alpha$. Then $\rho_{12}$ and $\rho_{23}$ are accepting, whereas $\rho_{13}$ is not. 
	
	Moreover, we use the notation $q_1(i), q_2(i), q_3(i)$ for the states of the run and $k_{12}(i), k_{23}(i), k_{13}(i)$ for the obligations. More specifically for a run $\rho_{ij}$, it is true that $\rho_{ij}(n) = (q_i(n), q_j(n), k_{ij}(n))$.
	
	As $\rho_{13}$ is not accepting, $k_{13}$ becomes $\checkmark$ only finitely often. By Lemma \ref{}, that means $k_{13}$ only grows smaller from some point on and reaches a minimum eventually. Let $n_0 \in \mathbb{N}$ be such a position from which on $k_{13}$ does not change anymore. We split the rest of the proof into two cases, depending on which priority from the following two states caused the change to $k_{13}(n_0)$.
	
	\paragraph{Case 1: $k_{13}(n_0) = c(q_1(n_0))$.} We know $c(q_1(n_0)) < c(q_3(n_0))$ and $c(q_1(n_0)) \prec_p c(q_3(n_0))$, so $c(q_1(n_0))$ must be even. Let $n_1 \geq n_0$ be the smallest position such that $k_{12}(n_1) = \checkmark$. By Lemma \ref{}, $c(q_2(n_1)) \preceq_p c(q_1(n_0))$. 
	
	Let $n_2 \geq n_1$ be the smallest position such that $k_{23} = \checkmark$. Again by lemma \ref{}, $c(q_3(n_2)) \preceq_p c(q_2(n_1)) \preceq_p c(q_1(n_0))$. As $c(q_1(n_0))$ is even, so must be $c(q_3(n_2))$. From these facts and the definition of $\gamma$ we can deduce 
	
	\begin{align*}
		& k_{13}(n_2) \\
		&= \gamma(c(q_1(n_2), c(q_3(n_2)), k_{13}(n_0)) \\
		&= \begin{cases}
		\checkmark & \text{if } c(q_3(n_2)) \leq_\checkmark k_{13}(n_0) \\
		\min_{\leq_\checkmark} \{c(q_1(n_2), c(q_3(n_2)), k_{13}(n_0)\} & \text{else}
	\end{cases}
	\end{align*}
	
	Finally, $k_{13}(n_0)$ is the same as $c(q_1(n_0))$, so $c(q_3(n_2)) \leq_\checkmark k_{13}(n_0)$ as both values are even. Thus, $k_{13}(n_2) = \checkmark$ and $n_2 \geq n_0$ which contradicts our choice of $n_0$.
	
	\paragraph{Case 2: $k_{13}(n_0) = c(q_3(n_0))$.} We know $c(q_3(n_0)) < c(q_1(n_0))$ and $c(q_1(n_0)) \prec_p c(q_3(n_0))$, so $c(q_3(n_0))$ is odd. %TODO
\end{proof}

\begin{lem}
	Let $\mathcal{A}$ be a DPA. Then $\equiv_\text{de}$ is a congruence relation.
\end{lem}

\begin{proof}
	The three properties that are required for $\equiv_\text{de}$ to be a equivalence relation are rather easy to see. Reflexivity and transitivity have been shown for $\leq_\text{de}$ already and symmetry follows from the definition. Congruence requires more elaboration.

	Let $p \equiv_\text{de} q$ be two equivalent states. Let $a \in \Sigma$ and $p' = \delta(p, a)$ and $q' = \delta(q, a)$. We have to show that also $p' \equiv_\text{de} q'$. Towards a contradiction, assume that $p' \not\leq_\text{de} q'$, so there is a word $\alpha \notin L(\mathcal{A}_\text{de}(p', q'))$. Let $(p', q', k) = \delta_\text{de}((p, q, \checkmark), a)$. By Lemma \ref{}, the run of $\mathcal{A}_\text{de}$ on $\alpha$ from $(p', q', k)$ cannot be accepting; otherwise, the run of $\mathcal{A}_\text{de}$ from $(p', q', \checkmark)$ would be accepting and $\alpha \in L(\mathcal{A}_\text{de}(p', q'))$. Hence, $a \alpha \notin L(\mathcal{A}_\text{de}(p, q))$, which means that $p \not\equiv_\text{de} q$.
\end{proof}

\begin{cor}
	Let $\mathcal{A}$ be a DPA and $\equiv_\text{de}$ the corresponding delayed simulation-relation. The quotient automaton $\bigslant{\mathcal{A}}{\equiv_\text{de}}$ is well-defined and deterministic.
\end{cor}

\subsubsection*{Correctness of the quotient}
%TODO
















\chapter{Iterated Moore Equivalence}
\label{chap:imoore}

\section{Theory}
Standard Moore equivalence was created to minimize Moore automata in which the priority of every state matters; in other words, the occurrence set of a given run is considered. Parity automata on the other hand only use the smaller infinity set. The idea of the upcoming approach is to use this relaxation to give the standard Moore equivalence more freedom and remove additional states.

\begin{defn}
	Let $\mathcal{A} = (P, \Sigma, \delta, c)$ and $\mathcal{B} = (Q, \Sigma, \varepsilon, d)$ be DPAs and $S \subseteq P$. We say that \emph{$\mathcal{A}$ prepends $S$ to $\mathcal{B}$} if 
	\begin{itemize}
		\item $Q \cap S = \emptyset$
		\item $P = Q \cup S$
		\item $\delta\upharpoonright_{Q \times \Sigma} = \varepsilon$
		\item $c\upharpoonright_Q = d$
	\end{itemize}
	
	We assume $S$ to be an SCC in these use cases, i.e. from every $s \in S$, every other $s' \in S$ is reachable in $\mathcal{A}$.
\end{defn} 

\begin{defn}
\label{def:fwe:itmoore}
	Let $\mathcal{A} = (Q, \Sigma, \delta, c)$ be a DPA with SCCs $\mathcal{S} \subseteq 2^Q$. Let $\preceq \,\subseteq \mathcal{S} \times \mathcal{S}$ be a total preorder on the SCCs such that $S \preceq S'$ implies that $S'$ is reachable from $S$. For the $i$-th element w.r.t. this order, we write $S_i$, i.e. $S_0 \prec S_1 \prec \dots \prec S_{|\mathcal{S}|}$. 
	
	For every state $q$ in $\mathcal{A}$, let $\text{SCC}(q)$ be the SCC of $q$ and let $\text{SCCi}(q)$ be the index of that SCC, i.e. $\text{SCC}(q) = S_{\text{SCCi}(q)}$. Let $\preceq_Q \,\subseteq Q \times Q$ be a total preorder on the states such that $q \preceq_Q q'$ implies $\text{SCCi}(q) \leq \text{SCCi}(q')$.
	
	We inductively define a sequence of automata $(\mathcal{B}_i)_{0 \leq i \leq |\mathcal{S}|}$. For every $i$, we write $\mathcal{B}_i = (Q_i, \Sigma, \delta_i, c_i)$.
	
	\begin{itemize}
		\item The state sets are defined as $Q_i = \bigcup_{j=i}^{|\mathcal{S}|} S_j$.
		\item The base case is $\mathcal{B}_{|\mathcal{S}|} = \mathcal{A}\upharpoonright_{S_{|\mathcal{S}|}}$.
		\item Given that $\mathcal{B}_{i+1}$ is defined, let $\mathcal{B}'_i = (Q_i, \Sigma, q_0^{\prime i}, \delta_i, c'_i)$ be a DPA that prepends $S_i$ to $\mathcal{B}_{i+1}$ such that $\delta \upharpoonright_{Q_i \times \Sigma} = \delta_i$ and $c \upharpoonright_{S_i} = c'_i \upharpoonright_{S_i}$.
		\item Let $M'_i$ be the Moore equivalence on $\mathcal{B}'_i$. If $S_i = \{q\}$ is a trivial SCC, $q$ is not $M'_i$-equivalent to any other state, and there is a $p \in Q_{i+1}$ such that for all $a \in \Sigma$ $(\delta'_i(q, a), \delta'_i(p, a)) \in M'_i$, then let $p_0$ be $\preceq_Q$-maximal among those $p$ and let $c_i(r) = \begin{cases} c_{i+1}(p_0) & \text{if } r = q \\ c_{i+1}(r) & \text{else} \end{cases}$. If any of the three conditions is false, simply set $c_i = c'_i$.
	\end{itemize} 
	
	Let $M_i$ be the Moore equivalence on $\mathcal{B}_i$. We define $\equiv_{IM} \,:= M_0$ and call this the \emph{iterated Moore equivalence} of $\mathcal{A}$.
\end{defn}

At first, this definition might confuse when written down formally like this. We continuously add the SCCs of $\mathcal{A}$ starting from the \enquote{back}. In addition to computing the usual Moore equivalence on our automaton, we also take trivial SCCs into special consideration; as their priority cannot appear infinitely often on any run, its value is effectively arbitrary. We can therefore perform extra steps to more liberally merge it with other states.


\begin{defn}
	Let $\mathcal{A} = (Q, \Sigma, \delta, c)$ be a DPA. We define the \emph{iterated Moore merger function} $\mu_{IM} : \mathfrak{C}(\equiv_{IM}, \mathcal{A}) \rightarrow 2^Q$ as $\mu_{IM}(\kappa) = \{ q \in \kappa \mid q \text{ is } \preceq_\text{reach}^\mathcal{A} \text{-maximal in } \kappa \}$.
\end{defn}

\begin{theorem}
	Let $\mathcal{A}$ be a DPA and let $\mathcal{A}'$ be a representative merge of $\mathcal{A}$ w.r.t. $\mu_{IM}$. For all $q \in Q'$, $(\mathcal{A}, q) \equiv_L (\mathcal{A}', q)$.
\end{theorem}

\begin{proof}
	By definition of the iterated Moore equivalence, we only have $c(p) \neq c'(p)$ if $\{p\}$ is a trivial SCC in $\mathcal{A}$. By having candidates of the merge be $\preceq_\text{reach}^\mathcal{A}$-maximal, we assure that for all chosen representatives $r_\kappa$, $c(r_\kappa) = c_{|\mathcal{S}|}(r_\kappa)$. With Lemma \ref{lem:general:trivial_scc_dont_matter}, this finishes our proof.
\end{proof}

\vspace{10pt}

Unfortunately, we were not able to find or prove any relation between iterated Moore equivalence and other relations we established so far. We therefore end this section with two open questions.

\begin{itemize}
	\item Is $|\mathfrak{C}(\equiv_M, \mathcal{A})| \geq |\mathfrak{C}(\equiv_{IM}, \mathcal{A})|$ true for all $\mathcal{A}$?
	\item If $\mathcal{A}$ is $c$-normalized, what is the relation between $\equiv_{IM}$ and $\equiv_\text{de}$?
\end{itemize}


\section{Computation}
\begin{lem}
	For a given DPA $\mathcal{A}$, $\mu_{IM}$ can be computed in $\mathcal{O}(|\mathcal{S}| \cdot |Q| \cdot \log |Q|)$.
\end{lem}

\begin{proof}
	$\equiv_{IM}$ is already basically defined as an algorithm. We use $|\mathcal{S}|$ many steps of computing Moore equivalence and then checking whether the priority of the newly added state should be changed. The first can be done in log-linear time (Corollary \ref{cor:general:M_loglin}), the second in linear time. This gives us a runtime of $\mathcal{O}(|\mathcal{S}| \cdot |Q| \cdot \log |Q|)$ to calculate $\equiv_{IM}$.

	To build $\mu_{IM}$ from that can be done in linear time by Lemma \ref{lem:general:reach_topo_lintime}.
\end{proof}


\section{Efficiency}
Figure \ref{fig:imoore:empirical_size_hist} shows that the iterated Moore merger actually performs an acceptable job (considering its low complexity) at reducing the number of states in \textsf{detspot} and \textsf{detnbaut}, even when compared to the normal Moore merger in figure \ref{fig:general:empirical_moore_size_hist}. However, the absolute number of removed states actually stays rather low for \textsf{detnbaut} and consequently the automata that witness a large relative reduction have a small number of states to begin with. This can be seen in figure \ref{fig:imoore:empirical_reduct_abs}: \textsf{detnbaut} rarely has more than 10 states removed, and \textsf{detspot} stays within a similar range of reduction as figure \ref{fig:general:empirical_moore_reduct_abs}, with the exception of a few outliers.

This result is to be expected. As figure \ref{fig:rawstats:rawstats_sccs} showed, few automata contain more than three SCCs. The iterated Moore procedure can only exceed the normal Moore quotient by using trivial SCCs.

As predicted by the theoretical complexity of the algorithm, figure \ref{fig:imoore:empirical_time} confirms that the iterated Moore merger is fast to apply, only reaching a run time of one second at 200 states.


\begin{figure}
	\centering
	\begin{minipage}{0.49\textwidth}
		\includegraphics[page=6,height=.3\textheight]{../data/analysis/iterated_moore/gendet_ap1.pdf} 
		\includegraphics[page=6,height=.3\textheight]{../data/analysis/iterated_moore/detspot_ap1.pdf} 
		\includegraphics[page=6,height=.3\textheight]{../data/analysis/iterated_moore/detnbaut_ap1.pdf} 
		\caption{State reduction of different automata using $\mu_{IM}$.}
		\label{fig:imoore:empirical_size_hist}
	\end{minipage}
	\hfill
	\begin{minipage}{0.49\textwidth}
		\includegraphics[page=2,height=.3\textheight]{../data/analysis/iterated_moore/gendet_ap1.pdf} 
		\includegraphics[page=2,height=.3\textheight]{../data/analysis/iterated_moore/detspot_ap1.pdf} 
		\includegraphics[page=2,height=.3\textheight]{../data/analysis/iterated_moore/detnbaut_ap1.pdf} 
		\caption{State reduction of different automata using $\mu_{IM}$.}
		\label{fig:imoore:empirical_reduct_abs}
	\end{minipage}
\end{figure}


\begin{figure}
	\centering
	\begin{minipage}{0.49\textwidth}
		\includegraphics[page=1,height=.3\textheight]{../data/analysis/iterated_moore/gendet_ap1.pdf} 
		\includegraphics[page=1,height=.3\textheight]{../data/analysis/iterated_moore/detspot_ap1.pdf} 
		\includegraphics[page=1,height=.3\textheight]{../data/analysis/iterated_moore/detnbaut_ap1.pdf} 
		\caption{Run time of the state reduction of different automata using $\mu_{IM}$.}
		\label{fig:imoore:empirical_time}
	\end{minipage}
\end{figure}





\section{Congruence Path Refinement}

In this section we present an algorithm that uses an existing congruence relation and refines it to the point where equivalent states can be \enquote{merged}.

\begin{defn}
	Let $\mathcal{A} = (Q, \Sigma, q_0, \delta, c)$ be a DPA and let $\equiv \,\subseteq Q \times Q$ be an equivalence relation on the state set. For every equivalence class $\kappa \subseteq Q$, let $r_\kappa \in \kappa$ be an arbitrary representative of that class. For a DPA $\mathcal{A}' = (Q', \Sigma, q'_0, \delta', c')$, we say that $\mathcal{A}'$ is a \emph{representative merge of $\mathcal{A}$ w.r.t. $\equiv$} if it satisfies the following:
	\begin{itemize}
		\item $Q' = \{ r_{[q]_\equiv} \subseteq Q \mid q \in Q \}$
		\item $q'_0 = r_{[q_0]_\equiv}$
		\item For all $q \in Q'$ and $a \in \Sigma$: $\delta'(q, a) = r_{[\delta(q, a)]_\equiv}$
		\item $c' = c\upharpoonright_{Q'}$
	\end{itemize}
\end{defn}

\begin{defn}
	Let $\mathcal{A} = (Q, \Sigma, q_0, \delta, c)$ be a DPA and let $R \subseteq Q \times Q$ be a congruence relation on the state space. For any states $p, q \in Q$, let $L_{[p]_R \rightarrow [q]_R} \subseteq \Sigma^*$ be the set of words $w$ such that for any $u \sqsubseteq w$, $(\delta(p, u), q) \in R$ iff $u \in \{\varepsilon, w\}$. In other words, the set contains all minimal words by which the automaton reaches $[q]_R$ from $[p]_R$.
	
	Let $\kappa \subseteq Q$ be an equivalence class of $R$ and let $p, q \in \kappa$. We define $R_\kappa \subseteq \kappa \times \kappa$ as the largest set $(p, q) \in R_\kappa$ iff the following holds for all words $w \in L_{\kappa \rightarrow \kappa}$:
	\begin{itemize}
		\item $(\delta^*(p, w), \delta^*(q, w)) \in R_\kappa$
		\item $\min \{ c(\delta^*(p, u)) \mid u \sqsubset w \} = \min \{ c(\delta^*(q, u)) \mid u \sqsubset w \}$
	\end{itemize}
	
	Finally, we call $\equiv_\text{PR}^R \,= \bigcup_{q \in Q} R_{[q]_R}$ the \emph{path refinement of $R$}.
\end{defn}

\begin{lem}
	The path refinement is a well defined equivalence relation.
	\label{lem:pr:pr_well_def}
\end{lem}

\begin{proof}
	We have to consider the sets $L_{[p]_R \rightarrow [q]_R}$ and the sets $R_\kappa$. For $L_{[p]_R \rightarrow [q]_R}$, the definition works because $R$ has the congruence property. 
	
	For $R_\kappa$, consider the following function $f : 2^{Q \times Q} \rightarrow 2^{Q \times Q}$: 
	$$ f(X) = \{ (p, q) \in X \mid \text{for all } w \in L_{\kappa \rightarrow \kappa}: (\delta^*(p, w), \delta^*(q, w)) \in X \}$$
	$$ Y_\kappa = \{ (p, q) \in Q \times Q \mid \text{for all } w \in L_{\kappa \rightarrow \kappa}: \min \{ c(\delta^*(p, u)) \mid u \sqsubset w \} = \min \{ c(\delta^*(q, u)) \mid u \sqsubset w \} \} $$

	Now Let $X_0 = Y_\kappa$ and $X_{i+1} = f(X_i)$. $f$ is monotone w.r.t. $\subseteq$, so there must be a fixed point $X_\infty$. By Kleene's fixed point theorem and from the definition of $R_\kappa$, we have $X_\infty = \text{gfp}(f) = R_\kappa$.
	
	Every $X_i$ is an equivalence relation on $\kappa$: for $i = 0$, every state is only equivalent to itself, and for $i > 0$, the three properties can easily be verified via induction. Hence, $X_\infty = R_\kappa$ is also an equivalence relation. All $R_\kappa$ are disjoint and thus $\equiv_\text{PR}^R$ has to be an equivalence relation as well.
\end{proof}

\begin{theorem}
	Let $\mathcal{A} = (Q, \Sigma, q_0, \delta, c)$ be a DPA and let $R \subseteq Q \times Q$ be a congruence relation that implies language equivalence. Let $\mathcal{B}$ be a representative merge of $\mathcal{A}$ w.r.t. $\equiv_\text{PR}^R$. Then $L(\mathcal{A}) = L(\mathcal{B})$.
\end{theorem}

\begin{proof}
	Let $n$ be the number of non-trivial equivalence classes in $\equiv_\text{PR}^R$, i.e. classes with size greater than 1. If $n = 0$, then $p \equiv_\text{PR}^R q$ iff $p = q$ and therefore $\mathcal{B} = \mathcal{A}$. 
	
	Now assume for an argument of induction that the statement is true for $n$ and we want to show that it still holds for $n+1$ classes. Let $\kappa \subseteq Q$ be an arbitrary non-trivial equivalence class of $\equiv_\text{PR}^R$. Let $\mathcal{A}' = (Q', \Sigma, q'_0, \delta', c')$ be the representative merge of $\mathcal{A}$ w.r.t. $\equiv_\text{PR}^R \upharpoonright_\kappa$ with the same representative $r_\kappa$ as in $\mathcal{B}$. The path refinement equivalence of $\mathcal{A}'$ then is equal to $\equiv_\text{PR}^R \upharpoonright_{Q'}$ and has $n$ non-trivial equivalence classes (as $\kappa$ was merged into a single state). By induction, $L(\mathcal{A}') = L(\mathcal{B})$. It remains to be proven that $L(\mathcal{A}) = L(\mathcal{A}')$.
	
	Let $\alpha \in \Sigma^\omega$ be a word with runs $\rho \in Q^\omega$ and $\rho' \in (Q')^\omega$ of $\mathcal{A}$ and $\mathcal{A}'$ respectively. Let $\lambda \subseteq Q$ be the equivalence class of $R$ from which $\kappa$ was extracted.

	\vspace{5pt}
	\textbf{Claim 1}: At every position $i$, $\rho(i) \in \kappa$ iff $\rho'(i) \in \kappa$. \\
	Let $k_0$ be the first position at which $\rho(k_0) \in \kappa$ is true. For all $i < k_0$, we have $\rho(i) = \rho'(i)$, and at $k_0$ we have $\rho(k_0) \equiv_\text{PR}^R r_\kappa = \rho'(k_0)$.
	
	Now assume that the claim holds for all $i \leq k$, where $k$ is a position at which $\rho(k) \in \kappa$. Let $l > k$ be the next position at which $\rho(l) \in \lambda$. If $l$ does not exist, then neither $\rho(i)$ nor $\rho'(i)$ are elements of $\kappa$ for any $i > k$.
	
	Let $w = \alpha[k, l]$. Since $\kappa \subseteq \lambda$, $w \in L_{\lambda \rightarrow \lambda}$. By definition of $\equiv_\text{PR}^R$, that means $\delta^*(\rho(k), w) = \rho(l) \equiv_\text{PR}^R \delta^*(\rho'(k), w)$. Between $k$ and $l-1$, no redirected edge is used in $\rho'$, so $\delta^*(\rho'(k), \alpha[k, l-1]) = \rho'(l-1)$. Finally, $\rho'(l) = \delta'(\rho'(l-1), \alpha(l)) = r_{[\delta(\rho'(l-1), \alpha(l)]_{\equiv_\text{PR}^R}} \equiv_\text{PR}^R \delta(\rho'(l-1), \alpha(l)) = \delta^*(\rho'(k), w)$. Thus, $\rho(l) \equiv_\text{PR}^R \rho'(l)$.
	
	Now, if $\rho(l) \in \kappa$, then $\rho(l+1) \in \kappa$ and our proof of induction is complete. If $\rho(l) \notin \kappa$, then $\rho'(l) = \rho(l)$, so the runs visit the same states in all positions until $\kappa$ is reached again. This also completes the proof of our claim.
	
	\vspace{5pt}
	\textbf{Claim 2}: If $\kappa$ only occurs finitely often in $\rho$ and $\rho'$, then $\rho$ is accepting iff $\rho'$ is accepting.
	
	Let $k \in \mathbb{N}$ be the last position at which $\rho(k)$ and $\rho'(k)$ are in $\kappa$. From this point on, $\rho'[k, \omega]$ is also a valid run of $\mathcal{A}$ on $\alpha[k, \omega]$. $\rho(k) \equiv_\text{PR}^R \rho'(k)$, so $(\rho(k), \rho'(k)) \in R$. As $R$ implies language equivalence, reading $\alpha[k, \omega]$ from either state in $\mathcal{A}$ leads to the same acceptance status. This also means that $\rho'(k)$ has the same acceptance status as $\rho(k)$.
	
	\vspace{5pt}
	\textbf{Claim 3}: If $\kappa$ occurs infinitely often in $\rho$ and $\rho'$, then $\rho$ is accepting iff $\rho'$ is accepting.
	
	Let $(k_i)_{i \in \mathbb{N}}$ be all positions at which $\kappa$ is visited. For each $k_i$, let $l_i > k_i$ be the minimal position at which $\rho(l_i) \in \lambda$. In two steps, we first show that $c(\rho[l_i, k_{i+1}]) = c'(\rho'[l_i, k_{i+1}])$ and second that $\min \text{Occ}(c(\rho[k_i, l_i])) = \min \text{Occ}(c'(\rho'[k_i, l_i]))$. Together, these results mean that the minimal priority that is seen infinitely often in the two runs is the same.
	
	First, observe that at every $l_i$, we either have $l_i = k_{i+1}$ (if $\rho(l_i) \in \kappa$) or $\rho(l_i) = \rho'(l_i)$. In the first case, $\rho[l_i, k_{i+1}]$ is empty, so $c(\varepsilon) = c'(\varepsilon)$ is true. In the second case, $\rho[l_i, k_{i+1}] = \rho'[l_i, k_{i+1}]$ and therefore $c(\rho[l_i, k_{i+1}]) = c'(\rho'[l_i, k_{i+1}])$.
	
	Second, let $w_i = \alpha[k_i, l_i]$. Then $\alpha \in L_{\lambda \rightarrow \lambda}$ and $\min \text{Occ}(c(\rho[k_i, l_i])) = \min \text{Occ}(c'(\rho'[k_i, l_i]))$ holds directly by definition of $\equiv_\text{PR}^R$.
\end{proof}



\subsection{Algorithmic Definition}
The definition of path refinement that we introduced is useful for the proofs of correctness. It however does not provide one with a way to actually compute the relation. That is why we now provide an alternative definition that yields the same results but is more algorithmic in nature.

\begin{defn}
	Let $\mathcal{A} = (Q, \Sigma, q_0, \delta, c)$ be a DPA and let $R \subseteq Q \times Q$ be a congruence relation. For each equivalence class $\lambda$ of $R$, we define the \emph{path refinement automaton} $\mathcal{G}_\text{PR}^{R,\lambda}(p, q) = (Q_\text{PR}, \Sigma, q_{0, \text{PR}}^{p,q}, \delta^\lambda_\text{PR}, F_\text{PR})$, which is a DFA.
	
	\begin{itemize}
		\item $Q_\text{PR} = (Q \times Q \times c(Q) \times \{<, >, =\}) \cup \{ \perp \}$
		\item $q_{0, \text{PR}}^{p,q} = (p, q, \eta_k(c(p), c(q), \checkmark), \eta_x(c(p), c(q), \checkmark, =))$
		\item $\delta^\lambda_\text{PR}((p, q, k, x), a) = \begin{cases}
			(p', q', \eta_k(c(p'), c(q'), k), \eta_x(c(p'), c(q'), k, x)) & \text{if } p' \notin \lambda \\
			q_{0,\text{PR}}^{p',q'} & \text{if } p' \in \lambda \text{ and } (x =\, =) \\
			\perp & \text{else}
		\end{cases}$ \\
			where $p' = \delta(p, a)$ and $q' = \delta(q, a)$. \\
			$\eta_k(k_p, k_q, k) = \min_{\leq_\checkmark} \{k_p, k_q, k\}$ \\
			$\eta_x(k_p, k_q, k, x) = \begin{cases}
				< & \text{if } (k_p <_\checkmark k_q \text{ and } k_p <_\checkmark k) \text{ or } (k < k_q \text{ and } (x =\, <)) \\
				> & \text{if } (k_p >_\checkmark k_q \text{ and } k >_\checkmark k_q) \text{ or } (k_p > k \text{ and } (x =\, >)) \\
				= & \text{else}
			\end{cases}$ 
		\item $F_\text{PR} = Q_\text{PR} \setminus \{\perp\}$
	\end{itemize}
\end{defn}

\begin{lem}
	Let $\mathcal{A}$ be a DPA with a congruence relation $R$. Let $\lambda$ be an equivalence class of $R$, $p, q \in \lambda$, and $w \in L_{\lambda \rightarrow \lambda}$. For every $v \sqsubset w$ and $\oplus \in \{<, >, =\}$, the fourth component of $(\delta_\text{PR}^\lambda)^*(q_{0,\text{PR}}, v)$ is $\oplus$ if and only if $\min \{ c(\delta^*(p, u)) \mid u \sqsubseteq v \} \oplus \min \{ c(\delta^*(q, u)) \mid u \sqsubseteq v \}$.
\end{lem}

\begin{proof}
	This proof is a rather formal analysis of the definition of $\eta_x$. For $v = \varepsilon$, we have to show that the fourth component $x$ of $q_{0,\text{PR}}^{p,q}$ is $\oplus$ iff $c(p) \oplus c(q)$. We can simplify 
	$$x = \eta_x(c(p), c(q), \checkmark, =) = \begin{cases}
			< & \text{if } k_p <_\checkmark k_q \\
			> & \text{if } k_p >_\checkmark k_q \\
			= & \text{else}
		\end{cases}.$$
	This is exactly what we hoped to find.
	
	Now let $v = v' a$ and assume the statement is true for $v'$. Set $m_p = \min \{ c(\delta^*(p, u)) \mid u \sqsubseteq v' \}$ and $m_q$ analogously. Let $k_p = c(\delta^*(p, v))$ and $k_q = c(\delta^*(q, v))$. %TODO
\end{proof}

\begin{theorem}
	Let $\mathcal{A}$ be a DPA with a congruence relation $R$. Let $\lambda$ be an equivalence class of $R$ and $p, q \in \lambda$. Then $p \equiv_\text{PR}^R q$ iff $L(\mathcal{G}_\text{PR}^{R,\lambda}(p, q)) = \Sigma^*$.
\end{theorem}

\begin{proof}
	\textbf{If } Let $p \not\equiv_\text{PR}^R q$. Similarly to the proof of Lemma \ref{lem:pr:pr_well_def}, we use the inductive definition of $R_\kappa \subseteq\, \equiv_\text{PR}^R$ using $f$ and the sets $X_i$ here. Let $m$ be the smallest index at which $(p, q) \notin X_m$. Let $\rho = (p_i, q_i, k_i, x_i)_{0 \leq i \leq |w|}$ be the run of $\mathcal{G}_\text{PR}^{R,\lambda}(p, q)$ on $w$. We prove that $\rho(|w|) = \perp$ and therefore $\rho$ is not accepting by induction on $m$.
	
	If $m = 0$, then $(p, q) \notin Y_\lambda$, meaning that there is a word $w$ such that $\min \{ c(\delta^*(p, u)) \mid u \sqsubset w \} \neq \min \{ c(\delta^*(q, u)) \mid u \sqsubset w \}$. Without loss of generality, assume $\min \{ c(\delta^*(p, u)) \mid u \sqsubset w \} < \min \{ c(\delta^*(q, u)) \mid u \sqsubset w \}$. By Lemma \ref{}, $x_{|w|-1} =\, <$. Furthermore, $\delta(p_{|w|-1}, w_{|w|-1}) \in \lambda$, as $w \in L_{\lambda \rightarrow \lambda}$. Thus, $\rho(|w|) =\, \perp$ and the run is rejecting.
	
	Now consider $m+1 > 1$. Since $(p, q) \in X_m \setminus f(X_m)$, there must be a word $w \in L_{\lambda \rightarrow \lambda}$ such that $(p', q') \notin X_m$, where $p' = \delta^*(p, w)$ and $q' = \delta^*(q, w)$. As $R_\kappa \subseteq X_m$, $(p', q') \notin R_\kappa$ and therefore $p' \not\equiv_\text{PR}^R q'$. By induction, $w \notin L(\mathcal{G}_\text{PR}^{R,\lambda}(p', q'))$; since that run is a suffix of $\rho$, $\rho$ itself is also a rejecting run.
	
	\paragraph{Only If} Let $L(\mathcal{G}_\text{PR}^{R,\lambda}(p, q)) \neq \Sigma^*$. Since $\varepsilon$ is always accepted, there is a word $w \in \Sigma^+ \setminus L(\mathcal{G}_\text{PR}^{R,\lambda}(p, q))$, meaning that $\delta_\text{PR}^*(q_{0,\text{PR}}, w) = \perp$. Split $w$ into sub-words $w = u_1 \cdots u_m$ such that $u_1, \dots, u_m \in L_{\lambda \rightarrow \lambda}$. Note that this partition is unique. We show $p \not\equiv_\text{PR}^R q$ by induction on $m$. Let $\rho = (p_i, q_i, k_i, x_i)_{0 \leq i < |w|}$ be the run of $\mathcal{G}_\text{PR}^{R,\lambda}(p, q)$ on $w$.
	
	If $m = 1$, then $w \in L_{\lambda \rightarrow \lambda}$. Since $\rho(|w|) =\, \perp$, it must be true that $x_{|w|-1} \neq\, =$. Without loss of generality, assume $x_{|w|-1} =\, <$. By Lemma \ref{}, $\min \{ c(\delta^*(p, u)) \mid u \sqsubset w \} < \min \{ c(\delta^*(q, u)) \mid u \sqsubset w \}$. Therefore, $p \not\equiv_\text{PR}^R q$.
	
	Now consider $m+1 > 1$. Let $p' = \delta^*(p, u_1)$ and $q' = \delta^*(q, u_1)$. By induction on the word $u_2 \cdots u_m$, $p' \not\equiv_\text{PR}^R q'$. Since $u_1 \in L_{\lambda \rightarrow \lambda}$, that also means $p \not\equiv_\text{PR}^R q$.
\end{proof}

The differences between different $\mathcal{G}_\text{PR}^{R,\lambda}$ for different $\lambda$ are minor and the question whether the accepted language is universal boils down to a simple question of reachability. Thus, $\equiv_\text{PR}^R$ can be computed in $\mathcal{O}(|\mathcal{G}_\text{PR}^{R,\lambda}|)$ which is $\mathcal{O}(|Q|^2 \cdot |c(Q)|)$.
























\section{Threshold Moore}
\begin{defn}
	Let $\mathcal{A} = (Q, \Sigma, \delta, c)$ be a DPA. For a set $K \subseteq \mathbb{N}$, we define the \emph{priority threshold} of $\mathcal{A}$ as $\mathcal{T}(\mathcal{A}, K) := (Q, \Sigma, \delta, c')$ with $c'(q) = \begin{cases} c(q) & \text{if } c(q) \notin K \\ \max c(Q) + 1 & \text{else} \end{cases}$.
	
	For a relation $\sim$, we define $\mathcal{T}(\sim, K) :=\, \approx$ as $(\mathcal{A}, p) \approx (\mathcal{B}, q)$ iff $(\mathcal{T}(\mathcal{A}, K), p) \sim (\mathcal{T}(\mathcal{B}, K), q)$.
\end{defn}

\begin{lem}
	If $\sim$ is an equivalence relation, then so is $\mathcal{T}(\sim, K)$. If $\sim$ is a congruence relation, then so is $\mathcal{T}(\sim, K)$.
\end{lem}

\begin{proof}
	%TODO
\end{proof}

\begin{defn}
	We set $\equiv_M^{\leq k} \,:= \mathcal{T}(\equiv_M, \{ n \in \mathbb{N} \mid n > k \})$.
\end{defn}

\vspace{5pt}

\begin{defn}
	Let $\mathcal{A}$ and $\mathcal{B}$ be DPAs and let $\sim$ be an equivalence relation that implies language equivalence. We define a relation $\equiv_\text{TM}^\sim$ such that $(\mathcal{A}, p) \equiv^\sim_\text{TM} (\mathcal{B}, q)$ if and only if all of the following are satisfied:
	\begin{enumerate}
		\item $c_1(p) = c_2(q)$
		\item $(\mathcal{A}, p) \equiv_M^{\leq c(p)} (\mathcal{B}, q)$
		\item $(\mathcal{A}, p) \sim (\mathcal{B}, q)$
	\end{enumerate}
\end{defn}


\begin{defn}
	Let $\sim$ be an equivalence relation. We write $\sim^{\leq k} := \mathcal{T}(\sim, \{n \in \mathbb{N} \mid n > k\})$. We call $\sim$ \emph{TM-suitable} if it satisfies the following: 
	
	Let $\mathcal{A}$ be a DPA and let $\lambda \subseteq \kappa \in \mathfrak{C}(\sim)$ such that all states in $\lambda$ share the same priority $k$. For all representative merges $\mathcal{A}'$ of $\mathcal{A}$ w.r.t. $\lambda$, for all $q \in Q'$ with $c(q) \leq k$, $(\mathcal{A}, q) \sim^{\leq c(q)} (\mathcal{A}', q)$.
	
	Furthermore, for all representative merges $\mathcal{A}''$ of $\mathcal{T}(\mathcal{A}, \{n \in \mathbb{N} \mid n > k\})$ w.r.t. $\lambda$, for all words $\alpha \in \Sigma^\omega$ and all runs $\rho$ and $\rho''$ of $\mathcal{A}$ and $\mathcal{A}''$ on that word starting in the same state such that $\min \text{Inf}(c(\rho)) \leq k$, $\rho$ and $\rho''$ have the same acceptance. %TODO
\end{defn}





\vspace{20pt}

\begin{lem}
	$\equiv_M$ is TM-suitable.
\end{lem}

\begin{proof}
	Let $\mathcal{A}$ be a DPA and let $\lambda \subseteq \kappa \in \mathfrak{C}(\sim)$ such that all states in $\lambda$ share the same priority $k$.
	
	For the first part, let $\mathcal{A}'$ of $\mathcal{A}$ w.r.t. $\lambda$ and $q \in Q'$ with $c(q) \leq k$. Let $\rho$ and $\rho'$ be the runs of $\mathcal{A}$ and $\mathcal{A}'$ on $\alpha \in \Sigma^\omega$ starting from $q$. By Lemma \ref{lem:general:cong_stays_in_merge}, $(\mathcal{A}, \rho(i)) \equiv_M^{\leq k} (\mathcal{A}, \rho'(i))$ for all $i$ which especially means that for all $i$, $c(\rho(i)) =^{\leq k} c(\rho'(i))$. Since $c(\rho'(i)) = c'(\rho'(i))$, that also implies $c(\rho(i)) =^{\leq k} c'(\rho'(i))$ which means that $(\mathcal{A}, q) \equiv_M^{\leq k} (\mathcal{A}', q)$.
	
	For the second part, let $\mathcal{A}''$ be a representativ merge of $\mathcal{T}(\mathcal{A}, \{n \in \mathbb{N} \mid n > k\})$ w.r.t. $\lambda$. Let $\alpha \in \Sigma^\omega$ with runs $\rho$ and $\rho''$ of the two automata on $\alpha$ starting in some $q \in Q''$ such that $\min \text{Inf}(c(\rho)) \leq k$. Again by Lemma \ref{lem:general:cong_stays_in_merge}, $(\mathcal{A}, \rho(i)) \equiv_M^{\leq k} (\mathcal{A}, \rho'(i))$ for all $i$. In particular, whenever $\min \text{Inf}(c(\rho))$ or $\min \text{Inf}(c''(\rho''))$ is seen, the same is true for the other run. That is why $\min \text{Inf}(c(\rho)) = \min \text{Inf}(c''(\rho''))$ and the two runs have the same acceptance.
\end{proof}


\vspace{25pt}

\begin{defn}
	Let $\mathcal{A}$ and $\mathcal{B}$ be DPAs. Let $\sim$ be an equivalence relation that implies language equivalence and let $\approx$ be an equivalence relation that is TM suitable. We define a relation $\equiv_\text{TM}^{\sim, \approx}$ such that $(\mathcal{A}, p) \equiv^{\sim, \approx}_\text{TM} (\mathcal{B}, q)$ if and only if all of the following are satisfied:
	\begin{enumerate}
		\item $c_1(p) = c_2(q)$
		\item $(\mathcal{A}, p) \approx^{\leq c(p)} (\mathcal{B}, q)$
		\item $(\mathcal{A}, p) \sim (\mathcal{B}, q)$
	\end{enumerate}
\end{defn}


We can notice that merging classes of $\equiv_\text{TM}^{\sim, \approx}$ in a specific order is a valid language-preserving operation. This is then used to define a fitting merger function. For this next part, let $\mathcal{A}$ be a DPA and let $\kappa \in \mathfrak{C}(\equiv^{\sim, \approx}_\text{TM})$ be a set of states in $\mathcal{A}$. By definition of $\equiv^{\sim, \approx}_\text{TM}$, all states in this class have a unique priority $k$. Let $\mathcal{A}'$ be a representative merge of $\mathcal{A}$ w.r.t. $\kappa$.

\begin{lem}
	For all $q \in Q'$, $(\mathcal{A}, q) \sim (\mathcal{A}', q)$.
	\label{lem:tremoore:merge_keep_sim}
\end{lem}

\begin{proof} 
	Let $\rho$ and $\rho'$ be the runs of $\mathcal{A}$ and $\mathcal{A}'$ on $\alpha \in \Sigma^\omega$ starting from $q$. We claim that $\rho$ is accepting iff $\rho'$ is accepting.
	
	By Lemma \ref{lem:general:cong_stays_in_merge}, $(\mathcal{A}, \rho(i)) \equiv_L (\mathcal{A}, \rho'(i))$ and $(\mathcal{A}, \rho(i)) \approx^{\leq k} (\mathcal{A}, \rho'(i))$ for all $i$. If $c(\rho)$ sees infinitely many priorities of at most $k$, then %TODO
	
	 Now there are two cases: if $c(\rho)$ sees infinitely many priorities of at most $k$, then $c(\rho')$ sees the same priorities at the same positions and thus $\min \text{Inf}(c(\rho)) = \min \text{Inf}(c(\rho'))$. By definition of a representative merge, $c(\rho') = c'(\rho')$.
	
	 Otherwise there is a position $n$ from which $c(\rho)$ only is greater than $k$ and therefore the same is true for $c(\rho')$. That means reading $\alpha[n,\omega]$ from $\rho'(n)$ in either $\mathcal{A}$ or $\mathcal{A}'$ yields the same run which has the same acceptance as $\rho$.
\end{proof}

\begin{lem}
	For all $q \in Q'$ with $c(q) \leq k$, $(\mathcal{A}, q) \approx^{\leq k} (\mathcal{A}', q)$.
	\label{lem:tremoore:merge_keep_tmoore}
\end{lem}

\begin{proof} 
	Let $\rho$ and $\rho'$ be the runs of $\mathcal{A}$ and $\mathcal{A}'$ on $\alpha \in \Sigma^\omega$ starting from $q$. We claim that $\rho$ is accepting iff $\rho'$ is accepting.
	
	By Lemma \ref{lem:general:cong_stays_in_merge}, $(\mathcal{A}, \rho(i)) \equiv_M^{\leq k} (\mathcal{A}, \rho'(i))$ for all $i$ which especially means that for all $i$, $c(\rho(i)) =^{\leq k} c(\rho'(i))$. Since $c(\rho'(i)) = c'(\rho'(i))$, that also implies $c(\rho(i)) =^{\leq k} c'(\rho'(i))$ which means that $(\mathcal{A}, q) \equiv_M^{\leq k} (\mathcal{A}', q)$.
\end{proof}

\begin{lem}
	Let $L \subseteq K \subseteq \mathbb{N}$. Then $\mathcal{T}(\equiv_M, K) \subseteq \mathcal{T}(\equiv_M, L)$.
	\label{lem:tremoore:moore_less_thresh_is_subset}
\end{lem}

\begin{proof}
	%TODO
\end{proof}

\begin{lem}
	For all $q \in Q'$ with $c(q) \leq k$, $(\mathcal{A}, q) \equiv^\sim_\text{TM} (\mathcal{A}', q)$.
	\label{lem:tremoore:merge_keep_tm}
\end{lem}

\begin{proof}
	Representative merges never change priorities assigned to states. Together with Lemma \ref{lem:tremoore:merge_keep_sim}, Lemma \ref{lem:tremoore:merge_keep_tmoore}, and Lemma \ref{lem:tremoore:moore_less_thresh_is_subset} this already finishes the proof.
\end{proof}

\begin{lem}
	For all $p, q \in Q'$ with $\min \{c(p), c(q)\} \leq k$, $(\mathcal{A}, p) \equiv^\sim_\text{TM} (\mathcal{A}, q)$ iff \linebreak $(\mathcal{A}', p) \equiv^\sim_\text{TM} (\mathcal{A}', q)$.
	\label{lem:tremoore:merge_changes_only_higher}
\end{lem}

\begin{proof}
	If $c(p) = c(q) \leq k$: By Lemma \ref{lem:tremoore:merge_keep_tm}, we know that $(\mathcal{A}, p) \equiv^\sim_\text{TM} (\mathcal{A}', p)$ and $(\mathcal{A}, q) \equiv^\sim_\text{TM} (\mathcal{A}', q)$. If now $(\mathcal{A}, p) \equiv^\sim_\text{TM} (\mathcal{A}, q)$ holds, then also $(\mathcal{A}', p) \equiv^\sim_\text{TM} (\mathcal{A}', q)$ and vice versa.
	
	If $c(p) \neq c(q)$, then also $c'(p) \neq c'(q)$ and we have both $(\mathcal{A}, p) \not\equiv^\sim_\text{TM} (\mathcal{A}, q)$ and $(\mathcal{A}', p) \not\equiv^\sim_\text{TM} (\mathcal{A}', q)$.
\end{proof}

\newpage



\vspace{25pt}

We can notice that merging classes of $\equiv_\text{TM}^\sim$ in a specific order is a valid language-preserving operation. This is then used to define a fitting merger function. For this next part, let $\mathcal{A}$ be a DPA and let $\kappa \in \mathfrak{C}(\equiv^\sim_\text{TM})$ be a set of states in $\mathcal{A}$. By definition of $\equiv^\sim_\text{TM}$, all states in this class have a unique priority $k$. Let $\mathcal{A}'$ be a representative merge of $\mathcal{A}$ w.r.t. $\kappa$.

\begin{lem}
	For all $q \in Q'$, $(\mathcal{A}, q) \sim (\mathcal{A}', q)$.
	\label{lem:tremoore:merge_keep_sim}
\end{lem}

\begin{proof} 
	Let $\rho$ and $\rho'$ be the runs of $\mathcal{A}$ and $\mathcal{A}'$ on $\alpha \in \Sigma^\omega$ starting from $q$. We claim that $\rho$ is accepting iff $\rho'$ is accepting.
	
	By Lemma \ref{lem:general:cong_stays_in_merge}, $(\mathcal{A}, \rho(i)) \equiv_L (\mathcal{A}, \rho'(i))$ and $(\mathcal{A}, \rho(i)) \equiv_M^{\leq k} (\mathcal{A}, \rho'(i))$ for all $i$. Now there are two cases: if $c(\rho)$ sees infinitely many priorities of at most $k$, then $c(\rho')$ sees the same priorities at the same positions and thus $\min \text{Inf}(c(\rho)) = \min \text{Inf}(c(\rho'))$. By definition of a representative merge, $c(\rho') = c'(\rho')$.
	
	 Otherwise there is a position $n$ from which $c(\rho)$ only is greater than $k$ and therefore the same is true for $c(\rho')$. That means reading $\alpha[n,\omega]$ from $\rho'(n)$ in either $\mathcal{A}$ or $\mathcal{A}'$ yields the same run which has the same acceptance as $\rho$.
\end{proof}

\begin{lem}
	For all $q \in Q'$ with $c(q) \leq k$, $(\mathcal{A}, q) \equiv_M^{\leq k} (\mathcal{A}', q)$.
	\label{lem:tremoore:merge_keep_tmoore}
\end{lem}

\begin{proof} 
	Let $\rho$ and $\rho'$ be the runs of $\mathcal{A}$ and $\mathcal{A}'$ on $\alpha \in \Sigma^\omega$ starting from $q$. We claim that $\rho$ is accepting iff $\rho'$ is accepting.
	
	By Lemma \ref{lem:general:cong_stays_in_merge}, $(\mathcal{A}, \rho(i)) \equiv_M^{\leq k} (\mathcal{A}, \rho'(i))$ for all $i$ which especially means that for all $i$, $c(\rho(i)) =^{\leq k} c(\rho'(i))$. Since $c(\rho'(i)) = c'(\rho'(i))$, that also implies $c(\rho(i)) =^{\leq k} c'(\rho'(i))$ which means that $(\mathcal{A}, q) \equiv_M^{\leq k} (\mathcal{A}', q)$.
\end{proof}

\begin{lem}
	Let $L \subseteq K \subseteq \mathbb{N}$. Then $\mathcal{T}(\equiv_M, K) \subseteq \mathcal{T}(\equiv_M, L)$.
	\label{lem:tremoore:moore_less_thresh_is_subset}
\end{lem}

\begin{proof}
	%TODO
\end{proof}

\begin{lem}
	For all $q \in Q'$ with $c(q) \leq k$, $(\mathcal{A}, q) \equiv^\sim_\text{TM} (\mathcal{A}', q)$.
	\label{lem:tremoore:merge_keep_tm}
\end{lem}

\begin{proof}
	Representative merges never change priorities assigned to states. Together with Lemma \ref{lem:tremoore:merge_keep_sim}, Lemma \ref{lem:tremoore:merge_keep_tmoore}, and Lemma \ref{lem:tremoore:moore_less_thresh_is_subset} this already finishes the proof.
\end{proof}

\begin{lem}
	For all $p, q \in Q'$ with $\min \{c(p), c(q)\} \leq k$, $(\mathcal{A}, p) \equiv^\sim_\text{TM} (\mathcal{A}, q)$ iff \linebreak $(\mathcal{A}', p) \equiv^\sim_\text{TM} (\mathcal{A}', q)$.
	\label{lem:tremoore:merge_changes_only_higher}
\end{lem}

\begin{proof}
	If $c(p) = c(q) \leq k$: By Lemma \ref{lem:tremoore:merge_keep_tm}, we know that $(\mathcal{A}, p) \equiv^\sim_\text{TM} (\mathcal{A}', p)$ and $(\mathcal{A}, q) \equiv^\sim_\text{TM} (\mathcal{A}', q)$. If now $(\mathcal{A}, p) \equiv^\sim_\text{TM} (\mathcal{A}, q)$ holds, then also $(\mathcal{A}', p) \equiv^\sim_\text{TM} (\mathcal{A}', q)$ and vice versa.
	
	If $c(p) \neq c(q)$, then also $c'(p) \neq c'(q)$ and we have both $(\mathcal{A}, p) \not\equiv^\sim_\text{TM} (\mathcal{A}, q)$ and $(\mathcal{A}', p) \not\equiv^\sim_\text{TM} (\mathcal{A}', q)$.
\end{proof}

\vspace{10pt}

\begin{defn}
	Let $\mathcal{A}$ be a DPA and let $\sim$ be an equivalence relation that implies language equivalence. We define the \emph{Threshold Moore merger function} $\mu_\text{TM}^\sim := \mu_\div^{\equiv^\sim_\text{TM}}$.
\end{defn}

\begin{theorem}
	Let $\mathcal{A}$ be a DPA and let $\sim$ be an equivalence relation that implies language equivalence. Every representative merge of $\mathcal{A}$ w.r.t. $\mu_\text{TM}^\sim$ is language equivalent to $\mathcal{A}$.
\end{theorem}

\begin{proof}
	Let $\kappa_1, \dots, \kappa_m$ be an enumeration of $\mathfrak{C}(\equiv^\sim_\text{TM})$ such that the (unique) priority of states in $\kappa_i$ is $k_i$ and $k_1, \dots, k_m$ is a descending series.
	
	If we build a representative merge of $\mathcal{A}$ w.r.t. $\kappa_1$ by candidates $_1$, we obtain some DPA $\mathcal{A}_1$. By Lemma \ref{lem:tremoore:merge_changes_only_higher}, for all $i > 1$, the states in $\kappa_i$ are still pairwise $\equiv^\sim_\text{TM}$ equivalent. Moreover, $\kappa_i$ is an $\equiv^\sim_\text{TM}$ equivalence class in $\mathcal{A}_1$.
	
	That means we can continue building representative merges in order of the enumeration and our previous results apply. In the end, we obtain a DPA $\mathcal{A}'$ that is language equivalent to $\mathcal{A}$ by Lemma \ref{lem:tremoore:merge_keep_sim}.
\end{proof}

\newpage
























\section{Labeled SCC Filter}
\begin{defn}
	Let $\mathcal{A} = (Q, \Sigma, q_0, \delta, c)$ be a DPA. We define $\mathcal{A}\upharpoonright^c_{= k} := \mathcal{A}\upharpoonright_P$ with $P = \{q \in Q \mid c(q) = k\}$. Analogously, we define $\mathcal{A}\upharpoonright^c_{> k}$.
	
	We define a relation $R_k \subseteq Q \times Q$ such that $(p, q) \in R_k$ if and only if all of the following are true:
	\begin{enumerate}
		\item $\min \{c(p), c(q)\} > k$
		\item $p \equiv_L q$
		\item $p \equiv_M^{\leq k} q$
		\item In $\mathcal{A}\upharpoonright^c_{> k}$, $p$ and $q$ lie in different SCCs.
	\end{enumerate}
	
	We define $\equiv_\text{LSF}^k \,\subseteq Q \times Q$ to be the reflexive and transitive closure of $R_k$.
\end{defn}

\begin{lem}
	$\equiv_\text{LSF}^k$ is an equivalence relation.
\end{lem}

\begin{defn}
	Let $\mathcal{A}$ be a DPA and $k \in \mathbb{N}$. We define $\preceq_k \subseteq Q \times Q$ to be a total extension of the reachability preorder in $\mathcal{A}\upharpoonright^c_{\geq k}$.
	
	Let $\lambda$ be an equivalence class of $\equiv_\text{LSF}^k$. Let $r \in \lambda$ be a representative of $\lambda$ that is $\preceq_k$-maximal. We set $\lambda' := \{q \in \lambda \mid q \prec_k r\} \cup \{r\}$. We call an automaton $\mathcal{A}'$ a \emph{$\text{LSF}_\lambda^k$-merge} of $\mathcal{A}$ if it is a representative merge of $\mathcal{A}$ w.r.t. $\lambda'$ that uses the representative $r_{\lambda'} = r$.
\end{defn}

\begin{theorem}
	Let $\mathcal{A}$ be a DPA and let $\mathcal{A}'$ be a $\text{LSF}_\lambda^k$-merge of $\mathcal{A}$. Then $L(\mathcal{A}) = L(\mathcal{A}')$.
	\label{thm:lsf:correctness}
\end{theorem}

\begin{proof}
	Let $r_\lambda$ be the representative that is used in the construction of $\mathcal{A}'$. Let $q \in Q'$ be a state in the representative merge and let $\alpha \in \Sigma^\omega$. Let $\rho$ and $\rho'$ be the runs of $\mathcal{A}$ and $\mathcal{A}'$ on $\alpha$ starting from $q$. We claim that $\rho$ is accepting iff $\rho'$ is accepting.
	
	By Lemma \ref{lem:tremoore:cong_stays_in_merge}, we know that $\rho(i) \equiv_L \rho'(i)$ and $\rho(i) \equiv_M^{\leq k} \rho'(i)$ for all $i$. If there is a position $n$ from which on $\rho'[n,\omega]$ is both a valid run in $\mathcal{A}$ and $\mathcal{A}'$, then we know that $\rho$ is accepting if and only if $\rho'$ is accepting since $\rho(n) \equiv_L \rho'(n)$.
	
	If $\rho'$ visits infinitely many states with priority equal to or less than $k$, then $\rho$ and $\rho'$ share the same minimal priority that is visited infinitely often and thus have the same acceptance.
	
	For the last case, assume that $\rho'$ uses infinitely many redirected edges but from some point $n_1$ on stays in $\mathcal{A}\upharpoonright^c_{>k}$. Let $n_3 > n_2 > n_1$ be the next two positions at which $\rho'$ uses a redirected edge, i.e. $\delta(\rho'(n_2), \alpha(n_2)) \neq \delta'(\rho'(n_2), \alpha(n_2))$ and analogous for $n_3$. Note that $\delta'(\rho'(n_2), \alpha(n_2)) = \delta'(\rho'(n_3), \alpha(n_3)) = r_\lambda$, since all redirected transition target the representative state. Let we call $\delta(\rho'(n_3), \alpha(n_3)) = q$. Since between $n_2$ and $n_3$ no redirected transition is taken, $\rho'[n_2, n_3]$ is a valid path in $\mathcal{A}$, so we have $r_\lambda \preceq_k q$ by choice of $n_1$. The fact that transitions to $q$ are redirected to $r_\lambda$ however requires that $q \prec_k r_\lambda$, which would be a contradiction.
\end{proof}


\begin{lem}
	Let $\mathcal{A}$ be a DPA and let $\mathcal{A}'$ be a $\text{LSF}_\lambda^k$-merge of $\mathcal{A}$. Let $\equiv_\text{LSF}^l$ be the LSF-relation in $\mathcal{A}$ and let $\equiv_\text{LSF'}^l$ be the LSF-relation in $\mathcal{A}'$. If $l \leq k$, then $\equiv_\text{LSF}^l \upharpoonright_{Q' \times Q'} \,\supseteq\, \equiv_\text{LSF'}^l$.
	\label{lem:lsf:constr_does_not_change_lower_k}
\end{lem}

\begin{proof}
	Let $R_l$ and $R'_l$ be the relations used in the definition of $\equiv_\text{LSF}^l$ and $\equiv_\text{LSF'}^l$. We prove $R'_l \subseteq R'_l \upharpoonright_{Q' \times Q'}$. If that is true, then so is the statement of our Lemma. We do so by considering the four properties of $R_l$ individually.
	
	The first point is clear; $c' = c \upharpoonright_{Q'}$, so $c'(p) = c(p)$ and $c'(q) = c(q)$.
	
	For the second point, consider Theorem \ref{thm:lsf:correctness}. By making $p$ or $q$ the initial state of our DPA, we observe that neither state has its language changed by the construction, so they must still be equal.
	
	For the third point, let $\equiv_M^{\unlhd l}$ be the $l$-threshold Moore equivalence in $\mathcal{A}'$. Let $w \in \Sigma^*$ be an arbitrary word, $p \equiv_M^{\leq l}$, $p' := (\delta')^*(p, w)$, and $q' := (\delta')^*(q, w)$. Using Lemma \ref{lem:tremoore:cong_stays_in_merge}, we know that, since $p \equiv_M^{\leq l} q$, also $p' \equiv_M^{\leq l} q'$. In particular, this means $c(p') =^{\leq l} c(q')$. As $w$ was chosen to be arbitrary, that means $p \equiv_M^{\unlhd l} q$.
	
	Lastly, for the fourth point, assume that there are states $p, q$ which lie in different SCCs in $\mathcal{A} \upharpoonright^c_{> l}$ but not in $\mathcal{A}' \upharpoonright^c_{> l}$. Without loss of generality, we assume that in $\mathcal{A} \upharpoonright^c_{> l}$, $p$ is not reachable from $q$. In $\mathcal{A}' \upharpoonright^c_{> l}$ however, this is possible, so let $\rho'$ be a path from $q$ to $p$. We can assume $\rho'$ to pay exactly one visit to $\lambda$; there has to be at least one visit, as otherwise the path would also be available in $\mathcal{A}$; if there would be multiple visits, all of them would end at $r_\lambda$, so we could cut those parts from the run. Let $uv$ be words that induce that run, i.e. $\delta^*(q, u) \in \lambda$ and $\delta^*(r_\lambda, v) = p$.
	
	We distinguish two cases. In the first case, $q$ is reachable from $p$ in $\mathcal{A}$ by some word $w$. Here, consider reading the word $vwu$ from $r_\lambda$ in $\mathcal{A}$. The run moves to $p$ by $v$, then to $q$ by $w$, then to $\delta^*(q, u) \in \lambda$. $\delta^*(q, u)$ was the state from which the redirected transition was taken in $\rho'$, so it cannot be reachable from $r_\lambda$ by definition of the merge. This is a contradiction.
	
	For the second case, $q$ is not reachable from $p$ in $\mathcal{A}$. Since the two states lie in a common SCC in $\mathcal{A}'$ however, there is a path $\pi'$ from $p$ to $q$. With the same argument as before, we can assume that $\pi'$ leads to $r_\lambda$ via some word $u'$ and from there to $q$ via some $v'$. As in the first case, the word $v' u$ gives us a path from $r_\lambda$ to $\delta^*(q, u)$ which is a contradiction.
\end{proof}

The two previous statements provide us with a possible algorithm to perform state space reduction with the LSF method. Starting at $k = \min c(Q) - 1$ and iterating up to $\max c(Q)$, compute $\equiv_\text{LSF}^k$ and build representative merges of each equivalence class. By Lemma \ref{lem:lsf:constr_does_not_change_lower_k}, strictly iterating once through all $k$ in ascending order gives us all possible merges.

The final question that remains is how to compute $\equiv_\text{LSF}^k$ itself. This can be done rather easily 







\section{Schewe}
\label{sect:schewe}

This section is based heavily on \cite{Schewe2010} and partially adapts their results from B\"uchi to parity automata.

\begin{defn}
		Let $\mathcal{A} = (Q, \Sigma, \delta, c)$ be a DPA and let $\emptyset \neq C \subseteq M \subseteq Q$. Let $\mathcal{A}' = (Q', \Sigma, \delta', c')$ be another DPA. We call $\mathcal{A}'$ a \emph{Schewe merge of $\mathcal{A}$ w.r.t. $M$ by candidates $C$} if it satisfies the following:
	\begin{itemize}
		\item There is a state $r_M \in C$ such that $Q' = (Q \setminus M) \cup \{r_M\}$.
		\item $c' = c\upharpoonright_{Q'}$.
		\item Let $p \in Q'$ and $\delta(p, a) = q$. If $q \in M$ or if ($q \in C$ and $p$ is not reachable from $q$), then $\delta'(p, a) = r_M$. Otherwise, $\delta'(p, a) = q$. 
	\end{itemize}
\end{defn}

The definition of a Schewe merge is almost identical to that of a representative merge. The only difference lies therein that some additional transitions are redirected to the representative: when a transition leads to a candidate that is not in $M$ while also moving to a different SCC.

Using the Schewe merge instead of the representative merge does not actually remove any additional states from the automaton, it only provides a better \enquote{framework} for following algorithms such as the Moore reduction. Example figure \ref{fig:schewe:example_useful1} shows that using a Schewe merge followed by another merge of $\mu_M$ creates a better end result.

The automaton has six states. Regarding Moore equivalence, every state builds a singleton class. For priority almost equivalence, the classes are $\{q_0, q_1\}$, $\{q_2, q_4\}$, and $\{q_3\}$. As all states within one equivalence class lie in the same SCC, a representative merge w.r.t. $\mu_\text{skip}^{\equiv_\text{\Ankh}}$ will not change the number of states. However, a Schewe merge w.r.t. $\mu_\text{skip}^{\equiv_\text{\Ankh}}$ can result in figure \ref{fig:schewe:example_useful2}. Now, $q_0$ and $q_1$ are actually Moore equivalent and could be merged with $\mu_M$.

\begin{figure}
\centering
\begin{tikzpicture}[shorten >=1pt,node distance=2cm,on grid,initial text=]
  \node[state]           (0)                {$q_0,1$};
  \node[state]           (1) [right=of 0]   {$q_1,1$};
  \node[state]           (2) [above=of 0]   {$q_2,1$};
  \node[state]           (3) [right=of 2]   {$q_3,1$};
  \node[state]           (4) [right=of 3]   {$q_4,0$};
  \path[->] (0) edge [bend left] node [above] {a} (1)
  			(0) edge node [left] {b} (2)
            (1) edge [bend left] node [below] {a} (0)
            (1) edge node [below] {b} (4)
            (2) edge [bend left] node [above] {a,b} (3)
            (3) edge [bend left] node [below] {a} (2)
            (3) edge [bend left] node [above] {b} (4)
            (4) edge [bend left] node [below] {a,b} (3);
\end{tikzpicture}
\caption{Example to show the effect of a Schewe merge.}
\label{fig:schewe:example_useful1}
\end{figure}

\begin{figure}
\centering
\begin{tikzpicture}[shorten >=1pt,node distance=2cm,on grid,initial text=]
  \node[state]           (0)                {$q_0,1$};
  \node[state]           (1) [right=of 0]   {$q_1,1$};
  \node[state]           (2) [above=of 0]   {$q_2,1$};
  \node[state]           (3) [right=of 2]   {$q_3,1$};
  \node[state]           (4) [right=of 3]   {$q_4,0$};
  \path[->] (0) edge [bend left] node [above] {a} (1)
  			(0) edge node [left] {b} (2)
            (1) edge [bend left] node [below] {a} (0)
            (1) edge node [right] {b} (2)
            (2) edge [bend left] node [above] {a,b} (3)
            (3) edge [bend left] node [below] {a} (2)
            (3) edge [bend left] node [above] {b} (4)
            (4) edge [bend left] node [below] {a,b} (3);
\end{tikzpicture}
\caption{Automaton from figure \ref{fig:schewe:example_useful1} after Schewe merge.}
\label{fig:schewe:example_useful2}
\end{figure}

\begin{defn}
	Let $\mathcal{A}$ be DPA with a merger function $\mu : D \rightarrow 2^Q$. For a representative merge $\mathcal{A}'$, we define the \emph{candidate relation} $\sim_\mathcal{C}^\mu$ as $p \sim_\mathcal{C}^\mu q$ iff there is a $C \in \mu(D)$ with $p, q \in C$.
	
	We say that $\mu$ is \emph{Schewe suitable} if for all representative merges $\mathcal{A}'$, $\sim_\mathcal{C}^\mu$ is a congruence relation, it implies language equivalence, and the reachability order restricted to any $\kappa \in \mathfrak{C}(\sim_\mathcal{C}^\mu)$ is an equivalence relation.
\end{defn}


\begin{lem}
	Let $\mathcal{A}$ be a DPA and $\mu : D \rightarrow 2^Q$ be a merger function that is Schewe suitable. Let $\mathcal{A}'$ be a representative merge of $\mathcal{A}$ w.r.t. $\mu$ and let $\mathcal{A}''$ be the Schewe merge that uses the same choice of representatives. For all $p \sim_\mathcal{C}^\mu q$, $\delta'(p, a) \sim_\mathcal{C}^\mu \delta''(q, a)$.
	\label{lem:schewe:different_merges_still_congruent}
\end{lem}

\begin{proof}
	If $\delta''(q, a) = \delta'(q, a)$, then $\delta'(p, a) \sim_\mathcal{C}^\mu \delta'(q, a)$ because $\mu$ is Schewe suitable and the claim is true.
	
	Otherwise $\delta'(q, a) = q' \in \mu(M)$ for some $M$ and $q$ is not reachable from $q'$. Then $\delta''(q, a) = r_M$. By definition, $r_M \in \mu(M)$ as well. By definition of $\sim_\mathcal{C}^\mu$, $r_m \sim_\mathcal{C}^\mu q'$ and thus $\delta'(p, a) \sim_\mathcal{C}^\mu \delta'(q, a) \sim_\mathcal{C}^\mu \delta''(q, a)$.
\end{proof}


\begin{lem}
	Let $\mathcal{A}$ be a DPA and $\mu : D \rightarrow 2^Q$ be a merger function that is Schewe suitable. Let $\mathcal{A}'$ be a representative merge of $\mathcal{A}$ w.r.t. $\mu$ and let $\mathcal{A}''$ be the Schewe merge that uses the same choice of representatives. Every run of $\mathcal{A}''$ has a suffix that is a run of $\mathcal{A}'$.
	\label{lem:schewe:merge_has_suffix_run}
\end{lem}

\begin{proof}
	Let $K \subseteq \mathbb{N}$ be the set of positions at which $\mathcal{A}''$ uses a transition that is not in $\mathcal{A}'$. That is, given a run $\rho$ on $\alpha$, $\rho(k+1) \neq \delta'(\rho(k) \alpha(k))$ for all $k \in K$. If we can prove that $K$ is finite, then that means $\rho[\max K + 1, \omega]$ is a run in $\mathcal{A}'$.
	
	More precisely, we show that for every $\kappa \in \mathfrak{C}(\sim_\mathcal{C}^\mu)$, there is at most one $k \in K$ such that $\rho(k+1) \in \kappa$. Towards a contradiction assume the opposite, so there are $k < k'$ which both have this property. We label the positions in $K$ in ascending order and have $k = k_l$ and $k' = k_{l'}$ for some $l < l'$.
	
	Consider $k_{l+1}$. By definition of the Schewe merge, $\delta'(\rho(k_{l+1}), \alpha(k_{l+1})) \not\preceq_\text{reach}^\mathcal{A} \rho(k_{l+1})$. By Lemma \ref{lem:schewe:different_merges_still_congruent}, we know that $(\delta')^*(\rho(k_{l+1}), \alpha[k_{l+1}, k_{l'}+1]) \sim_\mathcal{C}^\mu (\delta'')^*(\rho(k_{l+1}), \alpha[k_{l+1}, k_{l'}+1]) = \rho(k_{l'}+1)$. That means there is a state $r \in \kappa$ that is reachable from $\delta'(\rho(k_{l+1}), \alpha(k_{l+1}))$ in $\mathcal{A}$ and therefore from $\rho(k_l + 1)$ as well.
	
	Reachability order restricted to $\kappa$ is an equivalence relation, so $r$ and $\rho(k_l + 1)$ must lie in the same SCC in $\mathcal{A}'$. That however contradicts the fact that at position $k_{l+1}$ a redirected transition was taken.
\end{proof}


\begin{lem}
	Let $\mathcal{A}$ be a DPA and $\mu : D \rightarrow 2^Q$ be a merger function that is Schewe suitable. Let $\mathcal{A}'$ be a representative merge of $\mathcal{A}$ w.r.t. $\mu$ that was built with representatives $R \subseteq Q$. If a Schewe merge $\mathcal{A}''$ is built with the same representatives, then $(\mathcal{A}', p) \equiv_L (\mathcal{A}'', q)$ for all $q \sim_\mathcal{C}^\mu q$.
	\label{lem:schewe:schewe_suitable_works}
\end{lem}

\begin{proof}
	Let $\alpha \in \Sigma^\omega$ be some word. Let $\rho'$ be the run of $\mathcal{A}'$ on $\alpha$ starting in $p$ and let $\rho''$ be the run of $\mathcal{A}''$ on $\alpha$ starting in $q$. Let $k_1, \dots, k_n$ be the positions at which $\rho''$ uses a transition that is not present in $\mathcal{A}'$, i.e. $\rho''(k_i + 1) \neq \delta'(\rho(k_i), \alpha(k_i))$. By Lemma \ref{lem:schewe:merge_has_suffix_run}, this list must be finite. 
	
	Our goal now is to prove $\rho'(k_i + 1) \sim_\mathcal{C}^\mu \rho''(k_i + 1)$ for all $i$. If that is true, then $\rho'(k_n + 1) \sim_\mathcal{C}^\mu \rho''(k_n + 1)$ in particular is true and therefore $\rho'(k_n + 1) \equiv_L \rho''(k_n + 1)$. By choice of $k_n$, $\rho''[k_n + 1, \omega]$ is also a run in $\mathcal{A}'$ which has the same acceptance as $\rho''$. Since the two states are language equivalent, this is also the same acceptance as $\rho'$.
	
	Assume that for some $i$, $\rho'(k_j + 1) \sim_\mathcal{C}^\mu \rho''(k_j + 1)$ is true for all $j < i$. As $\sim_\mathcal{C}^\mu$ is a congruence relation in $\mathcal{A}'$, $\rho'(k_i) \sim_\mathcal{C}^\mu \rho''(k_i)$ and $\rho'(k_i + 1) = \delta'(\rho'(k_i), \alpha(k_i)) \sim_\mathcal{C}^\mu \delta'(\rho''(k_i), \alpha(k_i))$. By Lemma \ref{lem:schewe:different_merges_still_congruent}, $\delta'(\rho''(k_i), \alpha(k_i)) \sim \delta''(\rho''(k_i), \alpha(k_i)) = \rho''(k_i + 1)$.
\end{proof}


\begin{lem}
	Let $\sim$ be a congruence relation that implies language equivalence. Then $\mu_\text{skip}^\sim$ is Schewe suitable.
	\label{lem:schewe:skip_suitable}
\end{lem}

\begin{proof}
	Let $\mathcal{A}'$ be a representative merge of a DPA $\mathcal{A}$ w.r.t. $\mu_\text{skip}^\sim$. We prove that $p \sim q$ iff $p \sim_\mathcal{C}^{\mu_\text{skip}^\sim} q$. The required properties then follow from the assumptions in the statement and from Lemma \ref{lem:skip:equiv_same_scc}.
	
	We have $\text{dom}(\mu_\text{skip}^\sim) = D = \{M_\kappa \mid \kappa \in \mathfrak{C}(\sim)\}$ and $\mu_\text{skip}^\sim(M_\kappa) = C_\kappa \subseteq \kappa$. Thus, $p \sim_\mathcal{C}^{\mu_\text{skip}^\sim} q$ implies $p \sim q$.
	
	On the other hand, $\kappa = C_\kappa \cup M_\kappa$, so all states of $\kappa$ that remain in $\mathcal{A}'$ are $C_\kappa$ and thus lie in the same equivalence class of $\sim_\mathcal{C}^{\mu_\text{skip}^\sim}.$
\end{proof}


\begin{cor}
	Let $\mathcal{A}$ be a DPA and let $\sim$ be a congruence relation that implies language equivalence. For each Schewe merge $\mathcal{A}'$ of $\mathcal{A}$ w.r.t. $\mu_\text{skip}^\sim$, $\mathcal{A} \equiv_L \mathcal{A}'$.
\end{cor}

\begin{proof}
	Follows from Lemma \ref{lem:schewe:schewe_suitable_works}, Lemma \ref{lem:schewe:skip_suitable}, and Theorem \ref{thm:skip:lang_equiv}.
\end{proof}

\vspace{10pt}

For our final proof in this section we want to adapt a result from \cite{Schewe2010} to show a special relation between Schewe merges and priority almost equivalence.

\begin{lem}
	Let $\mathcal{A}$ be a DPA, $\mathcal{S}$ be a Schewe merge of $\mathcal{A}$ w.r.t. $\mu_\text{skip}^{\equiv_\text{\Ankh}}$, and $\mathcal{S}'$ be a representative merge of $\mathcal{S}$ w.r.t. $\mu_M$. There is no smaller DPA than $\mathcal{S}'$ that is priority almost equivalent to $\mathcal{A}$.
\end{lem} 

\begin{proof}
	Let $\mathcal{B}$ be a DPA that is smaller than $\mathcal{S}'$. Our goal is to show that $\mathcal{A} \not\equiv_\text{\Ankh} \mathcal{B}$.
	
	At first observe that $\mathcal{A} \equiv_\text{\Ankh} \mathcal{S}'$: $\mathcal{A}$ and $\mathcal{S}$ are priority almost equivalent as for every state in $\mathcal{A}$, there is an equivalent representative in $\mathcal{S}$. $\mathcal{S}$ and $\mathcal{S}'$ are Moore equivalent by Lemma \ref{lem:general:congrel_prio_implies_moore} and thus they are also priority almost equivalent by Theorem \ref{thm:general:M_subs_Ankh_subs_L}.

	Assume that $\mathcal{S}' \equiv_\text{\Ankh} \mathcal{B}$ holds, so for all states in $\mathcal{S}'$, there is a a priority almost equivalent state in $\mathcal{B}$. We define a function $f$ that maps to each equivalence class of $\equiv_\text{\Ankh}$ in $\mathcal{S}'$ all states in $\mathcal{B}$ that are equivalent to it, i.e. for $\kappa \in \mathfrak{C}(\equiv_\text{\Ankh}, \mathcal{S'})$ then $f(\kappa) = \{q \in Q_\mathcal{B} \mid \exists p \in \kappa: (\mathcal{S}', p) \equiv_\text{\Ankh} (\mathcal{B}, q) \}$. Note that $f(\kappa)$ can never be empty by the assumption of $\mathcal{S}' \equiv_\text{\Ankh} \mathcal{B}$.
	
	As $\mathcal{B}$ is smaller than $\mathcal{S}'$, the pigeonhole principle applies and we can fix $\kappa$ to be one equivalence class such that $|f(\kappa)| < |\kappa|$.
	
	There is an SCC $C$ in $\mathcal{S}'$ that contains all states in $\kappa$ (argumentation is done similar to Lemma \ref{lem:skip:equiv_same_scc}). Without loss of generality we can assume that likewise there is an SCC $D$ in $\mathcal{B}$ that contains $f(\kappa)$. If no such SCC would exist, we could simply apply the Schewe merger to $\mathcal{B}$ to find an automaton that is smaller than $\mathcal{S}'$ and does have this property.
	
	$C$ and $D$ must be non-trivial SCCs: if $C$ would be trivial, $\kappa$ would contain only one element and $f(\kappa)$ would be empty. If $D$ would be trivial, $f(\kappa) = \{q\}$ would contain of only one state. Since $\mathcal{S}' \equiv_\text{\Ankh} \mathcal{B}$, there is a state $p \in \kappa \subseteq C$ in $\mathcal{S}'$ with $(\mathcal{S}', p) \equiv_\text{\Ankh} (\mathcal{B}, q)$. Since $C$ is not trivial, there is a word $w$ from which $\mathcal{S}'$ moves from $p$ back to $p$. $\mathcal{B}$ however, leaves $f(\kappa)$ with that word, as $D$ is trivial, so $q$ cannot reach itself again. This is a contradiction, as $\equiv_\text{\Ankh}$ is a congruence relation.
	
	\vspace{5pt}
	
	We claim that there is a state $p \in \kappa$ such that there is a family of words $(w_q)_{q \in f(\kappa)}$ such that $\mathcal{S}'$ does not leave $C$ reading these words from $p$ and $c_{\mathcal{S}'}(\delta_{\mathcal{S}'}^*(p, w)) \neq c_{\mathcal{B}}(\delta_{\mathcal{B}}^*(q, w))$. (in other words, $w_q$ is a witness for $p$ and $q$ being not Moore-equivalent.)
	
	Towards a contradiction, assume that the claim is false and that for every $p \in \kappa$, there is a state $q_p \in f(\kappa)$ that does not satisfy the property. As $|\kappa| > |f(\kappa)|$ we can again use the pigeonhole principle and obtain two states $p_1, p_2 \in \kappa$ such that $q_{p_1} = q_{p_2}$. We call this state $q := q_{p_1}$.
	
	For each word $w \in \Sigma^*$, $c_{\mathcal{S}'}(\delta^*_{\mathcal{S'}}(p_1, w)) = c_{\mathcal{B}}(\delta^*_{\mathcal{B}}(q, w))$ or $\mathcal{S}'$ leaves $C$ while reading $w$ from $p_1$. The same holds for $p_2$. As $p_1$ and $p_2$ are distinct states in $\mathcal{S}'$, they cannot be Moore equivalent. That means there has to be a word $u$ with $c_{\mathcal{S}'}(\delta^*_{\mathcal{S'}}(p_1, u)) \neq c_{\mathcal{S}'}(\delta^*_{\mathcal{S'}}(p_2, u))$. However, $\mathcal{S}'$ cannot leave $C$ while reading $u$ from $p_1$; if it did, it would reach a class $\lambda$ and the transition would lead to the representative $r_\lambda$. As $\equiv_\text{\Ankh}$ is a congruence relation, reading $u$ from $p_2$ would lead to $r_\lambda$ at the same position, so from that point on the two runs would be exactly the same.
	
	Hence, $p'_1 = \delta_{\mathcal{S}'}^*(p_1, u)$ and $p'_2 = \delta_{\mathcal{S}'}^*(p_2, u)$ are still in $C$ and $c_{\mathcal{S}'}(p'_1) \neq c_{\mathcal{S}'}(p'_2)$. That means at least one of these values must be different to $c_{\mathcal{B}}(\delta_{\mathcal{B}}^*(q, u))$, which contradicts our assumption.
	
	\vspace{5pt}
	
	We can use the claim that he have just shown to finish our overall proof. Fix an arbitrary $q_0 \in f(\kappa)$. We define a sequence of finite words $(\alpha_n)_{n \in \mathbb{N}}$ such that every $\alpha_n$ is a prefix of $\alpha_{n+1}$ and the runs of $\mathcal{S}'$ and $\mathcal{B}$ from $p$ and $q_0$ respectively differ in priority at least $n$ times. Then $\alpha := \bigcup_n \alpha_n$ is an $\omega$-word that is a witness for $(\mathcal{S}', p)$ and $(\mathcal{B}, q_0)$ not being priority almost equivalent.
	
	We make sure that after reading any $\alpha_n$ from $p$, $\mathcal{S}'$ moves back to $p$. Let $\alpha_0 := \varepsilon$ and assume for induction that $\alpha_n$ has already been defined. After reading $\alpha_0$, $\mathcal{S}'$ reaches $p$ and $\mathcal{B}$ reaches some $q$. If $q \notin f(\kappa)$, i.e. $(\mathcal{S}', p) \not\equiv_\text{\Ankh} (\mathcal{B}, q)$, there is a witness $\beta$ and we can simply set $\alpha := \alpha_n \beta$. Otherwise, $q \in f(\kappa)$. With the claim we have proven, we can find a word $w_q$ such that reading the word from $p$ and $q$ leads to states $p'$ and $q'$ that have different priorities but $p'$ is still in $C$. Thus, there is a word $u$ that leads back from $p'$ to $p$. We then set $\alpha_{n+1} := \alpha_n w_q u$. This finishes our proof.
\end{proof}








\begin{appendices}

\chapter{Examples}

\section{Skip merger}
Figure \ref{fig:examples:skip1} shows an automaton. If we use $\sim = \equiv_L$, then both states are $\sim$-equivalent. As $q_0$ is not reachable from $q_1$, the skip merger can remove $q_0$ to get to the automaton in figure \ref{fig:examples:skip2}.

\begin{figure}
\centering
\begin{tikzpicture}[shorten >=1pt,node distance=2cm,on grid,initial text=]
  \node[state]           (0)                {$q_0,0$};
  \node[state]           (1) [right=of 0]   {$q_1,1$};
  \path[->] (0) edge node [above] {a} (1)
  			(1) edge [loop right] node {a} (1);
\end{tikzpicture}
\caption{Example for the skip merger.}
\label{fig:examples:skip1}
\end{figure}


\begin{figure}
\centering
\begin{tikzpicture}[shorten >=1pt,node distance=2cm,on grid,initial text=]
  \node[state]           (1) [right=of 0]   {$q_1,1$};
  \path[->] (1) edge [loop right] node {a} (1);
\end{tikzpicture}
\caption{Example for the skip merger.}
\label{fig:examples:skip2}
\end{figure}



\section{Delayed Simulation}
Figure \ref{fig:examples:desim1} shows an automaton. The entire delayed simulation automaton is of size 75, so we only consider one pair of states for the example. To determine whether $q_0 \leq_\text{de} q_1$ is true, we need the part of $\mathcal{A}_\text{de}$ that is displayed in figure \ref{fig:examples:desim2}. The question to check is whether from $(q_0, q_1, \checkmark)$, every word is accepted. 

As we can see, the run induced by $ba^\omega$ ends in the loop $(q_3, q_3, 0) (q_2, q_2, 0)$ which is not accepting. Thus, $q_0 \not\leq^{ba^\omega}_\text{de} q_1$ and therefore $q_0 \not\leq_\text{de} q_1$. This coincides with the definition of delayed simulation: if the run starting in $q_0$ once sees priority 0, afterwards neither run will see that priority again, as they are stuck in the loop $q_3 q_2$.

\begin{figure}
\centering
\begin{tikzpicture}[shorten >=1pt,node distance=2cm,on grid,initial text=]
  \node[state]           (0)                {$q_0,1$};
  \node[state]           (1) [right=of 0]   {$q_1,1$};
  \node[state]           (2) [above=of 0]   {$q_2,1$};
  \node[state]           (3) [right=of 2]   {$q_3,1$};
  \node[state]           (4) [right=of 3]   {$q_4,0$};
  \path[->] (0) edge [bend left] node [above] {a} (1)
  			(0) edge node [left] {b} (2)
            (1) edge [bend left] node [below] {a} (0)
            (1) edge node [below] {b} (4)
            (2) edge [bend left] node [above] {a,b} (3)
            (3) edge [bend left] node [below] {a} (2)
            (3) edge [bend left] node [above] {b} (4)
            (4) edge [bend left] node [below] {a,b} (3);
\end{tikzpicture}
\caption{Example for the delayed simulation merger.}
\label{fig:examples:desim1}
\end{figure}


\begin{figure}
\centering
\begin{tikzpicture}[shorten >=1pt,node distance=3cm,on grid,initial text=]
  \node[state,accepting] (0)			    {$q_0,q_1,\checkmark$};
  \node[state,accepting] (1) [right=of 0]   {$q_2,q_4,\checkmark$};
  \node[state,accepting] (2) [right=of 1]   {$q_3,q_3,\checkmark$};
  \node[state] 			 (3) [below=of 0]   {$q_1,q_0,\checkmark$};
  \node[state] 			 (4) [right=of 3]   {$q_4,q_2,0$};
  \node[state] 			 (5) [right=of 4]   {$q_3,q_3,0$};
  \node[state,accepting] (6) [right=of 5]   {$q_4,q_4,\checkmark$};
  \node[state] 			 (7) [below=of 5]   {$q_2,q_2,0$};
  \node[state,accepting] (8) [above right=of 2]   {$q_2,q_2,\checkmark$};
  \path[->] (0) edge node [above] {a} (1)
  			(0) edge [bend left] node [right] {b} (3)
            (1) edge node [above] {a,b} (2)
            (2) edge [bend left] node [above] {a} (6)
            (2) edge [bend left] node [above] {b} (8)
            (3) edge node [above] {a} (4)
            (3) edge [bend left] node [left] {b} (0)
            (4) edge node [above] {a,b} (5)
            (5) edge [bend left] node [right] {a} (7)
            (5) edge node [above] {b} (6)
            (6) edge [bend left] node [left] {a,b} (2)
            (7) edge [bend left] node [left] {a,b} (5)
            (8) edge [bend left] node [left] {a,b} (2);
\end{tikzpicture}
\caption{Example for the delayed simulation merger. (Delayed simulation game)}
\label{fig:examples:desim2}
\end{figure}



\section{Iterated Moore}
Figure \ref{fig:examples:im1} shows our example automaton for $\mu_{IM}$. We have $q_0 \not\equiv_M q_1$, as the two states have different priorities. With iterated Moore equivalence, we actually have $q_0 \equiv_{IM} q_1$:

First, we choose our order of SCCs as $S_0 = \{q_0\}$ and $S_1 = \{q_1\}$. $\mathcal{B}_1$ then has only one state $q_1$ with the transition to itself. $\mathcal{B}'_0$ is then the same automaton as $\mathcal{A}$. $S_0$ is a trivial SCC, it is not $M'_0$-equivalent to any other state and $(\delta'_0(q_0, a), \delta'_0(q_1, a)) \in M'_0$ for all $a$. Therefore, $c_0(q_0)$ is set to $1$ and we have $(q_0, q_1) \in M_0$. This results in figure \ref{fig:examples:im2}.

\begin{figure}
\centering
\begin{tikzpicture}[shorten >=1pt,node distance=2cm,on grid,initial text=]
  \node[state]           (0)                {$q_0,0$};
  \node[state]           (1) [right=of 0]   {$q_1,1$};
  \path[->] (0) edge node [above] {a} (1)
            (1) edge [loop right] node [right] {a} (1);
\end{tikzpicture}
\caption{Example for the iterated Moore merger.}
\label{fig:examples:im1}
\end{figure}


\begin{figure}
\centering
\begin{tikzpicture}[shorten >=1pt,node distance=2cm,on grid,initial text=]
  \node[state]           (1)   {$q_1,1$};
  \path[->] (1) edge [loop right] node [right] {a} (1);
\end{tikzpicture}
\caption{Example for the iterated Moore merger.}
\label{fig:examples:im2}
\end{figure}


\section{Path Refinement}
We use the same example automaton as for delayed simulation, \ref{fig:examples:desim1}. $\equiv_L$ has the three equivalence classes $\{q_0, q_1\}$, $\{q_2, q_4\}$, and $\{q_3\}$. We choose to merge by $\lambda = \{q_2, q_4\}$.

Figure \ref{fig:examples:pr1} displays the transition structure $\mathcal{A}^\lambda_\text{visit}$. The relation $V$ then consists of three equivalence classes: $\{(q_3, 1, \perp), (q_3, 0, \perp)\}$, $\{(q_2, 1, 0), (q_4, 0, 0)\}$, and $\{(q_2, 1, 1), (q_4, 0, 1)\}$. We are interested in the question whether $\iota_{q_2}^1 = (q_2, 1, 1)$ and $\iota_{q_4}^1 = (q_4, 0, 1)$ are equivalent.

Unfortunately, $V_M$, the congruence refinement of $V$, actually only contains singleton classes, so $q_2 \not\equiv_\text{PR}^\lambda q_4$. This is because $\delta_\text{visit}^*(\iota_{q_2}^1, aa) = (q_2, 1, 1)$ and $\delta_\text{visit}^*(\iota_{q_4}^1, aa) = (q_2, 1, 0)$, which is not a pair in $V$. 

This shows that by reading $aa$ from $q_2$, the automaton reaches back to $\lambda$ and visits priorities of at least 1. On the other hand, from $q_4$ reading $aa$ sees priority 0 once.

\begin{figure}
\centering
\begin{tikzpicture}[shorten >=1pt,node distance=2.5cm,on grid,initial text=]
  \node[state]           (0)                {$q_2,1,0$};
  \node[state]           (1) [right=of 0]   {$q_3,0,\perp$};
  \node[state]           (2) [below=of 0]   {$q_3,1,\perp$};
  \node[state]           (3) [right=of 2]   {$q_4,0,0$};
  \node[state]           (4) [left=of 2]   {$q_2,1,1$};
  \node[state]           (5) [right=of 1]   {$q_4,0,1$};
  \path[->] (0) edge node [left] {a,b} (2)
  			(1) edge node [above] {a} (0)
  			(1) edge [bend left] node [right] {b} (3)
  			(2) edge [bend left] node [above] {a} (4)
  			(2) edge node [above] {b} (3)
  			(3) edge [bend left] node [left] {a,b} (1)
  			(4) edge [bend left] node [above] {a,b} (2)
  			(5) edge node [above] {a,b} (1);
\end{tikzpicture}
\caption{Example for the path refinement merger.}
\label{fig:examples:pr1}
\end{figure}


\section{Threshold Moore}
Figure \ref{fig:examples:tm1} shows an example DPA. All states are language equivalent; to be specific, they all accept the language $\Sigma^* b^\omega$. If we consider $\equiv_M^{\leq 0}$, we find that $q_0 \equiv_M^{\leq 0} q_1$ and therefore also $q_0 \equiv_\text{TM}^{\equiv_L} q_1$.

\begin{figure}
\centering
\begin{tikzpicture}[shorten >=1pt,node distance=2cm,on grid,initial text=]
  \node[state]           (0)                {$q_0,0$};
  \node[state]           (1) [right=of 0]   {$q_2,1$};
  \node[state]           (2) [below=of 0]   {$q_1,0$};
  \node[state]           (3) [below=of 1]   {$q_3,2$};
  \path[->] (0) edge node [above] {a} (1)
  			(0) edge node [above] [above] {b} (3)
  			(1) edge [bend left] node [right] {a} (3)
  			(1) edge [loop right] node {b} (1)
  			(2) edge node [above] {a} (3)
  			(2) edge node [below] {b} (1)
  			(3) edge [bend left] node [right] {a} (1)
  			(3) edge [loop right] node {b} (3);
\end{tikzpicture}
\caption{Example for the Threshold Moore merger.}
\label{fig:examples:tm1}
\end{figure}


\section{LSF} %TODO desim?
For the LSF merger, we use figure \ref{fig:examples:lsf1} as an example. As parameters, we choose $\sim = \equiv_L$ and $k = 1$. 

All states are language equivalent: from any state, the language $(\Sigma^* a)^\omega$ is accepted. Also, $q_1 \equiv_M^{\leq 1} q_3$ and $q_2 \equiv_M^{\leq 1} q_4$. Thus, the equivalence classes of $\equiv_\text{LSF}^{1, \equiv_L}$ are $\{q_0\}$, $\{q_1, q_3\}$, and $\{q_2, q_4\}$.

One possible choice for $\preceq_k$ is $q_1 \simeq_k q_2 \prec_k q_3 \simeq_k q_4$. Thus, we can merge these states and obtain  what is shown in figure \ref{fig:examples:lsf2}.


\begin{figure}
\centering
\begin{tikzpicture}[shorten >=1pt,node distance=2cm,on grid,initial text=]
  \node[state]           (0)                {$q_0,0$};
  \node[state]           (1) [below left=of 0]   {$q_1,2$};
  \node[state]           (2) [below right=of 0]   {$q_3,4$};
  \node[state]           (3) [below=of 1]   {$q_2,3$};
  \node[state]           (4) [below=of 2]   {$q_4,5$};
  \path[->] (0) edge [bend left] node [below] {a} (2)
  			(0) edge [bend left] node [above] {b} (1)
  			(1) edge [bend left] node [above] {a} (0)
  			(1) edge [bend left] node [right] {b} (3)
  			(2) edge [bend left] node [below] {a} (0)
  			(2) edge [bend left] node [right] {b} (4)
  			(3) edge [bend left] node [left] {a} (1)
  			(3) edge [loop left] node [left] {b} (3)
  			(4) edge [bend left] node [left] {a} (2)
  			(4) edge [loop right] node [right] {b} (4);
\end{tikzpicture}
\caption{Example for the LSF merger.}
\label{fig:examples:lsf1}
\end{figure}




\begin{figure}
\centering
\begin{tikzpicture}[shorten >=1pt,node distance=2cm,on grid,initial text=]
  \node[state]           (0)                {$q_0,0$};
  \node[state]           (2) [below=of 0]   {$q_3,4$};
  \node[state]           (4) [below=of 2]   {$q_4,5$};
  \path[->] (0) edge [bend left] node [right] {a,b} (2)
  			(2) edge [bend left] node [left] {a} (0)
  			(2) edge [bend left] node [right] {b} (4)
  			(4) edge [bend left] node [left] {a} (2)
  			(4) edge [loop right] node [right] {b} (4);
\end{tikzpicture}
\caption{Example for the LSF merger.}
\label{fig:examples:lsf2}
\end{figure}


\end{appendices}

%\nocite{*}
\bibliography{citations}
\bibliographystyle{plain}


%\clearpage\null\thispagestyle{empty}\newpage

\end{document}









