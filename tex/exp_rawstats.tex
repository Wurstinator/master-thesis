



\section{Experimental Setup}
Even though the focus of this thesis lies in theoretical research, we consider it important to at least roughly analyze the practical \enquote{real world} efficiency of ever algorithm. All of them were implemented and tested on a set of deterministic parity automata.

The programming language of choice was C++14. The source code can be found in \cite{Tollkoetter2018}. The computer used to run the tests was an Arch Linux 4.19.4 64 bit machine powered by an AMD Ryzen 5 1600 processor and 16 GB DDR4-2400 RAM.

Several automata generated randomly using different parameters were used in the testing process. Three major different techniques of generation were used:

\begin{enumerate}
	\item Use Spot (\cite{duret.16.atva2}) to generate a random DPA. (called \textsf{gendet})
	\item Use Spot to generate a random non-deterministic B\"uchi automaton and use Spot to convert it to a DPA. (called \textsf{detspot})
	\item Use Spot to generate a random non-deterministic B\"uchi automaton and use nbautils (\cite{Pirogov2018}) to convert it to a DPA. (called \textsf{detnbaut})
\end{enumerate}

Figures \ref{fig:rawstats:rawstats_size}, \ref{fig:rawstats:rawstats_prios}, \ref{fig:rawstats:rawstats_sccs}, \ref{fig:rawstats:rawstats_langclasnum} and \ref{fig:rawstats:rawstats_avg_langclassize} present some information about the automata. Given that most algorithms become too slow at about 150 states, we did not continue the generation beyond that point. 

For the \textsf{detnbaut} set, some proecdures can be sped up which is why the upper limit is higher. This is because the nbautils tool internally computes an equivalence relation that implies language equivalence during the determinization. The relation lets us skip the explicit computation of language equivalent states. We refer to \cite{Schewe2009} and \cite{Piterman2007} for more information.

Regarding the plots for stats other than the number of states, we need to mention that we removed far outliers from the data set. For example, the x axis of the SCC plots only ranges up to 10 even though there was as a single example with close to 100 SCCs. Showing these data points would only obscure the information of the plot though.

The number of priorities is rather small in general. For \textsf{gendet}, we intentionally limited the number to 5 to mimic real world behavior that can be found in the \textsf{detnbaut} set. In comparison to nbautils, Spot does not perform priority reduction on the determinization result which explains why the \textsf{detspot} automata use more priorities in general. 

The number of SCCs is consistently small among all three sets. As said before, \textsf{detspot} and \textsf{gendet} contain a few examples of automata with up to 100 SCCs but these are extreme outliers. This low number of SCCs is important to consider, as multiple of the algorithms such as the skip merger perform better the less connected the automaton is.

Finally, the average size of equivalence classes $\mathfrak{C}(\equiv_L)$ and the number of classes with more than one element. Again, this is of relevance to some of the reduction algorithms such as LSF as only states which are language equivalent can be merged. We can observe that the \textsf{gendet} set almost entirely consists of trivial classes (i.e. classes of size 1) while \textsf{detspot} and \textsf{detnbaut} show more promise in that regard.

\vspace{5pt}

Besides the three classes of data sets that we described above, we also looked at special families of automata that are specifically designed to be difficult to minimize; see \cite{Michel1988} and \cite{Thomas1997}. Unfortunately, these automata proved to be too difficult for our procedures and almost no algorithm could achieve any reduction on an automaton of any size. Thus, we will not discuss these results in any more detail in the upcoming chapters.

\begin{figure}
	\centering
	\begin{minipage}{0.49\textwidth}
		\includegraphics[page=1,height=.3\textheight]{../data/analysis/rawstats_gendet.pdf} 
		\includegraphics[page=1,height=.3\textheight]{../data/analysis/rawstats_detspot.pdf} 
		\includegraphics[page=1,height=.3\textheight]{../data/analysis/rawstats_detnbaut.pdf}
		\caption{Sizes of the automata in the testing environment.}
		\label{fig:rawstats:rawstats_size}
	\end{minipage}
	\hfill
	\begin{minipage}{0.49\textwidth}
		\includegraphics[page=2,height=.3\textheight]{../data/analysis/rawstats_gendet.pdf} 
		\includegraphics[page=2,height=.3\textheight]{../data/analysis/rawstats_detspot.pdf} 
		\includegraphics[page=2,height=.3\textheight]{../data/analysis/rawstats_detnbaut.pdf}
		\caption{Number of priorities in the automata in the testing environment.}
		\label{fig:rawstats:rawstats_prios}
	\end{minipage}
\end{figure}

\begin{figure}
	\centering
	\begin{minipage}{0.49\textwidth}
		\includegraphics[page=3,height=.3\textheight]{../data/analysis/rawstats_gendet.pdf} 
		\includegraphics[page=3,height=.3\textheight]{../data/analysis/rawstats_detspot.pdf} 
		\includegraphics[page=3,height=.3\textheight]{../data/analysis/rawstats_detnbaut.pdf}
		\caption{Number of SCCS in the automata in the testing environment.}
		\label{fig:rawstats:rawstats_sccs}
	\end{minipage}
	\hfill
	\begin{minipage}{0.49\textwidth}
		\includegraphics[page=4,height=.3\textheight]{../data/analysis/rawstats_gendet.pdf} 
		\includegraphics[page=4,height=.3\textheight]{../data/analysis/rawstats_detspot.pdf} 
		\includegraphics[page=4,height=.3\textheight]{../data/analysis/rawstats_detnbaut.pdf}
		\caption{Number of classes in $\mathfrak{C}(\equiv_L)$ that contain more than one state.}
		\label{fig:rawstats:rawstats_langclasnum}
	\end{minipage}
\end{figure}


\begin{figure}
	\centering
	\begin{minipage}{0.49\textwidth}
		\includegraphics[page=5,height=.3\textheight]{../data/analysis/rawstats_gendet.pdf} 
		\includegraphics[page=5,height=.3\textheight]{../data/analysis/rawstats_detspot.pdf} 
		\includegraphics[page=5,height=.3\textheight]{../data/analysis/rawstats_detnbaut.pdf}
		\caption{Average size of $\equiv_L$-classes of the automata in the testing environment.}
		\label{fig:rawstats:rawstats_avg_langclassize}
	\end{minipage}
\end{figure}


