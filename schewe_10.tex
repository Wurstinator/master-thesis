

\subsection{Schewe Automaton}

\begin{defn}
	Let $\mathcal{A} = (Q, \Sigma, q_0, \delta, c)$ be a deterministic parity automaton. For $w \in \Sigma^* \cup \Sigma^\omega$ and $q \in Q$, we define $\lambda_\mathcal{A}(q, w) \in \mathbb{N}^{1+|w|}$ as follows: Let $q_0 q_1 \dots \in Q^{1+|w|}$ be the unique run of $\mathcal{A}$ on $w$. Then $\lambda_\mathcal{A}(q, w)(n) = c(q_n)$.
	
	We define the \textbf{reachability order} $\preceq_\text{reach}^\mathcal{A} \subseteq Q \times Q$ as $p \preceq_\text{reach}^\mathcal{A} q$ iff $q$ is reachable from $p$. (\enquote{$p$ is closer to $q_0$ than $q$}). Note that $p \preceq_\text{reach}^\mathcal{A} q$ and $q \preceq_\text{reach}^\mathcal{A} p$ together mean that $p$ and $q$ reside in the same SCC.
\end{defn}

\begin{defn}
	Let $\mathcal{A} = (Q, \Sigma, q_0, \delta, c)$ be a DPA and let $\sim \subseteq Q \times Q$ be a congruence relation on $\mathcal{A}$. We define the \textbf{Schewe} automaton $\mathcal{S}(\mathcal{A}, \sim)$ as follows:
	
	Let $\preceq \subseteq Q \times Q$ be a relation s.t.
	\begin{itemize}
		\item $\preceq$ is a total preorder.
		\item If $q$ is reachable from $p$, then $p \preceq q$.
		\item If $p \simeq q$, then $p$ and $q$ are in the same SCC.
	\end{itemize}
	
	For each state $q$, let $[q]_\sim = \{ p \in Q \mid q \sim p \}$ be its equivalence class of $\sim$ and let $\bigslant{Q}{\sim} = \{[q]_\sim \mid q \in Q\}$ be the set of equivalence classes. For each such class $\mathfrak{c}$ we fix a representative $r_\mathfrak{c} \in \mathfrak{c}$ which is $\preceq$-maximal in its class.
	
	The automaton is then almost the same as the original DPA, with only a few modifications. Namely, $\mathcal{S} = (Q, \Sigma, r_{[q_0]_\sim}, \delta_\mathcal{S}, c)$.
	
	For each transition $\delta_\mathcal{S}(q, a)$, let $\delta(q, a) = p$. If $q \prec r_{[p]_\sim}$, then $\delta_\mathcal{S}(q, a) = r_{[p]_\sim}$. Otherwise, we keep $\delta_\mathcal{S}(q, a) = p$. In other words, every time a transition leaves the SCC, it skips to the representative which lies as \enquote{deep} inside the automaton as possible.
\end{defn}

Note that the choice of $\preceq$ and the representatives allow some degree of freedom when construction the automaton, so in general there are multiple Schewe automata for each DPA. The relevant properties that we consider hold for all of them, so we just fix an arbitrary choice of all Schewe automata and call this \textbf{the} Schewe automaton.

\begin{lem}
	For a given $\mathcal{A}$ and $\sim$, the Schewe automaton $\mathcal{S}$ can be computed in $\mathcal{O}(|\mathcal{A}|)$.
\end{lem}

\begin{proof}
	%TODO
\end{proof}





\newpage

\subsection{Schewe automaton structure}

\begin{lem}
\label{lem:schewe:run_growing}
	Let $\mathcal{A}$ be a DPA and $\sim$ be a congruence relation. Let $\rho$ be a run on $\alpha$ starting at a reachable state in $\mathcal{S}(\mathcal{A}, \sim)$. Then for all $i$, $\rho(i) \preceq \rho(i+1)$.
	
	Furthermore, we have $\rho(i) \prec \rho(i+1)$ if and only if $\rho(i) \prec r_{[\delta_\mathcal{A}(\rho(i), \alpha(i))]_\sim}$.
\end{lem}

\begin{proof}
	Let $i$ be an arbitrary index of the run. If $\rho(i)$ to $\rho(i+1)$ is also a transition in $\mathcal{A}$, then $\rho(i+1)$ is reachable from $\rho(i)$ in $\mathcal{A}$ and hence $\rho(i) \preceq \rho(i+1)$ by definition of the preorder. Otherwise the transition used was redirected in the construction. The way the redirection is defined, this implies $\rho(i) \prec \rho(i+1)$.
	
	We move on to the second part of the lemma. If $\rho(i) \prec r_{[\delta_\mathcal{A}(\rho(i), \alpha(i))]_\sim}$, then the transition is redirected to $\rho(i+1) = r_{[\delta_\mathcal{A}(\rho(i), \alpha(i))]_\sim}$ and the statement holds. 
	
	For the other direction, let $\rho(i) \prec \rho(i+1)$ and assume towards a contradiction that $\rho(i) \not\prec r_{[\delta_\mathcal{A}(\rho(i), \alpha(i))]_\sim}$. This means that the transition was not redirected and $\rho(i+1) = \delta_\mathcal{A}(\rho(i), \alpha(i))$. Since $\preceq$ is total, we have $r_{[\delta_\mathcal{A}(\rho(i), \alpha(i))]_\sim} = r_{[\rho(i+1)]_\sim} \preceq \rho(i) \prec \rho(i+1)$ which contradicts the $\preceq$-maximality of representatives.
\end{proof}

\begin{lem}
	Let $\mathcal{A}$ be a DPA and $\sim$ be a congruence relation. Let $p$ and $q$ be two reachable states in $\mathcal{S}(\mathcal{A}, \sim)$. If $p \sim q$, then $p$ and $q$ lie in the same SCC. 
\end{lem}

\begin{proof}
	It suffices to restrict ourselves to $q = r_{[q]_\sim} = r_{[p]_\sim}$. If we can prove the Lemma for this case, then the general statement follows as every state in $[q]_\sim$ must lie in the same SCC as $q$.
	
	Let $p_0 \cdots p_n$ be a minimal run of $\mathcal{S}$ that reaches $p$. By Lemma \ref{lem:schewe:run_growing}, we have $p_0 \preceq \dots \preceq p_n$. Whenever $p_i \prec p_{i+1}$, a redirected transition to the representative $r_{[p_{i+1}]_\sim} = p_{i+1}$ is taken. 
	
	Let $k$ be the first position after which no redirected transition is taken anymore. For the first case, assume that $k < n$. Then $p_i \simeq r_{[p_{i+1}]_\sim}$ for all $i \geq k$. In particular, $p_{n-1} \simeq q$. Since $p_{n-1} \preceq p_n$, we also have $q \preceq p_n$. The representatives are chosen $\preceq$-maximal in their $\sim$-class, so $q \simeq p_n$.
	
	The second case is $k = n$. In that case, the transition from $p_{n-1}$ to $p_n$ is redirected and $p_n = r_{[p_n]_\sim} = q$.
\end{proof}


\begin{lem}
\label{lem:schewe:run_suffix}
	Let $\rho \in Q^\omega$ be an infinite run in $\mathcal{S}$ starting at a reachable state. Then $\rho$ has a suffix that is a run in $\mathcal{A}$.
\end{lem} 

\begin{proof}
	We show that only finitely often a redirected transition is used in $\rho$. Then, from some point on, only transitions that also exist in $\mathcal{A}$ are used. The suffix starting at this point is the run that we are looking for.
	
	Let $\rho = p_0 p_1 \cdots$. By Lemma \ref{lem:schewe:run_growing}, we have $p_i \preceq p_{i+1}$ for all $i$ and $p_i \prec p_{i+1}$ whenever a redirected transition is taken. As $Q$ is finite, we can only move up in the order finitely often. This proves our claim.
\end{proof}


\begin{lem}
\label{lem:schewe:sim_a_s}
	For all $q \in Q$ and $a \in \Sigma$, $\delta_\mathcal{A}(q, a) \sim \delta_{\mathcal{S}(\mathcal{A}, \sim)}(q, a)$.
\end{lem}

\begin{proof}
	If $\delta_\mathcal{A}(q, a) = \delta_{\mathcal{S}(\mathcal{A}, \sim)}(q, a)$, the statement is clear. Otherwise, the transition was redirected and $\delta_{\mathcal{S}(\mathcal{A}, \sim)}(q, a) = r_{[\delta_\mathcal{A}(q, a)]_\sim}$ which also shows the $\sim$-equivalence of the two states.
\end{proof}


\newpage


\subsection{Special cases of the Schewe automaton}
\begin{defn}
	Two DPAs $\mathcal{A}$ and $\mathcal{B}$ are \textbf{priority almost-equivalent}, if for all words $\alpha \in \Sigma^\omega$, $\lambda_\mathcal{A}(q_0^\mathcal{A}, \alpha)$ and $\lambda_\mathcal{B}(q_0^\mathcal{B}, \alpha)$ differ in only finitely many positions.
	We call two states $p, q \in Q$ of $\mathcal{A}$ priority almost-equivalent, $\mathcal{A}_q$ and $\mathcal{A}_p$ are priority almost-equivalent, where $\mathcal{A}_q$ behaves like $\mathcal{A}$ with initial state $q$.
\end{defn}

\begin{lem}
	Priority almost-equivalence is a congruence relation.
\end{lem}

\begin{proof}
	Obvious.
\end{proof}


\begin{lem}
	Priority almost-equivalence implies language equivalence.
\end{lem}

\begin{proof}
	Let $\mathcal{A} = (Q_\mathcal{A}, \Sigma, q_0^\mathcal{A}, \delta_\mathcal{A}, c_\mathcal{A})$ and $\mathcal{B} = (Q_\mathcal{B}, \Sigma, q_0^\mathcal{B}, \delta_\mathcal{B}, c_\mathcal{B})$ be two DPA that are priority almost-equivalent and assume towards a contradiction that they are not language equivalent. Due to symmetry we can assume that there is a $w \in L(\mathcal{A}) \setminus L(\mathcal{B})$. 
	
	Consider $\alpha = \lambda_\mathcal{A}(q_0^\mathcal{A}, w)$ and $\beta = \lambda_\mathcal{B}(q_0^\mathcal{B}, w)$, the priority outputs of the automata on $w$. By choice of $w$, we know that $a := \max \text{Inf}(\alpha)$ is even and $b := \max \text{Inf}(\beta)$ is odd. Without loss of generality, assume $a > b$. That means $a$ is seen only finitely often in $\beta$ but infinitely often in $a$. Hence, $\alpha$ and $\beta$ differ at infinitely many positions where $a$ occurs in $\alpha$. That would mean $w$ is a witness that the two automata are not priority almost-equivalent, contradicting our assumption.
\end{proof}


\begin{lem}
	Let $\mathcal{A}$ be a DPA and $\sim$ be a congruence relation that implies language equivalence. Then $\mathcal{A}$ and $\mathcal{S} := \mathcal{S}(\mathcal{A}, \sim)$ are language equivalent.
\end{lem}

\begin{proof}
	Let $\alpha \in \Sigma^\omega$ be a word and let $\rho$ be the run of $\mathcal{S}$ on $\alpha$. By Lemma \ref{lem:schewe:run_suffix}, $\rho$ has a suffix $\pi$ which is a run segment of $\mathcal{A}$ on some suffix $\beta$ of $\alpha$. The acceptance condition of DPAs is prefix independent, so $\alpha \in L(\mathcal{S})$ iff $\rho$ is an accepting run iff $\pi$ is an accepting run. Since the priorities do not change during the construction, $\pi$ is accepting in $\mathcal{S}$ iff it is accepting in $\mathcal{A}$.
	
	Let $w \in \Sigma^*$ be the prefix of $\alpha$ with $\alpha = w \beta$. By Lemma \ref{lem:schewe:sim_a_s}, we know that $\delta^*_\mathcal{A}(q_0, w) \sim \delta^*_\mathcal{S}(q_0, w)$. Since every state is $\sim$-equivalent to its representative and $\sim$ is a congruence relation, we also know $\delta^*_\mathcal{S}(q_0, w) \sim \delta^*_\mathcal{S}(r_{[q_0]_\sim}, w)$. From $\delta^*_\mathcal{S}(r_{[q_0]_\sim}, w)$, the run $\pi$ accepts $\beta$ iff $\alpha \in L(\mathcal{S})$. As $\sim$ implies language equivalence, the same must hold for $\delta^*_\mathcal{A}(q_0, w)$. Therefore, $\alpha \in L(\mathcal{A})$ iff $\alpha \in L(\mathcal{S})$.
\end{proof}


\begin{lem}
	Let $\mathcal{A}$ be a DPA and $\sim$ be the relation of priority almost-equivalence. Let $\mathcal{S}'$ be the Moore-minimization of $\mathcal{S}(\mathcal{A}, \sim)$. There is no smaller DPA than $\mathcal{S}'$ that is priority almost-equivalent to $\mathcal{A}$.
\end{lem}

\begin{proof}
	%TODO
\end{proof}


\begin{lem}
	The priority almost-equivalence of a DPA $\mathcal{A}$ can be computed in $\mathcal{O}(|\mathcal{A}|^2)$.
\end{lem}

\begin{proof}
	%TODO
\end{proof}


\newpage








	


\begin{defn}
	Let $\mathcal{A} = (Q, \Sigma, q_0, \delta, c)$ be a DPA. We define the \textbf{Moore-minimization} $\mathcal{B}$ as the parity automaton corresponding to the minimal Moore automaton of $\mathcal{A}$. That means it is the minimal automaton such that $\lambda_\mathcal{A}(\alpha) = \lambda_\mathcal{B}(\alpha)$ for all $\alpha \in \Sigma^\omega$.
	
	More specifically, we define the congruence relation $\sim \subseteq Q \times Q$ by $p \sim q$ iff $\forall w \in \Sigma^*: \lambda_\mathcal{A}(p, w) = \lambda_\mathcal{A}(q, w)$. Then $\mathcal{A}'$ is constructed from $\mathcal{A}$ by removing unreachable states (from $q_0$). $\mathcal{B} = (\bigslant{Q}{\sim}, \Sigma, [q_0]_\sim, \delta_\mathcal{B}, c_\mathcal{B})$ is the quotient automaton of $\bigslant{\mathcal{A}'}{\sim}$ with $\delta_\mathcal{B}([q]_\sim, a) = [\delta(q, a)]_\sim$ and $c_\mathcal{B}([q]_\sim) = c(q)$.
\end{defn}

\begin{lem}
	For a given DPA $\mathcal{A}$, the Moore-minimization can be computed in $\mathcal{O}$.
\end{lem}





