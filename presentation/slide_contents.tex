
\section{Deterministic Parity Automata}
\begin{frame}{$\omega$-automata}
$\omega$-words are words of one-sided infinite length: \\
$\Sigma^\omega = $ functions from $\mathbb{N}$ to $\Sigma$

\vspace{.5cm}

$\omega$-automata are finite transition structures that describe a language $L \subseteq \Sigma^\omega$ 

\vspace{.5cm}

Deterministic parity automata (DPA):
\begin{itemize}
	\item State set $Q$
	\item Alphabet $\Sigma$
	\item Transition function $\delta : Q \times \Sigma \rightarrow Q$
	\item Priority function $c : Q \rightarrow \mathbb{N}$
\end{itemize}

An $\omega$-word $\alpha$ starting in a state $q_0 \in Q$ induces a run $q_0 q_1 q_2 \dots$. \\
The DPA accepts $\alpha$ iff the \emph{smallest} priority that occurs infinitely often in the sequence $c(q_0) c(q_1) c(q_2) \dots$ is \emph{even}.

\end{frame}


\section{Why do we need heuristic reduction?}
\begin{frame}{Why do we need heuristic reduction?}

Goal: Reduce number of states in the automaton to ease run time of follow up algorithms.

\vspace{.7cm}

\textbf{Minimization Problem}: Given an automaton $\mathcal{A}$, what is the smallest number of states required to recognize the same language as $\mathcal{A}$?

\vspace{.7cm}

For DFAs: Minimization is solvable in $\mathcal{O}(n \log n)$. \cite{Hopcroft1971}

For DPAs: Minimization is NP-hard. \cite{Schewe2010}

\end{frame}


\begin{frame}{Moore Minimization}
A DPA can be interpreted as a Moore automaton with $c$ being the output function.

\begin{defn}[Moore equivalence]
	$p \equiv_M q$ iff $\forall w \in \Sigma^*: c(\delta^*(p, w)) = c(\delta^*(q, w))$.
\end{defn}

\vspace{.5cm}
\pause

\begin{theorem}
	Deterministic Moore automata can be minimized in log-linear time.
\end{theorem}

Idea: Build the quotient automaton w.r.t. $\equiv_M$.

\vspace{.5cm}

The same algorithm can be used to reduce DPAs but will not give minimal DPAs in general.
\end{frame}




\section{Merger functions as a framework}
\begin{frame}{Merger functions}
\emph{Merger functions} $\mu$ map from some $D \subseteq 2^Q$ into $2^Q \setminus \{\emptyset\}$.

\vspace{.5cm}

$M, C \subseteq Q$

\begin{equation*}
\tikz[baseline]{\node {$\mu($}} \tikz[baseline]{\node(d4) {$M$}} \tikz[baseline]{\node {$) =$}} \tikz[baseline]{\node(d5){$C$}}
\end{equation*}

\begin{tikzpicture}[remember picture,overlay]
\draw[blue,thick,->] (d4) to [in=90,out=245] + (225:2.7cm) node[anchor=north,text = black] {
	All states from the \emph{merge set} \dots
};
\draw[blue,thick,->] (d5) to [in=90,out=265] +(300:2.8cm) node[anchor=north,text = black] {
	\begin{tabular}{cc}
		\dots can be represented by any single \\
		one representative from the \emph{candidate set}.
	\end{tabular}
};
\end{tikzpicture}

\vfill
\end{frame}

\begin{frame}{Merger functions generalize quotient automata}
Special case: $\mu(M) = M$. \\
Remove all states from $M$ except for one (arbitrarily chosen) representative.

\vspace{.5cm}

For a congruence relation $\sim$, let $\mathfrak{C} \subseteq 2^Q$ be the equivalence classes. \\
The quotient automaton is defined by state set $\mathfrak{C}$. \\
This is captured by the merger function $\mu_\div : \mathfrak{C} \rightarrow 2^Q , \kappa \mapsto \kappa$.
\end{frame}




\section{Delayed Simulation}
\begin{frame}{Delayed Simulation}
\begin{defn}
	$p \equiv_\text{de} q$ iff for all $w \in \Sigma^*$, every run that starts in $\delta^*(p, w)$ or $\delta^*(q, w)$ eventually sees a priority of at most $\min \{c(\delta^*(p, w)), c(\delta^*(q, w))\}$.
\end{defn}
\end{frame}

\begin{frame}{Delayed Simulation}
\begin{figure}
	\centering
	\begin{tikzpicture}[shorten >=1pt,node distance=2cm,on grid,initial text=]
  	\node[state]           (0)                {$p$};
  	\node[state]           (1) [below=of 0]   {$q$};
  	\node           (2) [above right=of 0]   {\dots};
  	\node           (3) [below right=of 1]   {\dots};
  	\node[state]           (4) [below right=of 2]   {$p'$};
  	\node[state]           (5) [above right=of 3]   {$q'$};
  	\node			(6) [below=1cm of 0] {$p \equiv_\text{de} q$};
  	\node			(7) [below=1cm of 4] {$c(p') < c(q')$};
  	\node[state]	(8) [right=4cm of 5] {$q''$};
  	\node	(9) [above=1cm of 8] {$c(q'') \leq c(p')$};
  	\path[-] 	(0) edge [bend left] node [below right] {$w$} (2)
  				(1) edge [bend right] node [above right] {$w$} (3);
  	\path[->]   (2) edge [bend left] node {} (4)
  				(3) edge [bend right] node {} (5);
  	\draw[->,decorate,decoration={snake,amplitude=1mm,segment length=4mm,post length=1mm}] (5) -- node[below] {on all paths} (8);
	\end{tikzpicture}
\end{figure}
\end{frame}

\begin{frame}{Delayed Simulation}
\begin{defn}
	Let $\mathfrak{C}_\text{de} = \{ [q]_{\equiv_\text{de}} \mid q \in Q \}$ be the set of $\equiv_\text{de}$-equivalence classes. \\
	Define the \emph{delayed simulation merger} as $$\mu_\text{de} : \mathfrak{C}_\text{de} \rightarrow 2^Q, \kappa \mapsto \{ q \in \kappa \mid c(q) = \min c(\kappa) \}.$$
\end{defn}

\begin{theorem}
	Merging states according to $\mu_\text{de}$ preserves language.
\end{theorem}
\end{frame}


\begin{frame}{Computing Delayed Simulation}
We define a det.\ Büchi automaton $\mathcal{G}_\text{de}$ with states $q_\text{de}^0(p, q)$ such that: \\
	$p \equiv_\text{de} q$ iff both $L(\mathcal{G}_\text{de}, q_\text{de}^0(p, q))$ and $L(\mathcal{G}_\text{de}, q_\text{de}^0(q, p))$ are universal ($\Sigma^\omega$).
	
	\pause
	\vspace{.6cm}

$\mathcal{G}_\text{de} = (Q_\text{de}, \Sigma, \delta_\text{de}, F_\text{de})$

\begin{itemize}
	\item States are $Q_\text{de} = \alert<3>{Q} \times \alert<3>{Q} \times \alert<4>{(c(Q) \cup \{\checkmark\})}$. \\
The first two components are a \enquote{simulation} of the original DPA. \\
The third component are the so called \enquote{obligations}.
	\item Accepting states are $F_\text{de} = Q \times Q \times \{\checkmark\}$.
	\item Transitions $\delta_\text{de}$.
\begin{align*}
	\delta_\text{de}((p, q, k), a) = (& \alert<3>{\delta(p, a)}, \\
									& \alert<3>{\delta(q, a)}, \\
									& \alert<4>{\gamma( \quad c(\delta(p, a)), \quad c(\delta(q, a)), \quad k \quad )})
\end{align*}
\end{itemize}

\end{frame}


\begin{frame}{Delayed Simulation Automaton: $\gamma$}
	(Actual definition of $\gamma$ is more complex for some additional properties.) \\
	
	\vspace{.5cm}
	
	$\gamma : \mathbb{N} \times \mathbb{N} \times (\mathbb{N} \cup \{\checkmark\}) \rightarrow \mathbb{N} \cup \{\checkmark\}$
	
	\begin{align*}
		& \gamma(i, j, \checkmark) = \begin{cases}
			\checkmark & \text{if } j \leq i \\
			i & \text{else}
		\end{cases} \\
		\text{for } k \in \mathbb{N}: \quad & \gamma(i, j, k) = \begin{cases}
			\checkmark & \text{if } j \leq \min \{i, k\} \\
			\min \{i, k\} & \text{else}
		\end{cases}
	\end{align*}
	
	\vspace{.5cm}
	
	$q_\text{de}^0(p, q) = (p, q, \gamma(c(p), c(q), \checkmark))$.
\end{frame}

%
%\begin{frame}{Delayed Simulation Automaton: $\gamma$}
%	Let $0 \leq_\checkmark 1 \leq_\checkmark 2 \leq_\checkmark \dots \leq_\checkmark \checkmark$.
%
%	For $p, q \in Q$, $k \in c(Q) \cup \{\checkmark\}$, $a \in \Sigma$, set $\gamma((p, q, k), a) = \gamma'(\delta^*(p, a), \delta^*(q, a), k)$, where $\gamma'$ is defined as follows: \\
%	If any of the following is true, then $\gamma'(i, j, k) = \checkmark$.
%	
%	\begin{itemize}
%		\item $i$ is odd, $j$ is even, and $i \leq_\checkmark k$
%		\item $i$ is odd, $j$ is even, and $j \leq_\checkmark k$
%		\item $i$ is odd, $j$ is odd, $j \geq i$, and $i \leq_\checkmark k$
%		\item $i$ is even, $j$ is even, $j \leq i$, and $j \leq_\checkmark k$
%	\end{itemize}
%	
%	Otherwise, $\gamma'(i, j, k) = min_{\leq_\checkmark} \{ i,j,k \}$.
%	
%	$q_\text{de}^0(p, q) = (p, q, \gamma'(c(p), c(q), \checkmark))$.
%\end{frame}


\begin{frame}{Delayed Simulation Automaton}
\begin{figure}
\centering
\begin{tikzpicture}[shorten >=1pt,node distance=2cm,on grid,initial text=]
  \node[state]           (0)                {$q_0,1$};
  \node[state]           (1) [right=of 0]   {$q_1,1$};
  \node[state]           (2) [above=of 0]   {$q_2,1$};
  \node[state]           (3) [right=of 2]   {$q_3,1$};
  \node[state]           (4) [right=of 3]   {$q_4,0$};
  \path[->] (0) edge [bend left] node [above] {a} (1)
  			(0) edge node [left] {b} (2)
            (1) edge [bend left] node [below] {a} (0)
            (1) edge node [below] {b} (4)
            (2) edge [bend left] node [above] {a,b} (3)
            (3) edge [bend left] node [below] {a} (2)
            (3) edge [bend left] node [above] {b} (4)
            (4) edge [bend left] node [below] {a,b} (3);
\end{tikzpicture}
\end{figure}

A DPA with 5 states. We want to check whether $q_0 \equiv_\text{de} q_1$ is true. 
\end{frame}

\begin{frame}{Delayed Simulation Automaton}
\begin{figure}
\centering
\begin{tikzpicture}[shorten >=1pt,node distance=2.9cm,on grid,initial text=]
  \node[state,accepting,onslide=<2>{highlight}] (0)			    {$q_0,q_1,\checkmark$};
  \node[state,accepting] (1) [right=of 0]   {$q_2,q_4,\checkmark$};
  \node[state,accepting] (2) [right=of 1]   {$q_3,q_3,\checkmark$};
  \node[state,accepting,onslide=<3>{highlight}] (3) [below=of 0]   {$q_1,q_0,\checkmark$};
  \node[state,onslide=<4>{highlight}] 			 (4) [right=of 3]   {$q_4,q_2,0$};
  \node[state,onslide=<5>{highlight},onslide=<7>{highlight}] 			 (5) [right=of 4]   {$q_3,q_3,0$};
  \node[state,accepting] (6) [right=of 5]   {$q_4,q_4,\checkmark$};
  \node[state,onslide=<6-7>{highlight}] 			 (7) [below=of 5]   {$q_2,q_2,0$};
  \node[state,accepting] (8) [right=of 2]   {$q_2,q_2,\checkmark$};
  \path[->] (0) edge node [above] {b} (1)
  			(0) edge [bend left] node [right] {a} (3)
            (1) edge node [above] {a,b} (2)
            (2) edge node [above] {a} (6)
            (2) edge [bend left] node [above] {b} (8)
            (3) edge node [above] {b} (4)
            (3) edge [bend left] node [left] {a} (0)
            (4) edge node [above] {a,b} (5)
            (5) edge [bend left,onslide=<7>{highlight}] node [right] {a} (7)
            (5) edge node [above] {b} (6)
            (6) edge [bend left] node [right] {a,b} (2)
            (7) edge [bend left,onslide=<7>{highlight}] node [left] {a,b} (5)
            (8) edge node [above] {a,b} (2);
\end{tikzpicture}
\end{figure}


\begin{figure}
\centering
\begin{tikzpicture}[shorten >=1pt,node distance=1.4cm,on grid,initial text=,every node/.style={scale=0.7},overlay,remember picture]
  \node[state,anchor=south west,xshift=-6cm,yshift=2cm,onslide=<2>{highlightb},onslide=<3>{highlightg}]           (0)                {$q_0,1$};
  \node[state,onslide=<3>{highlightb},onslide=<2>{highlightg}]           (1) [right=of 0]   {$q_1,1$};
  \node[state,onslide=<6>{highlightb},onslide=<4>{highlightg}]           (2) [above=of 0]   {$q_2,1$};
  \node[state,onslide=<5>{highlightb}]           (3) [right=of 2]   {$q_3,1$};
  \node[state,onslide=<4>{highlightb}]           (4) [right=of 3]   {$q_4,0$};
  \path[->] (0) edge [bend left] node [above] {a} (1)
  			(0) edge node [left] {b} (2)
            (1) edge [bend left] node [below] {a} (0)
            (1) edge node [below] {b} (4)
            (2) edge [bend left] node [above] {a,b} (3)
            (3) edge [bend left] node [below] {a} (2)
            (3) edge [bend left] node [above] {b} (4)
            (4) edge [bend left] node [below] {a,b} (3);
\end{tikzpicture}
\end{figure}
\end{frame}


\begin{frame}{Complexity}
	$\mathcal{G}_\text{de}$ uses the state set $Q_\text{de} = Q \times Q \times (c(Q) \cup \{\checkmark\})$. \\
	Computing states of universal language in a DBA requires linear time.
	
	\vspace{.5cm}
	
	\begin{theorem}
		$\mu_\text{de}$ can be computed in $\mathcal{O}(n^2 k)$.
	\end{theorem}
	\small{$n = |Q|$, $k = |c(Q)|$}
\end{frame}





\section{Congruence Path Refinement}
\begin{frame}{Congruence Path Refinement}
	Idea: take a given relation and refine it until states can be merged.
	
	\vspace{1cm}

	\begin{defn}
	Let $\sim$ be a congruence relation and let $\lambda \subseteq Q$ be an equiv.\ class of $\sim$. \\
	We define $L_{\lambda \hookleftarrow} \subseteq \Sigma^*$ as the set of all words such that the induced run from a state in $\lambda$ moves back to $\lambda$ exactly once and ends there. 
	\end{defn}
\end{frame}	

\begin{frame}{Congruence Path Refinement}
\begin{figure}
\centering
\begin{tikzpicture}[shorten >=1pt,node distance=2cm,on grid,initial text=]
  \node[state]           (0)                {$q_0,1$};
  \node[state]           (1) [right=of 0]   {$q_1,1$};
  \node[state,onslide=<2>{highlight},onslide=<4>{highlight}]           (2) [above=of 0]   {$q_2,1$};
  \node[state,onslide=<3>{highlight}]           (3) [right=of 2]   {$q_3,1$};
  \node[state,onslide=<5>{highlight}]           (4) [right=of 3]   {$q_4,0$};
  \path[->] (0) edge [bend left] node [above] {a} (1)
  			(0) edge node [left] {b} (2)
            (1) edge [bend left] node [below] {a} (0)
            (1) edge node [below] {b} (4)
            (2) edge [bend left,onslide=<3>{highlight}] node [above] {a,b} (3)
            (3) edge [bend left,onslide=<4>{highlight}] node [below] {a} (2)
            (3) edge [bend left,onslide=<5>{highlight}] node [above] {b} (4)
            (4) edge [bend left] node [below] {a,b} (3);
\end{tikzpicture}
\end{figure}

$\lambda = \{q_2, q_4\}$ \\
\uncover<1-2>{Because $\sim$ is a congruence relation, we only need to consider one state. }

\vspace{.5cm}

$L_{\lambda \hookleftarrow} = \{ \alt<1-3>{}{\alt<4>{aa, ba}{aa, ba, ab, bb \}}}$
\end{frame}
	

\begin{frame}{Congruence Path Refinement}
	\begin{defn}
	The \emph{path refinement} equivalence $\equiv_\text{PR}^\lambda$ is the \emph{largest relation} s.t.: \\
	For $p, q \in \lambda$, $p \equiv_\text{PR}^\lambda q$ if and only if 
	\begin{itemize}
		\item $\forall w \in L_{\lambda \hookleftarrow}:$ the smallest priority seen when reading $w$ is the same from $p$ and from $q$.
		\item $\forall w \in L_{\lambda \hookleftarrow}: \delta^*(p, w) \equiv_\text{PR}^\lambda \delta^*(q, w)$
	\end{itemize}
	\end{defn}
	
	\vspace{.5cm}
	First point: Makes sure path segments have the same acceptance. \\
	Second point: Makes sure the argument can be applied repeatedly.
\end{frame}



\begin{frame}{Path Refinement Merger}
\begin{defn}
	Let $\mathfrak{C}_\text{PR}^\lambda = \{ [q]_{\equiv_\text{PR}^\lambda} \mid q \in Q \}$ be the set of $\equiv_\text{PR}^\lambda$-equivalence classes. Define the \emph{path refinement merger} as $\mu_\text{PR}^\lambda : \mathfrak{C}_\text{PR}^\lambda \rightarrow 2^Q, \kappa \mapsto \{ q \in \kappa \mid c(q) = \min c(\kappa) \}$.
\end{defn}

\begin{theorem}
	If all states in $\lambda$ are pairwise language equivalent, merging states according to $\mu_\text{PR}^\lambda$ preserves language.
\end{theorem}
\end{frame}


\begin{frame}{Computing Path Refinement}
	Define a DPA that relates Moore equivalence to $\equiv_\text{PR}$.
	\vspace{.2cm}
	\pause

	\begin{defn}
		Define the \emph{visit graph} DPA $\mathcal{A}_\text{visit}^\lambda = (Q_\text{visit}^\lambda, \Sigma, \delta_\text{visit}^\lambda, c_\text{visit}^\lambda)$.
		\begin{itemize}
			\item $Q_\text{visit}^\lambda = \alert<3>{Q} \times \alert<4>{c(Q)} \times \alert<5>{(c(Q) \cup \{-1\})}$
			\item $\delta^\lambda_\text{visit}((q, k, k'), a) = \begin{cases}
				(\alert<3>{q'}, \alert<4>{\min \{k, c(q')\}}, \alert<5>{-1}) & \text{if } q' \notin \lambda \\
				(\alert<3>{q'}, \alert<4>{c(q')}, \alert<5>{\min \{k, c(q')\}}) & \text{if } q' \in \lambda
			\end{cases}$, \\ where $\alert<3>{q' = \delta(q, a)}$.
			\item $c_\text{visit}^\lambda((q, k, \alert<5>{k'})) = \alert<5>{k'}$.
		\end{itemize}
	\end{defn}
	
	\vspace{.5cm}
	
	The \alert<3>{first} component \enquote{simulates} the original automaton $\mathcal{A}$. \\
	The \alert<4>{second} component tracks the minimal priority seen on one run from $\lambda$ to $\lambda$. \\
	The \alert<5>{third} component is required to distinguish the different priorities.
\end{frame}


\begin{frame}{Computing Path Refinement}
	\begin{defn}
		For $q \in Q$, we set $\iota_q := (q, c(q), \max c(Q)) \in Q_\text{visit}^\lambda$.
	\end{defn}
	
	\begin{theorem}
		$p \equiv_\text{PR}^\lambda q$ iff $\iota_p \equiv_M \iota_q$.
	\end{theorem}
	
	\begin{theorem}
		$\equiv_\text{PR}^\lambda$ can be computed in $\mathcal{O}(k^2 n \log n)$.
	\end{theorem}
	\small{$n = |Q|$, $k = |c(Q)|$}
\end{frame}


\begin{frame}{Visit Graph}
\begin{figure}
\centering
\begin{tikzpicture}[shorten >=1pt,node distance=2cm,on grid,initial text=]
  \node[state]           (0)                {$q_0,1$};
  \node[state]           (1) [right=of 0]   {$q_1,1$};
  \node[state]           (2) [above=of 0]   {$q_2,1$};
  \node[state]           (3) [right=of 2]   {$q_3,1$};
  \node[state]           (4) [right=of 3]   {$q_4,0$};
  \path[->] (0) edge [bend left] node [above] {a} (1)
  			(0) edge node [left] {b} (2)
            (1) edge [bend left] node [below] {a} (0)
            (1) edge node [below] {b} (4)
            (2) edge [bend left] node [above] {a,b} (3)
            (3) edge [bend left] node [below] {a} (2)
            (3) edge [bend left] node [above] {b} (4)
            (4) edge [bend left] node [below] {a,b} (3);
\end{tikzpicture}
\end{figure}

Potential choices for $\lambda$ are the equivalence classes of $\equiv_L$: \\ 
$\{q_0, q_1\}$, $\{q_2, q_4\}$, or $\{q_3\}$. 

\vspace{.5cm}

We take $\lambda = \{q_2, q_4\}$ and ask if $q_2 \equiv_\text{PR}^\lambda q_4$ is true.
\end{frame}

\begin{frame}{Visit Graph}

$\mathcal{A}_\text{visit}^{\{q_2, q_4\}}$ \\
$\iota_{q_2} = (q_2, 1, 1) \qquad \iota_{q_4} = (q_4, 0, 1)$ \hfill Question: $\iota_{q_2} \equiv_M \iota_{q_4}$?

\begin{figure}
\centering
\begin{tikzpicture}[shorten >=1pt,node distance=2.5cm,on grid,initial text=]
  \node[state,onslide=<4>{highlight}]           (0)                {$q_2,1,0$};
  \node[state,onslide=<3>{highlight}]           (1) [right=of 0]   {$q_3,0,-1$};
  \node[state,onslide=<3>{highlight}]           (2) [below=of 0]   {$q_3,1,-1$};
  \node[state]           (3) [right=of 2]   {$q_4,0,0$};
  \node[state,onslide=<2>{highlight},onslide=<4>{highlight}]           (4) [left=of 2]   {$q_2,1,1$};
  \node[state,onslide=<2>{highlight}]           (5) [right=of 1]   {$q_4,0,1$};
  \path[->] (0) edge node [left] {a,b} (2)
  			(1) edge [onslide=<4>{highlight}] node [above] {a} (0)
  			(1) edge [bend left] node [right] {b} (3)
  			(2) edge [bend left,onslide=<4>{highlight}] node [above] {a} (4)
  			(2) edge node [above] {b} (3)
  			(3) edge [bend left] node [left] {a,b} (1)
  			(4) edge [bend left,onslide=<3>{highlight}] node [above] {a,b} (2)
  			(5) edge [onslide=<3>{highlight}] node [above] {a,b} (1);
\end{tikzpicture}
\end{figure}

\alt<1-3>{(Reminder: the third component defines the color of a state)}{$q_2$ and $q_4$ are not PR-equivalent.}
\end{frame}






%\section{Labeled SCC Filter}
%
%\begin{frame}{Labeled SCC Filter}
%\begin{defn}[Threshold Moore relation]
%	Define $\equiv_M^{\leq k}$ as the Moore equivalence that considers all priorities \emph{greater than k} to be equal. \\
%	$p \equiv_M^{\leq k} q$ iff for all $w \in \Sigma^*$: $c(\delta^*(p, w)) = c(\delta^*(q, w))$ or $k < c(\delta^*(p, w)), c(\delta^*(q, w))$.
%\end{defn}
%\end{frame}
%
%
%\begin{frame}{Threshold Moore relation}
%\begin{figure}
%\centering
%\begin{tikzpicture}[shorten >=1pt,node distance=2cm,on grid,initial text=]
%  \node[state]           (0)                {$q_0,0$};
%  \node[state]           (1) [below left=of 0]   {\alt<1>{$q_1,2$}{$q_1,\top$}};
%  \node[state]           (2) [below right=of 0]   {\alt<1>{$q_3,4$}{$q_3,\top$}};
%  \node[state]           (3) [below=of 1]   {\alt<1>{$q_2,3$}{$q_2,\top$}};
%  \node[state]           (4) [below=of 2]   {\alt<1>{$q_4,5$}{$q_4,\top$}};
%  \path[->] (0) edge [bend left] node [below] {a} (2)
%  			(0) edge [bend left] node [above] {b} (1)
%  			(1) edge [bend left] node [above] {a} (0)
%  			(1) edge [bend left] node [right] {b} (3)
%  			(2) edge [bend left] node [below] {a} (0)
%  			(2) edge [bend left] node [right] {b} (4)
%  			(3) edge [bend left] node [left] {a} (1)
%  			(3) edge [loop left] node [left] {b} (3)
%  			(4) edge [bend left] node [left] {a} (2)
%  			(4) edge [loop right] node [right] {b} (4);
%\end{tikzpicture}
%\end{figure}
%
%In $\equiv_M$, every state is its own singleton class.
%
%\pause
%In $\equiv_M^{\leq 1}$, $q_1$ is equivalent to $q_3$ and $q_2$ is equivalent to $q_4$.
%\end{frame}
%
%\begin{frame}
%\begin{defn}[LSF relation]
%	Let $\sim$ be an equivalence relation and $k \in \mathbb{N}$. \\
%	We define $p \equiv^{k,\sim}_\text{LSF} q$ iff $p \sim q$ and $p \equiv_M^{\leq k} q$.
%\end{defn}
%\end{frame}
%
%
%\begin{frame}{LSF Relation}
%\begin{figure}
%\centering
%\begin{tikzpicture}[shorten >=1pt,node distance=2cm,on grid,initial text=]
%  \node[state]           (0)                {$q_0,0$};
%  \node[state,blue]           (1) [below left=of 0]   {$q_1,2$};
%  \node[state,blue]           (2) [below right=of 0]   {$q_3,4$};
%  \node[state,green]           (3) [below=of 1]   {$q_2,3$};
%  \node[state,green]           (4) [below=of 2]   {$q_4,5$};
%  \path[->] (0) edge [bend left] node [below] {a} (2)
%  			(0) edge [bend left] node [above] {b} (1)
%  			(1) edge [bend left] node [above] {a} (0)
%  			(1) edge [bend left] node [right] {b} (3)
%  			(2) edge [bend left] node [below] {a} (0)
%  			(2) edge [bend left] node [right] {b} (4)
%  			(3) edge [bend left] node [left] {a} (1)
%  			(3) edge [loop left] node [left] {b} (3)
%  			(4) edge [bend left] node [left] {a} (2)
%  			(4) edge [loop right] node [right] {b} (4);
%\end{tikzpicture}
%\end{figure}
%
%All five states are language equivalent to each other.
%
%\vspace{.2cm}
%
%Equivalence classes of $\equiv_\text{LSF}^{1,\equiv_L}$: $\{q_0\}$, $\{q_1, q_3\}$, and $\{q_2, q_4\}$.
%\end{frame}
%
%
%\begin{frame}{LSF Merger}
%From $\mathcal{A}$ to $\mathcal{A}_k$: \\
%remove all states which have \emph{priority less or equal} to $k$.
%
%\vspace{.5cm}
%
%Build a total preorder $\preceq_k$ on $\mathcal{A}_k$ such that $q$ being reachable from $p$ implies $p \preceq_k q$. (weaker form of exact reachability)
%
%\vspace{.5cm}
%
%In focus are the set of states that are $\preceq_k$-maximal among a given set $P \subseteq Q$. These are all states in one SCC of $\mathcal{A}_k$ such that no other states in $P$ are reachable.
%\end{frame}
%
%
%\begin{frame}{$\mathcal{A}_1$ example}
%\begin{figure}
%\centering
%\alt<1>{
%\begin{tikzpicture}[shorten >=1pt,node distance=2cm,on grid,initial text=]
%  \node[state]           (0)                {$q_0,0$};
%  \node[state]           (1) [below left=of 0]   {$q_1,2$};
%  \node[state]           (2) [below right=of 0]   {$q_3,4$};
%  \node[state]           (3) [below=of 1]   {$q_2,3$};
%  \node[state]           (4) [below=of 2]   {$q_4,5$};
%  \path[->] (0) edge [bend left] node [below] {a} (2)
%  			(0) edge [bend left] node [above] {b} (1)
%  			(1) edge [bend left] node [above] {a} (0)
%  			(1) edge [bend left] node [right] {b} (3)
%  			(2) edge [bend left] node [below] {a} (0)
%  			(2) edge [bend left] node [right] {b} (4)
%  			(3) edge [bend left] node [left] {a} (1)
%  			(3) edge [loop left] node [left] {b} (3)
%  			(4) edge [bend left] node [left] {a} (2)
%  			(4) edge [loop right] node [right] {b} (4);
%\end{tikzpicture}
%}{
%\begin{tikzpicture}[shorten >=1pt,node distance=2cm,on grid,initial text=]
%  \node          (0)                {};
%  \node[state]           (1) [below left=of 0]   {$q_1,2$};
%  \node[state]           (2) [below right=of 0]   {$q_3,4$};
%  \node[state]           (3) [below=of 1]   {$q_2,3$};
%  \node[state]           (4) [below=of 2]   {$q_4,5$};
%  \path[->] (1) edge [bend left] node [right] {b} (3)
%  			(2) edge [bend left] node [right] {b} (4)
%  			(3) edge [bend left] node [left] {a} (1)
%  			(3) edge [loop left] node [left] {b} (3)
%  			(4) edge [bend left] node [left] {a} (2)
%  			(4) edge [loop right] node [right] {b} (4);
%\end{tikzpicture}
%}
%\end{figure}
%
%\only<1>{Remove all states with priority $\leq 1$.}
%\only<2->{$\preceq_1$ can be one of two relations: \\
%$\{q_1, q_2\} \prec_1 \{q_3, q_4\}$; or \\
%$\{q_3, q_4\} \prec_1 \{q_1, q_2\}$
%}
%\end{frame}
%
%
%\begin{frame}{LSF Merger}
%\begin{defn}
%	Let $\mathfrak{C}_\text{LSF}^{k,\sim}$ be the set of equivalence classes in $\equiv_\text{LSF}^{k,\sim}$ and let $\kappa$ be such an equivalence class. Define
%	$$C_\kappa^k = \{ r \in \kappa \mid c(r) > k \text{ and } r \text{ is } \preceq_k \text{-maximal among } \kappa \}$$
%	and 
%	$M_\kappa^k = \kappa \setminus C_\kappa^k$.
%\end{defn}
%
%\begin{defn}
%	Define the \emph{LSF merger function} $$\mu_\text{LSF}^{k,\sim} : \{ M_\kappa^k \mid \kappa \in \mathfrak{C}_\text{LSF}^{k,\sim} \} \rightarrow 2^Q , M_\kappa^k \mapsto C_\kappa^k$$
%\end{defn}
%\end{frame}
%
%
%\begin{frame}{LSF example}
%\begin{figure}
%\centering
%\begin{tikzpicture}[shorten >=1pt,node distance=2cm,on grid,initial text=]
%  \node[state]           (0)                {$q_0,0$};
%  \node[state]           (1) [below left=of 0]   {$q_1,2$};
%  \node[state]           (2) [below right=of 0]   {$q_3,4$};
%  \node[state]           (3) [below=of 1]   {$q_2,3$};
%  \node[state]           (4) [below=of 2]   {$q_4,5$};
%  \path[->] (0) edge [bend left] node [below] {a} (2)
%  			(0) edge [bend left] node [above] {b} (1)
%  			(1) edge [bend left] node [above] {a} (0)
%  			(1) edge [bend left] node [right] {b} (3)
%  			(2) edge [bend left] node [below] {a} (0)
%  			(2) edge [bend left] node [right] {b} (4)
%  			(3) edge [bend left] node [left] {a} (1)
%  			(3) edge [loop left] node [left] {b} (3)
%  			(4) edge [bend left] node [left] {a} (2)
%  			(4) edge [loop right] node [right] {b} (4);
%\end{tikzpicture}
%\end{figure}
%
%Equivalence classes of $\equiv_\text{LSF}^{1,\equiv_L}$: $\{q_0\}$, $\{q_1, q_3\}$, and $\{q_2, q_4\}$. 
%
%\vspace{.2cm}
%
%State order $\{q_1, q_2\} \prec_1 \{q_3, q_4\}$ 
%
%\vspace{.2cm}
%
%$C^1_{\{q_0\}} = \emptyset$ \qquad
%$C^1_{\{q_1, q_3\}} = \{q_3\}$ \qquad
%$C^1_{\{q_2, q_4\}} = \{q_4\}$
%
%\vspace{.1cm}
%
%$M^1_{\{q_0\}} = \emptyset$ \qquad
%$M^1_{\{q_1, q_3\}} = \{q_1\}$ \qquad
%$M^1_{\{q_2, q_4\}} = \{q_2\}$
%\end{frame}
%
%
%
%\begin{frame}{LSF example}
%After the merge:
%
%\begin{figure}
%\centering
%\begin{tikzpicture}[shorten >=1pt,node distance=2cm,on grid,initial text=]
%  \node[state]           (0)                {$q_0,0$};
%  \node[state]           (2) [below=of 0]   {$q_3,4$};
%  \node[state]           (4) [below=of 2]   {$q_4,5$};
%  \path[->] (0) edge [bend left] node [right] {a,b} (2)
%  			(2) edge [bend left] node [left] {a} (0)
%  			(2) edge [bend left] node [right] {b} (4)
%  			(4) edge [bend left] node [left] {a} (2)
%  			(4) edge [loop right] node [right] {b} (4);
%\end{tikzpicture}
%\end{figure}
%\end{frame}
%
%\begin{frame}{Language \& Computing LSF}
%\begin{theorem}
%	If $\sim$ implies language equivalence, merging states according to $\mu_\text{LSF}^{k,\sim}$ preserves language.
%\end{theorem}
%
%\pause
%\vspace{1cm}
%
%	\begin{theorem}
%		$\mu_\text{LSF}^{k,\sim}$ can be computed in $\mathcal{O}(n \log n)$.
%	\end{theorem}
%	
%$\equiv_M^{\leq k}$ is only a slight variation of the normal Moore equivalence. \\
%$\preceq_k$ can be computed with a topological sorting on the SCCs of $\mathcal{A}_k$.
%\end{frame}







\section{Efficiency}

\begin{frame}{Delayed Simulation}
\begin{figure}
	\centering
	\includegraphics[page=6,height=.8\textheight]{../data/analysis/fritzwilke/detnbaut_ap1.pdf} 
\end{figure}
\end{frame}

\begin{frame}{Path Refinement}
\begin{figure}
	\centering
	\includegraphics[page=6,height=.8\textheight]{../data/analysis/path_refinement/detnbaut_ap1.pdf} 
\end{figure}
\end{frame}
%
%\begin{frame}{LSF}
%\begin{figure}
%	\centering
%	\includegraphics[page=6,height=.8\textheight]{../data/analysis/lsf/detnbaut_ap1.pdf} 
%\end{figure}
%\end{frame}

\begin{frame}{Summary}
\begin{tabular}{ccccc}
\begin{minipage}{0.32\textwidth}
	\begin{itemize}
	\item<alert@1> Moore \\
	\item Skip \\
	\item<alert@1> Delayed Simulation \\
	\item Iterated Moore \\
	\item<alert@1> Path Refinement \\
	\item Threshold Moore \\
	\item LSF
	\end{itemize}
\end{minipage}
&
$\Rightarrow$ 
&
\begin{minipage}{0.11\textwidth}
	\alert{Merger function}
\end{minipage}
&
$\Rightarrow$
&
\begin{minipage}{0.3\textwidth}
	\begin{itemize}
	\item<alert@1> Representative merge
	\item Schewe merge
	\end{itemize}
\end{minipage}
\end{tabular}






\end{frame}










