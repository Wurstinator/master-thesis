\section{Threshold Moore}
The following approach is based on the idea of intersecting two equivalence relations which by themselves are too weak to allow for merging of states, but together then give a strong enough guarantee of similarity between the states. One of the two relations is one that implies language equivalence, similar to the previous path refinement algorithm. The second relation is a modification of Moore equivalence that only considers state priorities up to a certain upper limit.

\begin{definition}
	We define the \emph{threshold Moore equivalence relation} as $p \equiv_M^{\leq k} q$ if and only if for all finite words $w$, $\delta^*(p, w)$ and $\delta^*(q, w)$ have the same priority or both priorities are greater than $k$.
	
	Let $\sim$ be an equivalence relation that implies language equivalence. We define the \emph{TM equivalence relation} as $p \equiv_\text{TM}^\sim q$ if and only if $p \sim q$, $c(p) = c(q)$, and $p \equiv_M^{\leq c(p)} q$.
\end{definition}

We make the implicit assumption here that the relation $\sim$ is well-defined overspanning different automata: for two DPAs $\mathcal{A}$ and $\mathcal{A}'$ so that $Q'$ is a subset of $Q$ and for all states $q \in Q'$, both automata accept the same language starting from $q$, the relation $\sim$ considers two states to be equivalent in one automaton if and only if it does so in the other.

This assumption makes sure that if we merge some states while preserving language in the automaton, the $\sim$ relation on the untouched states does not change.

For the TM relation, the merger function is actually quite simple and simply merges classes of $\equiv_\text{TM}^\sim$. Note that this is not, however, a quotient automaton, as the TM relation is in general not a congruence relation.

\begin{definition}
	We define the \emph{TM merger function} $\mu_\text{TM}^\sim : \mathfrak{C}(\equiv_\text{TM}^\sim) \rightarrow 2^Q$ as $\mu_\text{TM}^\sim(\kappa) = \kappa$.
\end{definition}

\begin{lemma}
	Let $\mathcal{A}$ be a DPA and let $\mathcal{A}'$ be a representative merge w.r.t.\ some equivalence class $\kappa \in \mathfrak{C}(\mu_\text{TM}^\sim)$. Let $k$ be the priority of all states in $\kappa$. For all states $p$ and $q$ in $\mathcal{A}'$, the two states are $\equiv_L$-equivalent in $\mathcal{A}$ if and only if they are in $\mathcal{A}'$. If $k \geq c(p), c(q)$, then the same holds for $\equiv_M^{\leq k}$.
	\label{lem:threshm_kappamerge}
\end{lemma}

\begin{proof}
	We focus on the language equivalence first. Let $\rho$ and $\rho'$ be the runs of the two automata on some word $\alpha$ starting in some state $q_0$. We show that these two runs have the same acceptance status. Then the claim of the Lemma follows by transitivity of $\equiv_L$.
	
	$\equiv_L$ is a congruence relation, so we have $\rho(i) \equiv_L \rho'(i)$ for all positions $i$. The same is true for $\equiv_M^{\leq k}$.
	
	If $c(\rho)$ visits infinitely priorities of at most $k$, then the two runs will see the same smallest priority $l < k$ infinitely often, as $c(\rho(i)) = l$ if and only if $c'(\rho'(i)) = l$. Thus, they must have the same acceptance status.
	
	If $c(\rho)$ only visits finitely many priorities of at most $k$, then from some point $j$ on in $\rho'$, only transitions that also exist in $\mathcal{A}$ are taken. As $\rho(j) \equiv_L \rho'(j)$, that means the two runs have the same acceptance status.
	
	\vspace{5pt}
	
	Regarding the second part of the Lemma, let $q$ be a state with $k \geq c(q)$ and let $\rho$ and $\rho'$ be the runs of the two automata on some $\alpha$ starting in $q$. As $\equiv_M^{\leq k}$ is a congruence relation, $\rho(i) \equiv_M^{\leq k} \rho'(i)$ for all positions $i$. 
	
	Let $p$ be a second state with $k \geq c(p)$ and let $\pi$ and $\pi'$ be the corresponding runs. Towards a contradiction, assume that the two states are $\equiv_M^{\leq k}$-equivalent in $\mathcal{A}$ but not in $\mathcal{A}'$. The other direction works symmetrically. Then there is a position $j$ such that $c'(\pi'(j)) \neq^{\leq k} c'(\rho'(j))$ but $c(\pi(j)) =^{\leq k} c(\rho(j))$. Since $\pi(j) \equiv_M^{\leq k} \pi'(j)$ and $\rho(j) \equiv_M^{\leq k} \rho'(j)$, this violates transitivity of $\equiv_M^{\leq k}$ and finishes our contradiction. \qed
\end{proof}

\begin{theorem}
	A representative merge of a DPA w.r.t.\ $\mu_\text{TM}^\sim$ is language equivalent to the original.
\end{theorem}

\begin{proof}
	Let $\kappa_1, \dots, \kappa_m$ be an enumeration of the equivalence classes in $\mu_\text{TM}^\sim$ sorted by descending priority. With Lemma \ref{lem:threshm_kappamerge} and our assumption that $\sim$ behaves well with state merges, merging the states in $\kappa_i$ will not change the equivalence classes $\kappa_{i+1}, \dots, \kappa_n$. It is therefore a language preserving operation to merge all equivalence classes in the given order. The resulting automaton is the same as a representative merge w.r.t.\ $\mu_\text{TM}^\sim$. \qed
\end{proof}


The computation of $\mu_\text{TM}^\sim$ is rather straight forward. 

\begin{theorem}
	For a given $\sim$ in a suitable data structure, $\mu_\text{TM}^\sim$ can be computed in $\mathcal{O}(|Q| \cdot |c(Q)| \cdot \log |Q|)$.
\end{theorem}

\begin{proof}
	Assuming that $\equiv_\text{TM}^\sim$ is known, computing $\mu_\text{TM}^\sim$ is easy. $\equiv_\text{TM}^\sim$ is an intersection of three equivalence relations, so the total run time is the sum of the run time to compute each of the three parts.
	
	$\sim$ is already given, so no time is needed. Computing the relation of equal priorities is also doable in $\mathcal{O}(|Q|)$, assuming suitable data structures in the automaton. Finally, $\equiv_M^{\leq k}$ can be computed with just a slight adaption of usual algorithms for Moore equivalence in time $\mathcal{O}(|Q| \cdot \log |Q|)$. This needs to be done for every $k$, so $|c(Q)|$ times.
\end{proof}










