
\section{Path Refinement}

The upcoming technique, called congruence path refinement or just path refinement, is again one which takes in an already defined relation on the states and refines it to a point where merging classes is a valid operation. To be precise, we define the PR-equivalence for each equivalence class independently. 
Given such a class, which we call $\lambda$, we consider \enquote{loops} which move from $\lambda$ back to some state in $\lambda$ via some finite path. If two states in $\lambda$ see the same smallest priority on every such loop and we can guarantee that all states they will reach via the loop have the same property, then every run that visits $\lambda$ infinitely often can be divided into path segments on which runs starting in these two states see the same smallest priority. As this then holds for every segment in the run, the set of infinitely often occurring priorities must then also have the same minimum.

Formally, we describe these loops in a set called $L_{\lambda \hookleftarrow}$. This is the set of all finite words, such that reading them from $\lambda$ will move back to $\lambda$ at the end and never before.

\begin{definition}
	Let $\sim$ be a congruence relation that implies language equivalence and let $\lambda \in \mathfrak{C}(\sim)$ be an equivalence class. We define a relation $R_\lambda$ on $\lambda$ as $(p, q) \in R_\lambda$ if and only if for all $w \in L_{\lambda \hookleftarrow}$, the smallest priority seen on the path induced by $w$ is the same starting from $p$ and from $q$.
	
	We define \emph{path refinement equivalence} $\equiv_\text{PR}^\lambda$ on $\lambda$ as the largest subset of $R_\lambda$ such that $p \equiv_\text{PR}^\lambda q$ if and only if for all $w \in L_{\lambda \hookleftarrow}$, $\delta^*(p, w) \equiv_\text{PR}^\lambda \delta^*(q, w)$.
\end{definition}

Again, note that this relation is only defined on $\lambda$. Using the merger function below, one can perform this reduction independently on every class in $\mathfrak{C}(\sim)$. 

\begin{definition}
	We define the \emph{path refinement merger function} $\mu_\text{PR}^\lambda : \mathfrak{C}(\equiv_\text{PR}^\lambda) \rightarrow 2^Q$ with $\mu_\text{PR}^\lambda(\kappa) = \{q \in \kappa \mid c(q) = \min c(\kappa) \}$.
\end{definition}

\begin{theorem}
	A representative merge of a DPA w.r.t.\ $\mu_\text{PR}^\lambda$ is language equivalent to the original.
\end{theorem}

\begin{proof}
	Let $\mathcal{A}'$ be the representative merge. Assume there is a starting state $q_0 \in Q'$ and a word $\alpha$ such that the acceptance of the runs $\rho$ and $\rho'$ of the two DPAs differs. We will bring this assumption to a contradiction.
	
	First, note that at every position $i$, $\rho(i)$ and $\rho'(i)$ must be $\sim$-equivalent, as $\sim$ is a congruence relation. If in these runs, $\lambda$ is visited only finitely often, there is a position $j$ at which it is visited for the last time. Then from $j$ on, $\rho'$ only uses transitions that also exist in the original DPA $\mathcal{A}$. As $\rho(j) \sim \rho'(j)$, they must be language equivalent and therefore have the same acceptance status. This contradicts the assumption.
	
	Otherwise, $\lambda$ is visited infinitely often. However, for two consecutive positions $k$ and $k'$ at which $\lambda$ is seen, we can show that the smallest priorities in $c(\rho(k)), \dots, c(\rho(k'))$ and $c'(\rho'(k)), \dots, c'(\rho'(k'))$ are the same. Then it easily extends that the entire runs share the same smallest priority that is seen infinitely often.
	
	To observe that the two run segments see the same minimal priority, first observe that $\rho(k) \equiv_\text{PR}^\lambda \rho'(k)$ by induction on $k$. If $k$ is the first position at which $\lambda$ is visited, then $\rho'(k)$ is the representative of the equivalence class of $\rho(k)$ and therefore $\equiv_\text{PR}^\lambda$-equivalent to $\rho(k)$. Then by definition of the path refinement equivalence, the same holds for $k'$ and therefore all following positions.
	
	Now that we have established $\rho(k) \equiv_\text{PR}^\lambda \rho'(k)$, it follows directly from the definition of PR-equivalence that the smallest priorities in $c(\rho(k)), \dots, c(\rho(k'))$ and $c'(\rho'(k)), \dots, c'(\rho'(k'))$ are equal. \qed
\end{proof}

As for delayed simulation, the definition of path refinement is more theoretical and less constructive. We have to dedicate some additional thought to the question of how to actually compute $\mu_\text{PR}^\lambda$. We use a similar approach as before and reduce the computation of PR-equivalence to a known automata problem, which in this case is the Moore equivalence on DPAs.

A direct translation of the definition to an algorithm would be a similar automaton as for delayed simulation. One can build a deterministic finite product automaton with an additional third component that tracks the smallest priority so far and which component it was seen in, and at every visit to $\lambda$ makes sure that the tracked values coincide.

The \emph{visit graph} that we now define instead is a less intuitive solution but has a size only linear in $|Q|$ instead of quadratic. It also uses a \enquote{tracker} of the smallest priorities between one visit to $\lambda$ and the next but only does so for each state individually instead of tracking each state pair.

\begin{definition}
	The visit graph is a DPA $(Q_\text{visit}^\lambda, \Sigma, \delta_\text{visit}^\lambda, c_\text{visit}^\lambda)$ defined by
	\begin{itemize}
		\item $Q_\text{visit}^\lambda = Q \times c(Q) \times (c(Q) \cup \{\perp\})$
		\item $delta_\text{visit}^\lambda((q, k, k'), a) = \begin{cases}
			(q', \min \{c(q'), k\}, \perp) & \text{if } q' \notin \lambda \\
			(q', c(q'), \min \{c(q'), k\}) & \text{if } q' \in \lambda
		\end{cases}$, where $q' = \delta(q, a)$
		\item $c_\text{visit}((q, k, k')) = k'$
	\end{itemize}
\end{definition}

\begin{theorem}
	For a state $q \in Q$, let $\iota_q = (q, c(q), \max c(Q)) \in Q_\text{visit}^\lambda$. Then $p \equiv_\text{PR}^\lambda q$ if and only if $\iota_p \equiv_M \iota_q$.
\end{theorem}

\begin{proof}
	Our first observation is that for any state $p \in \lambda$, reading some $w \in L_{\lambda \hookleftarrow}$ from $(p, c(p), k)$ ends in $(q, c(q), k')$ where $k'$ is the smallest priority that occurs on the run segment.
	
	If $p \not\equiv_\text{PR}^\lambda q$, then there is a $w \in L_{\lambda \hookleftarrow}$ such that either the smallest priority when reading $w$ from $p$ and $q$ differs, or reading $w$ moves to non-PR-equivalent states. If the former is true, then reading $w$ from $\iota_p$ and $\iota_q$ brings the visit graph to state with different priorities and therefore $\iota_p \not\equiv_M \iota_q$. If the former is false and the latter is true, then one has to repeatedly apply this argument until at some point a state pair is reached at which the first case is violated.
	
	For the other direction, if $\iota_p \not\equiv_M \iota_q$, there must be a $w \in \Sigma^*$ such that the priority differs when reading $w$ from $\iota_p$ and $\iota_q$. As all states not in $\lambda$ have the same priority, we can split $w = v_1 \dots v_n$ such that all $v_i$ are words in $L_{\lambda \hookleftarrow}$. Then on the last segment, reading $v_n$ sees different minimal priorities from the initial states and therefore $p$ and $q$ cannot be PR-equivalent.
\end{proof}

\begin{theorem}
	$\equiv_\text{PR}^\lambda$ can be computed in $\mathcal{O}(|Q| \cdot |c(Q)|^2 \cdot \log |Q|)$.
\end{theorem}

\begin{proof}
	Moore equivalence can be computed in time $\mathcal{O}(n \log n)$ of its automaton (\cite{Hopcroft1971}). The visit graph has size $n \in \mathcal{O}(|Q| \cdot |c(Q)|^2)$. As $|c(Q)|$ is always at most $|Q|$, this gives us the desired complexity.
\end{proof}



\begin{figure}
	\centering
	\begin{tikzpicture}[shorten >=1pt,node distance=2cm,on grid,initial text=]
	\node[state]           (0)                {$q_0,0$};
	\node[state]           (1) [below=of 0]   {$q_1,0$};
	\node[state]           (2) [right=of 0]   {$q_2,0$};
	\node[state]           (3) [below=of 2]   {$q_3,1$};
	\path[->] 
	(0) edge [bend left] node [left] {a} (1)
	(0) edge [bend left] node [above] {b} (2)
	(0) edge [loop left] node {c} (0)
	(1) edge [bend left] node [left] {a} (0)
	(1) edge [bend left] node [above] {b} (3)
	(1) edge [loop left]node {c} (1)
	(2) edge [bend left] node [right] {a} (3)
	(2) edge [loop right] node {b} (2)
	(2) edge [bend left] node [above] {c} (0)
	(3) edge [loop right] node {a} (3)
	(3) edge [bend left] node [right] {b} (2)
	(3) edge [bend left] node [above] {c} (1);
	\end{tikzpicture}
	\caption{Example PR}
	\label{fig:example_pr}
\end{figure}
